\documentclass[a4paper,11pt]{article}

% ===================================================================
% NUOVO STILE: "La Bibbia" (Light Theme + Orange + JS Aesthetics)
% ===================================================================

% --- Pacchetti Fondamentali ---
\usepackage[utf8]{inputenc}
\usepackage[T1]{fontenc}
\usepackage[italian]{babel}
\usepackage{geometry}
\geometry{top=2.5cm, bottom=2.5cm, left=2.5cm, right=2.5cm}

% --- Font Moderni ---
\usepackage{lmodern}
\usepackage{helvet}
\renewcommand{\familydefault}{\sfdefault} % Usa sans-serif per tutto il documento

% --- Colori e Grafica ---
\usepackage{xcolor}
\usepackage{tikz}
\usepackage{amssymb}
\usepackage{amsmath} % Aggiunto per supporto matematico
\usepackage{amsthm}  % Aggiunto per proof e teoremi

\usepackage{graphicx} % Aggiunto per immagini
\usepackage{wrapfig}  % Aggiunto per compatibilità
\usepackage{caption} 
\usepackage{subcaption}

% --- Pacchetti Utility (Compatibilità Vecchio Testo) ---
\usepackage{multicol}
\usepackage{soul}
\usepackage{enumitem}
\usepackage{lipsum}
\usepackage{mwe}
\usepackage{multirow}
\usetikzlibrary{shapes, arrows.meta, positioning, chains, fit, calc, matrix, trees, backgrounds, decorations.pathmorphing, decorations.pathreplacing, snakes}

% --- Definizione Colori (Custom 5-Color Palette)
\definecolor{paletteIndigo}{HTML}{2D1E64}   % Dark Backgrounds
\definecolor{paletteRegalia}{HTML}{592F83}  % Primary Brand / Headers
\definecolor{paletteLenurple}{HTML}{B795D7} % Secondary Accent (Light Purple)
\definecolor{paletteLavender}{HTML}{EADCFA} % Box Backgrounds
\definecolor{paletteWenge}{HTML}{69525D}    % Text / Details

% --- Mapping Colors to Layout ---
\colorlet{mainOrange}{paletteRegalia}       % Sections / Main Theme
\colorlet{jsYellow}{paletteLenurple}        % Accents (Replacing Gold/Yellow)
\colorlet{jsDark}{paletteIndigo}            % Sidebar / Dark Elements
\colorlet{codeBg}{white}          % Box Backgrounds (White as requested)
\colorlet{niceBlue}{paletteRegalia}         % Concept Box Frames
\colorlet{textGray}{paletteWenge}           % Meta Text
\colorlet{tsBlue}{paletteRegalia}           % Misc Blue elements

\definecolor{codeBlue}{HTML}{0000FF}
\definecolor{codePink}{HTML}{800080}
\definecolor{codeGreen}{HTML}{008000}
% --- Colori Legacy (Compatibilità) ---
\definecolor{codegray}{named}{white}
\definecolor{commentpurple}{rgb}{0.5,0,0.5}
\definecolor{stringred}{rgb}{0.8,0,0}

% --- COMANDO COPERTINA (Pure TikZ - Left Aligned) ---
\newcommand{\customcover}[2]{
    \begin{titlepage}
    \thispagestyle{empty}
    \begin{tikzpicture}[remember picture, overlay]
        
        % 1. Sfondo Laterale (Indigo)
        \fill[jsDark] (current page.north west) rectangle ($(current page.south west) + (4cm,0)$);
        
        % 2. Linea Accento (Lenurple)
        \draw[jsYellow, line width=4pt] ($(current page.north west) + (4.2cm,-2cm)$) -- ($(current page.south west) + (4.2cm,2cm)$);
        
        % 3. Contenuto Testuale (ALLINEATO A SINISTRA)
        \node[anchor=north west, inner sep=0, text width=14cm, align=flush left] at ($(current page.north west) + (6cm,-5cm)$) {
            
                {\fontsize{18}{22}\selectfont \textcolor{textGray}{Università di Pisa}}\\[0.2cm]
                {\fontsize{16}{20}\selectfont \textcolor{textGray}{Corso di Laurea in Informatica}}\\[2cm]
                
                {\fontsize{60}{70}\selectfont \textbf{\textcolor{jsDark}{LA BIBBIA Di }}}\\[0.5cm]
                %{\fontsize{35}{45}\selectfont \textbf{\textcolor{black}{DI}}}\\[0.2cm]
                {\fontsize{35}{45}\selectfont \textbf{\textcolor{black}{PROGRAMMAZIONE}}}\\[0.2cm]
                {\fontsize{35}{45}\selectfont \textbf{\textcolor{black}{ED ALGORITMICA}}}\\[1.5cm]
                
                % Linea divisoria (Sinistra)
                \textcolor{jsYellow}{\rule{8cm}{4pt}}\\[1cm]
                
                {\Large \textbf{#1}}\\[0.2cm]
                {\large \textit{#2}}\\[1.5cm]
                
                {\large Docenti titolari:}\\[0.2cm]
                {\Large \textbf{[Nome Docente]}}\\[2cm]
                
                \textbf{\large Autore del riassunto:}\\
                {\large Joseph Zucchelli}\\[2cm]
                
                \vfill
                {\large \textbf{Anno Accademico 2024 - 2025}}
            
        };
        
        % Badge (Opzionale)
        \node[anchor=south east] at ($(current page.south east) + (-2cm, 2cm)$) {
            \tikz\draw[fill=jsYellow, draw=none] (0,0) rectangle (2,2) node[midway, black, font=\bfseries\huge] {P\&A};
        };

    \end{tikzpicture}
    \end{titlepage}
    \clearpage
}

% --- Box per il Codice e Note (tcolorbox) ---
\usepackage[most]{tcolorbox}
\usepackage{listings}
\usepackage{titlesec} % Per colorare i titoli
\usepackage[pdfusetitle, bookmarksopen=true, bookmarksnumbered=true]{hyperref}
\hypersetup{
    colorlinks=true,
    linkcolor=mainOrange,
    citecolor=niceBlue,
    urlcolor=niceBlue,
    linktoc=all
}


% Setup Titoli Arancioni
\titleformat{\section}
{\color{mainOrange}\normalfont\Large\bfseries}
{\color{mainOrange}\thesection}{1em}{}[\color{mainOrange}\hrule height 1pt]

\titleformat{\subsection}
{\color{mainOrange}\normalfont\large\bfseries}
{\color{mainOrange}\thesubsection}{1em}{}

% Stile per il codice JS
\lstdefinelanguage{JavaScript}{
  keywords={typeof, new, true, false, catch, function, return, null, catch, switch, var, if, in, while, do, else, case, break, const, let},
  keywordstyle=\color{green!60!black}\bfseries,
  ndkeywords={class, export, boolean, throw, implements, import, this, console, log, prompt, alert},
  ndkeywordstyle=\color{green!60!black}\bfseries,
  identifierstyle=\color{black}, 
  sensitive=false,
  comment=[l]{//},
  morecomment=[s]{/*}{*/},
  commentstyle=\color{teal}\itshape,
  stringstyle=\color{red!60!black}\ttfamily,
  morestring=[b]',
  morestring=[b]"
}

% Box Codice
\newtcblisting{jsbox}[1][]{
  colback=codeBg,
  colframe=gray!50,
  listing only,
  listing options={
    language=JavaScript,
    basicstyle=\ttfamily\small, % Testo nero di default
    numbers=left,
    numberstyle=\tiny\color{gray},
    breaklines=true,
    columns=fullflexible
  },
  title={\textbf{Codice}},
  coltitle=black,
  fonttitle=\bfseries\ttfamily,
  enhanced,
  attach boxed title to top right={yshift=-3mm, xshift=-3mm},
  boxed title style={colback=jsYellow},
  #1
}

% Box Funzioni Viola
\newtcblisting{funcbox}[1][]{
  colback=violet!5,        % Sfondo Viola chiarissimo
  colframe=violet!75,      % Bordo Viola
  listing only,
  listing options={
    language=JavaScript,
    basicstyle=\ttfamily\small,
    numbers=left,
    numberstyle=\tiny\color{gray},
    breaklines=true,
    columns=fullflexible
  },
  title={\textbf{Funzione: #1}},
  coltitle=white,
  fonttitle=\bfseries\ttfamily,
  enhanced,
  attach boxed title to top left={yshift=-3mm, xshift=3mm},
  boxed title style={colback=violet}, 
  drop shadow,
  #1
}

% Box Concetti/Note
\newtcolorbox{concept}[1]{
  colback=niceBlue!5,
  colframe=niceBlue,
  title={#1},
  fonttitle=\bfseries,
  sharp corners=downhill,
  enhanced,
  drop shadow
}

% --- COMPATIBILITÀ CON VECCHI AMBIENTI ---
\usepackage{algorithm}
\usepackage{algpseudocode}
\usepackage{etoolbox}

% 1. Box per Algoritmi (Replica lo stile JSBox ma per pseudocodice)
\newtcolorbox{algobox}{
  colback=codeBg,
  colframe=mainOrange,
  title={\textbf{Algoritmo}},
  coltitle=white,
  fonttitle=\bfseries\ttfamily,
  enhanced,
  attach boxed title to top left={yshift=-3mm, xshift=3mm},
  boxed title style={colback=mainOrange},
  drop shadow
}
% --- Global Listings Style (Matches User Image) ---
\lstset{
  basicstyle=\ttfamily\small,
  keywordstyle=\color{green!60!black}\bfseries,
  commentstyle=\color{teal}\itshape,
  stringstyle=\color{red!60!black}\ttfamily,
  numberstyle=\tiny\color{gray},
  breaklines=true,
  tabsize=2,
  backgroundcolor=\color{codeBg}
}

% --- Customizing Algorithmic Colors ---
\algrenewcommand\algorithmicprocedure{\textbf{\textcolor{green!60!black}{Procedure}}}
\algrenewcommand\algorithmicfunction{\textbf{\textcolor{green!60!black}{Function}}}
\algrenewcommand\algorithmicwhile{\textbf{\textcolor{green!60!black}{While}}}
\algrenewcommand\algorithmicfor{\textbf{\textcolor{green!60!black}{For}}}
\algrenewcommand\algorithmicif{\textbf{\textcolor{green!60!black}{If}}}
\algrenewcommand\algorithmicelse{\textbf{\textcolor{green!60!black}{Else}}}
\algrenewcommand\algorithmicreturn{\textbf{\textcolor{green!60!black}{Return}}}
\algrenewcommand\algorithmicstate{\State} % State remains default (usually black)

\BeforeBeginEnvironment{algorithmic}{\begin{algobox}}
\AfterEndEnvironment{algorithmic}{\end{algobox}}

% --- Explanation Box Style ---
\newtcolorbox{explanation}[1]{
    colback=yellow!10,
    colframe=orange!80!black,
    coltitle=black,
    fonttitle=\bfseries\sffamily,
    title={#1},
    arc=2mm,
    left=4mm, right=4mm, top=4mm, bottom=4mm,
    sharp corners=south,
    boxrule=1pt,
    width=\linewidth
}

% 3. Box per Codice Generico (codebox)
\newtcolorbox{codebox}[1]{
    colback=codeBg,
    colframe=paletteIndigo,
    title={\textbf{#1}},
    coltitle=white,
    fonttitle=\bfseries\ttfamily,
    enhanced,
    attach boxed title to top left={yshift=-3mm, xshift=3mm},
    boxed title style={colback=paletteIndigo},
    drop shadow
}

% 2. Mappatura Teoremi/Definizioni al nuovo stile 'concept'
% Definiamo nuovi ambienti che wrappano 'concept'
% Se i file inclusi usano \begin{definition} ... \end{definition}, useranno questo stile.

\newenvironment{definition}[1][Definizione]
  {\begin{concept}{#1}}
  {\end{concept}}

\newenvironment{theorem}[1][Teorema]
  {\begin{concept}{#1}}
  {\end{concept}}

\newenvironment{example}[1][Esempio]
  {\begin{concept}{#1}}
  {\end{concept}}

\newenvironment{observation}[1][Osservazione]
  {\begin{concept}{#1}}
  {\end{concept}}
  
\newenvironment{property}[1][Proprietà]
  {\begin{concept}{#1}}
  {\end{concept}}

\newenvironment{intuition}[1][Intuizione]
  {\begin{concept}{#1}}
  {\end{concept}}

\newenvironment{analysis}[1][Analisi]
  {\begin{concept}{#1}}
  {\end{concept}}

\newenvironment{note}[1][Nota]
  {\begin{concept}{#1}}
  {\end{concept}}

% Mappatura Dimostrazione (usa standard proof di amsthm o custom box)
\newenvironment{dimostrazione}
  {\begin{proof}[Dimostrazione]}
  {\end{proof}}

% Comandi di formattazione extra
\newcommand{\R}{\mathbb{R}}
\newcommand{\N}{\mathbb{N}}
\newcommand{\Z}{\mathbb{Z}}


% ===================================================================
% INIZIO DOCUMENTO
% ===================================================================
\begin{document}

% Copertina Personalizzata
\customcover{Appunti Completi}{Informatica - Università di Pisa}

\tableofcontents
\newpage

% ---- Inclusione Capitoli ----
\newpage % ===================================================================
% FILE: Lezione1.tex
% ===================================================================

\part{Lezione 1(13/10/2025) }

\section{Definizione di Algoritmo}
Un algoritmo è una sequenza finita di operazioni elementari (passi) , univocamente determinata (non ambiguo) , che, se eseguita su un calcolatore, porta alla risoluzione di un problema . [cite: 160]
\subsection{Modello RAM (Random Access Machine) }
Nel modello RAM, si assume che le seguenti operazioni elementari abbiano costo "unitario" (costante) : [cite: 161]
\begin{itemize}
    \item \textbf{Operazioni aritmetiche}: +, -, *, /, \% 
    \item \textbf{Operazioni di confronto}: $<, >, ==, !=$ 
    \item \textbf{Operazioni logiche}: AND, OR, NOT 
    \item \textbf{Operazioni di trasferimento}: load/store/assegnamento 
    \item \textbf{Operazioni di controllo}: chiamata di funzione, RETURN 
\end{itemize}

\section{Analisi di Complessità}
Si analizza il costo computazionale (Tempo o Spazio ) in funzione della dimensione dell'input, $n$ . [cite: 162]
\begin{itemize}
    \item \textbf{Complessità in Tempo $T(n)$}: Numero di operazioni elementari eseguite . [cite: 163]
    \item \textbf{Complessità in Spazio $S(n)$}: Numero di celle di memoria utilizzate (oltre a quelle dell'input) . [cite: 164]
\end{itemize}

Ci si concentra sull' \textbf{ordine di grandezza} della funzione $T(n)$, ignorando costanti moltiplicative e termini di ordine inferiore . [cite: 165]
Ad esempio, $T(n) = 3n + 2$ e $T(n) = 5n + \log n + 4$ sono entrambe considerate di complessità \textbf{Lineare} . [cite: 166] $T(n) = 8n^2$ è \textbf{Quadratica} . [cite: 166]

\subsection{Caso Ottimo, Pessimo, Medio}
\begin{itemize}
    \item \textbf{Caso Ottimo} : L'istanza di input che richiede il minor tempo. [cite: 167]
    \item \textbf{Caso Pessimo} : L'istanza di input che richiede il maggior tempo. [cite: 168]
    \item \textbf{Caso Medio}: Complessità media su tutte le possibili istanze. [cite: 169]
\end{itemize}
Ci si concentra sul \textbf{caso pessimo} perché fornisce un limite superiore al costo: l'algoritmo non impiegherà mai più di $T(n)$ . [cite: 170]
\section{Esempio 1: Minimo in Vettore}
\begin{itemize}
    \item \textbf{Input}: Array $A[1..n]$ di interi . [cite: 171]
    \item \textbf{Output}: Il valore minimo contenuto in $A$ . [cite: 171]
\end{itemize}

% Questo ambiente ora funziona grazie ai pacchetti in main.tex
\begin{algorithmic}[1]
\Procedure{Minimo}{A, n} 
    \State $min = A[1]$ \Comment{Costo costante $c_1$ }
    \For{$i = 2 \to n$} \Comment{Eseguito $n-1$ volte }
        \If{$A[i] < min$} \Comment{Costo $c_2$ }
            \State $min = A[i]$ \Comment{Costo $c_3$ }
        \EndIf
    \EndFor
    \State \Return $min$ \Comment{Costo costante $c_4$ }
\EndProcedure
\end{algorithmic}

\textbf{Analisi:}
Il costo totale è $T(n) = c_1 + (n-1)(c_2 \text{ (confronto)} + c_3 \text{ (assegn. caso pessimo)}) + c_4$ . [cite: 172]
$T(n) = c'n + b$. [cite: 173]
La complessità è \textbf{Lineare} , $T(n) \in \Theta(n)$, sia nel caso ottimo che in quello pessimo . [cite: 174]
\section{Esempio 2: Cerca K }
\begin{itemize}
    \item \textbf{Input}: Array $A[1..n]$ di interi, $k$ intero . [cite: 175]
    \item \textbf{Output}: $i$ tale che $A[i]=k$ , o $-1$ se $k \notin A$ . [cite: 176]
\end{itemize}

\begin{algorithmic}[1]
\Procedure{Cerca-K}{A, n, k} 
    \State $i = 1$ 
    \State $trovato = \text{false}$ 
    \While{(\textbf{not} $trovato$) \textbf{and} ($i \le n$)} 
        \If{$A[i] == k$}
            \State $trovato = \text{true}$ 
        \Else
            \State $i = i + 1$ 
        \EndIf
    \EndWhile
    \If{$trovato$}
        \State \Return $i$ [cite: 177]
    \Else
        \State \Return $-1$ 
    \EndIf
\EndProcedure
\end{algorithmic}

\textbf{Analisi:}
\begin{itemize}
    \item \textbf{Caso Ottimo}: $k = A[1]$ . Il ciclo `while` esegue 1 iterazione. $T(n) \in \Theta(1)$ (Costante) . [cite: 178]
    \item \textbf{Caso Pessimo}: $k \notin A$ (o $k=A[n]$) . Il ciclo `while` esegue $n$ iterazioni. $T(n) \in \Theta(n)$ (Lineare) . [cite: 179, 180]
\end{itemize}

\section{Esempio 3: Minimo in Vettore Ordinato }
\begin{itemize}
    \item \textbf{Input}: Array $A[1..n]$ di interi, \textbf{ordinato} .
    \item \textbf{Output}: Il valore minimo contenuto in $A$ .
\end{itemize}

\begin{algorithmic}[1]
\Procedure{Minimo-Ordinato}{A, n} 
    \State \Return $A[1]$ 
\EndProcedure
\end{algorithmic}
\textbf{Analisi}: $T(n) \in \Theta(1)$ (Costante) . [cite: 182]
\section{Esempio 4: Cerca K in Vettore Ordinato (Ricerca Binaria) }
\begin{itemize}
    \item \textbf{Input}: Array $A[1..n]$ di interi \textbf{ordinato}, $k$ intero . [cite: 183]
    \item \textbf{Output}: $i$ tale che $A[i]=k$ , o $-1$ se $k \notin A$ . [cite: 184]
\end{itemize}
L'idea è di confrontare $k$ con l'elemento centrale $A[q]$ e dimezzare lo spazio di ricerca . [cite: 185]
\begin{algorithmic}[1]
\Procedure{BS-IT}{A, p, r, k} 
    \If{($k < A[p]$) \textbf{or} ($k > A[r]$)} \Comment{Controllo opzionale}
        \State \Return $-1$ 
    \EndIf
    \While{$p \le r$} 
        \State $q = \lfloor (p+r)/2 \rfloor$ 
        \If{$A[q] == k$}
            \State \Return $q$ 
        \ElsIf{$A[q] > k$}
            \State $r = q - 1$ [cite: 186]
        \Else
            \State $p = q + 1$ 
        \EndIf
    \EndWhile
    \State \Return $-1$ 
\EndProcedure
\end{algorithmic}

\textbf{Analisi:}
\begin{itemize}
    \item \textbf{Caso Ottimo}: $k = A[q]$ al primo ciclo. $T(n) \in \Theta(1)$ (Costante) . [cite: 187]
    \item \textbf{Caso Pessimo}: $k \notin A$. Il numero di iterazioni è $\log_2 n$ . $T(n) \in \Theta(\log n)$ (Logaritmica) . [cite: 188]
\end{itemize}

\newpage
\newpage \input{extra/precisazione_00}
\newpage % ===================================================================
% FILE: Lezione2.tex
% ===================================================================

\part{Lezione 2 (16/10/2025) }

\section{Selection Sort (Analisi) }
Pseudocodice (identico a Lez12).
\begin{algorithmic}[1]
    \Procedure{SelectionSort}{A}
        \For{$i = 1 \to n-1$}
            \State $min = i$
            \For{$j = i+1 \to n$} \Comment{Il loop interno fa $(n-i)$ iterazioni }
                \If{$A[j] < A[min]$}
                    \State $min = j$
                \EndIf
            \EndFor
            \State \Call{Swap}{A[i], A[min]}
        \EndFor
    \EndProcedure
\end{algorithmic}

\begin{explanation}{Selection Sort}
L'algoritmo seleziona iterativamente il minimo dalla parte non ordinata e lo sposta alla fine della parte ordinata.
\begin{itemize}
    \item \textbf{Ciclo Esterno}: Avanza il confine tra ordinato e non ordinato.
    \item \textbf{Ciclo Interno}: Cerca il minimo nel sottoarray destro.
    \item \textbf{Swap}: Scambia il minimo trovato con l'elemento corrente.
\end{itemize}
\end{explanation}

\subsection{Analisi Complessità (Numero Confronti) }
Il costo è dominato dal numero di confronti ($A[j] < A[min]$).
Il ciclo esterno \texttt{for i} esegue $n-1$ iterazioni.
Il ciclo interno \texttt{for j} esegue $n-i$ iterazioni per ogni $i$.
Il numero totale di confronti $C(n)$ è:
\[ C(n) = \sum_{i=1}^{n-1} (n-i) \]
\[ C(n) = (n-1) + (n-2) + \dots + 2 + 1 \]
Questa è la somma dei primi $n-1$ numeri naturali.
\[ C(n) = \frac{(n-1)n}{2} = \frac{n^2}{2} - \frac{n}{2} \]

La complessità è \textbf{Quadratica}, $T(n) \in \Theta(n^2)$.
\begin{observation}
    A differenza di Insertion Sort, la complessità di Selection Sort è $\Theta(n^2)$ \emph{sempre}, sia nel caso ottimo, medio e pessimo, perché i cicli \texttt{for} vengono eseguiti sempre lo stesso numero di volte.
\end{observation}

\subsection{Invariante di Ciclo }

\begin{definition}[Invariante: Selection Sort]
    \textbf{Invariante:} All'inizio dell'iterazione $i$-esima del ciclo FOR esterno (per $i=1..n-1$):
    \begin{enumerate}
        \item Il sottoarray $A[1..i-1]$ contiene gli $i-1$ elementi più piccoli di A.
        \item Il sottoarray $A[1..i-1]$ è ordinato.
    \end{enumerate}
    (Si dimostra per induzione ).
\end{definition}

\section{Esercizi}


\begin{example}[Esercizio 1: Cerca A[i] = i (Array non ordinato)]
    \begin{itemize}
        \item \textbf{Input}: Array $A[1..n]$ di interi.
        \item \textbf{Output}: TRUE se $\exists i$ t.c. $A[i] = i$, FALSE altrimenti.
    \end{itemize}

    \begin{algorithmic}[1]
        \Procedure{Cerca-Indice}{A, n}
            \State $i = 1$
            \State $trovato = \text{false}$
            \While{(\textbf{not} $trovato$) \textbf{and} ($i \le n$)}
                \If{$A[i] == i$}
                    \State $trovato = \text{true}$
                \Else
                    \State $i = i + 1$
                \EndIf
            \EndWhile
            \State \Return $trovato$
        \EndProcedure
    \end{algorithmic}

\begin{explanation}{Ricerca Lineare}
Scansiona l'array elemento per elemento.
Se trova $A[i] == i$, si ferma e ritorna TRUE.
Nel caso pessimo (non trovato), scorre tutto l'array ($n$ passi).
\end{explanation}
    \textbf{Analisi:}
    \begin{itemize}
        \item Caso Ottimo: $A[1]=1$. $T(n) \in \Theta(1)$.
        \item Caso Pessimo: Nessun $i$ t.c. $A[i]=i$. $T(n) \in \Theta(n)$ (Lineare).
    \end{itemize}
\end{example}


\begin{example}[Esercizio 2: Cerca A[i] = i (Array ordinato)]
    \begin{itemize}
        \item \textbf{Input}: Array $A[1..n]$ di interi, \textbf{ordinato}.
        \item \textbf{Output}: TRUE se $\exists i$ t.c. $A[i] = i$.
    \end{itemize}
    Si può usare una modifica della Ricerca Binaria.
    Si calcola $q = \lfloor (p+r)/2 \rfloor$.
    \begin{itemize}
        \item Se $A[q] == q$: Trovato.
        \item Se $A[q] > q$: L'elemento $i$ (se esiste) non può essere a destra di $q$. Si cerca a sinistra ($r=q-1$).
        \item Se $A[q] < q$: L'elemento $i$ (se esiste) non può essere a sinistra di $q$. Si cerca a destra ($p=q+1$).
    \end{itemize}
    \textbf{Analisi}: $T(n) \in O(\log n)$.
\end{example}


\begin{example}[Esercizio 3: Cerca A[i] = i (Ordinato, positivi, distinti)]
    \begin{itemize}
        \item \textbf{Input}: Array $A[1..n]$ ordinato, di interi \textbf{positivi} e \textbf{distinti}.
        \item \textbf{Output}: TRUE se $\exists i$ t.c. $A[i] = i$.
    \end{itemize}
    Se $A[1] = 1$: Ritorna TRUE.
    Se $A[1] > 1$: (cioè $A[1] \ge 2$). Allora $A[i] \ge A[1] + (i-1) \ge 2 + i - 1 = i+1$.
    Quindi $A[i] > i$ per ogni $i$. Ritorna FALSE.
    L'algoritmo corretto è:
    \begin{algorithmic}[1]
        \Procedure{Cerca-i-Positivi}{A}
            \State \Return $(A[1] == 1)$
        \EndProcedure
    \end{algorithmic}
    \textbf{Analisi}: $T(n) \in \Theta(1)$ (Costante).
\end{example}

\begin{example}[Esercizio 4: Somma K]
    \begin{itemize}
        \item \textbf{Input}: Array $A[1..n]$ di interi, $k$ intero.
        \item \textbf{Output}: TRUE se $\exists i, j$ t.c. $A[i] + A[j] = k$.
    \end{itemize}
    \textbf{Soluzione 1 (Brute force):}
    \begin{algorithmic}[1]
        \For{$i = 1 \to n-1$}
            \For{$j = i+1 \to n$}
                \If{$A[i] + A[j] == k$}
                    \State \Return $\text{true}$
                \EndIf
            \EndFor
        \EndFor
        \State \Return $\text{false}$
    \end{algorithmic}
    \textbf{Analisi 1}: Caso pessimo $\Theta(n^2)$ (Quadratico).
    \textbf{Soluzione 2 (se $A$ è ordinato ):}
    Si usano due indici, $L=1$ e $R=n$.
    \begin{algorithmic}[1]
        \State $L=1$, $R=n$
        \While{$L < R$}
            \State $somma = A[L] + A[R]$
            \If{$somma == k$}
                \State \Return $\text{true}$
            \ElsIf{$somma < k$}
                \State $L = L + 1$ \Comment{Serve una somma più grande }
            \Else
                \State $R = R - 1$ \Comment{Serve una somma più piccola}
            \EndIf
        \EndWhile
        \State \Return $\text{false}$
    \end{algorithmic}

\begin{explanation}{Tecnica dei Due Indici}
Poiché l'array è ordinato, possiamo restringere la ricerca:
\begin{itemize}
    \item $Somma < K \to$ Incremento $L$ (serve valore più grande).
    \item $Somma > K \to$ Decremento $R$ (serve valore più piccolo).
\end{itemize}
Questo riduce la complessità da quadratica a lineare.
\end{explanation}
    \textbf{Analisi 2}: $T(n) \in \Theta(n)$ (Lineare).
\end{example}


\begin{example}[Esercizio 5: Array Palindromo]
    \begin{itemize}
        \item \textbf{Input}: Array $A[1..n]$.
        \item \textbf{Output}: TRUE se $A$ è palindromo, FALSE altrimenti. (E.g., \texttt{[3, 7, 21, 40, 21, 7, 3]} ).
    \end{itemize}
    \textbf{Soluzione (con due indici):}
    \begin{algorithmic}[1]
        \State $i=1$, $j=n$
        \While{$i < j$}
            \If{$A[i] \neq A[j]$}
                \State \Return $\text{false}$
            \EndIf
            \State $i = i + 1$
            \State $j = j - 1$
        \EndWhile
        \State \Return $\text{true}$
    \end{algorithmic}

\begin{explanation}{Verifica Palindromo}
Confronta gli estremi convergendo verso il centro.
Se una coppia non corrisponde, non è palindromo.
\end{explanation}
    \textbf{Analisi}: $T(n) \in \Theta(n)$.
\end{example}

\newpage

% \input{lezioni/Lezione3} (Removed duplicate)
\newpage % ===================================================================
% FILE: Lezione4.tex
% ===================================================================

\part{Lezione 14 (20/10/2025) }

\section{Notazione Asintotica }
La complessità $T(n)$ si esprime in \textbf{ordine di grandezza} , ignorando costanti moltiplicative e termini di ordine inferiore . [cite: 41]
\begin{itemize}
    \item $T(n) = 3n^2 + 2n + 5 \to$ Quadratica ($\Theta(n^2)$) 
    \item $T(n) = 7n + 24 \to$ Lineare ($\Theta(n)$) 
    \item $T(n) = 5 \to$ Costante ($\Theta(1)$) 
    \item $T(n) = \log_3 n + 2 \to$ Logaritmica ($\Theta(\log n)$) 
\end{itemize}
Si usano funzioni di riferimento semplici $g(n)$ (es. $n^2$, $n$, $\log n$) per classificare $f(n) = T(n)$ . [cite: 42]
\section{Notazione $\Theta$ (Theta) - Limite Stretto }
% --- APPLICAZIONE STILE ---
\begin{definition}[Notazione $\Theta$ (Theta)]
$$ \Theta(g(n)) = \{ f(n) \mid \exists c_1, c_2, n_0 > 0 : \forall n \ge n_0, 0 \le c_1 g(n) \le f(n) \le c_2 g(n) \} $$
Si dice "$f(n)$ è in Theta di $g(n)$" .
$g(n)$ è un \textbf{limite asintotico stretto} per $f(n)$ . [cite: 44]
Graficamente, da $n_0$ in poi, $f(n)$ è "intrappolata" tra $c_1 g(n)$ e $c_2 g(n)$ . [cite: 45]
\begin{itemize}
    \item Esempio: Selection Sort è $\Theta(n^2)$ . [cite: 46]
    \item Esempio: $\frac{1}{2}n^2 - 2n \in \Theta(n^2)$ . [cite: 46]
\end{itemize}
\end{definition}

\section{Notazione $O$ (O-grande) - Limite Superiore }
% --- APPLICAZIONE STILE ---
\begin{definition}[Notazione $O$ (O-grande)]
$$ O(g(n)) = \{ f(n) \mid \exists c, n_0 > 0 : \forall n \ge n_0, 0 \le f(n) \le c g(n) \} $$
Si dice "$f(n)$ è in O-grande di $g(n)$" .
$g(n)$ è un \textbf{limite asintotico superiore} per $f(n)$ . [cite: 47]
Graficamente, da $n_0$ in poi, $f(n)$ non cresce più velocemente di $c g(n)$ . [cite: 48]
\begin{itemize}
    \item Esempio: $f(n) = an^2 + bn + c \in O(n^2)$ . [cite: 49]
    \item Esempio: $f(n) = an^2 + bn + c \in O(n^3)$ . [cite: 50]
    \item Esempio: $f(n) = an^2 + bn + c \notin O(n)$ . [cite: 51]
\end{itemize}
Proprietà: $f(n) \in \Theta(g(n)) \implies f(n) \in O(g(n))$ .
\end{definition}

\section{Notazione $\Omega$ (Omega) - Limite Inferiore }
% --- APPLICAZIONE STILE ---
\begin{definition}[Notazione $\Omega$ (Omega)]
$$ \Omega(g(n)) = \{ f(n) \mid \exists c, n_0 > 0 : \forall n \ge n_0, 0 \le c g(n) \le f(n) \} $$
Si dice "$f(n)$ è in Omega di $g(n)$" .
$g(n)$ è un \textbf{limite asintotico inferiore} per $f(n)$ . [cite: 52]
Graficamente, da $n_0$ in poi, $f(n)$ non cresce più lentamente di $c g(n)$ . [cite: 53]
\begin{itemize}
    \item Esempio: $an^2 + bn + c \in \Omega(n^2)$ . [cite: 54]
    \item Esempio: $an^2 + bn + c \in \Omega(n)$ . [cite: 54]
    \item Esempio: $an^2 + bn + c \notin \Omega(n^3)$ . [cite: 55]
\end{itemize}
Proprietà: $f(n) \in \Theta(g(n)) \implies f(n) \in \Omega(g(n))$ .
\end{definition}

\subsection{Teorema}
% --- APPLICAZIONE STILE ---
\begin{theorem}
$$ f(n) \in \Theta(g(n)) \iff f(n) \in O(g(n)) \text{ e } f(n) \in \Omega(g(n)) $$
\end{theorem}

\section{Proprietà e Gerarchia}
% --- APPLICAZIONE STILE ---
\begin{observation}[Proprietà della Notazione Asintotica]
\begin{itemize}
    \item \textbf{Riflessività}: $f(n) \in \Theta(f(n))$, $f(n) \in O(f(n))$, $f(n) \in \Omega(f(n))$ . [cite: 56]
    \item \textbf{Simmetria (Theta)}: $f(n) \in \Theta(g(n)) \iff g(n) \in \Theta(f(n))$ . [cite: 57]
    \item \textbf{Trasposta (O/Omega)}: $f(n) \in O(g(n)) \iff g(n) \in \Omega(f(n))$ . [cite: 57]
    \item \textbf{Transitività}: Vale per $O, \Omega, \Theta$. [cite: 58]
    Es: $f_1 \in O(f_2)$ e $f_2 \in O(f_3) \implies f_1 \in O(f_3)$ . [cite: 59]
    \item \textbf{Somma}: $f_1 \in O(g_1)$ e $f_2 \in O(g_2) \implies f_1+f_2 \in O(\max(g_1, g_2))$ . [cite: 60]
    \item \textbf{Prodotto}: $f_1 \in O(g_1)$ e $f_2 \in O(g_2) \implies f_1 \cdot f_2 \in O(g_1 \cdot g_2)$ . [cite: 61]
\end{itemize}
\end{observation}


\begin{observation}[Equivalenza dei Logaritmi]
Tutte le basi dei logaritmi sono asintoticamente equivalenti . [cite: 62]
Dalla formula del cambio di base: $\log_a n = \frac{\log_b n}{\log_b a}$ . [cite: 63]
Poiché $\frac{1}{\log_b a}$ è una costante , si ha $\Theta(\log_a n) = \Theta(\log_b n)$ . [cite: 64]
Per questo motivo, si scrive genericamente $O(\log n)$ . [cite: 64]
\end{observation}

\subsection{Gerarchia degli ordini di grandezza }

\begin{example}[Gerarchia di Crescita]
Per $0 < h \le k$ e $1 < a < b$ :
$$ \Theta(1) \subset \dots \subset \Theta(\log n) \subset \dots \subset \Theta(n^h) \subset \Theta(n^k) \subset \Theta(n^k \log n) \subset \dots \subset \Theta(a^n) \subset \Theta(b^n) \subset \dots $$
Ordinando le funzioni in per ordine crescente:
$1$ (costante), $4^5$ (costante) , $\log n$, $\log^2 n$, $n^{1/2}$ (o $\sqrt{n}$), $n$, $n \log n$, $n^4 - 7n^3 (\sim n^4)$, $n^5 - 5n^2 (\sim n^5)$, $2^n$, $3^n$. [cite: 65]
\end{example}
\newpage
\newpage % ===================================================================
% FILE: Lezione5.tex
% ===================================================================

\part{Lezione 21 (06/11/2025) }

\section{Paradigma Divide et Impera }
Il paradigma "Divide et Impera" (Dividi e Conquista) è una tecnica per progettare algoritmi, tipicamente ricorsivi, che si articola in tre fasi:

\begin{definition}[Paradigma Divide et Impera]
    \begin{enumerate}
        \item \textbf{DIVIDE}: Il problema di dimensione $n$ viene suddiviso in $a$ sottoproblemi dello stesso tipo, ma di dimensione minore ($n/b$).
        \item \textbf{IMPERA}: I sottoproblemi vengono risolti. Se sono abbastanza piccoli (casi base), vengono risolti direttamente. Altrimenti, vengono risolti ricorsivamente con la stessa tecnica.
        \item \textbf{COMBINE}: Le soluzioni degli $a$ sottoproblemi vengono combinate per ottenere la soluzione del problema originale.
    \end{enumerate}
\end{definition}

\subsection{Diagramma Concettuale }
\begin{center}
    \begin{tikzpicture}[
        node distance=2cm and 1.5cm,
        block/.style={draw, rectangle, minimum height=1cm, minimum width=1.5cm, text centered, font=\small},
        dot/.style={node distance=1.5cm and 1cm}
        ]

        \node[block] (P) {$P, n$ (Problema) };
        \node[block] (P1) [below left of=P, yshift=-1cm] {$P_1, n_1$ };
        \node (P_dots) [right=of P1, style=dot] {$\dots$};
        \node[block] (Pa) [right=of P_dots, style=dot] {$P_a, n_a$ };

        \node[block] (S1) [below=of P1] {$S_1, n_1$ };
        \node (S_dots) [right=of S1, style=dot] {$\dots$};
        \node[block] (Sa) [right=of S_dots, style=dot] {$S_a, n_a$ };
        \node[block] (S) [below right of=S_dots, yshift=-1cm] {$S, n$ (Soluzione) };

        \draw[->, thick] (P.south) -- (P1.north) node [midway, left, xshift=-5mm] {\textbf{DIVIDE} };
        \draw[->, thick] (P.south) -- (P_dots.north);
        \draw[->, thick] (P.south) -- (Pa.north);

        \draw[->, thick, decorate, decoration={snake, segment length=8mm, amplitude=1mm}] (P1.south) -- (S1.north) node [midway, left, xshift=-5mm] {\textbf{"IMPERA"} };
        \draw[->, thick, decorate, decoration={snake, segment length=8mm, amplitude=1mm}] (P_dots.south) -- (S_dots.north);
        \draw[->, thick, decorate, decoration={snake, segment length=8mm, amplitude=1mm}] (Pa.south) -- (Sa.north);

        \draw[->, thick] (S1.south) -- (S.west);
        \draw[->, thick] (S_dots.south) -- (S.north);
        \draw[->, thick] (Sa.south) -- (S.east) node [midway, below, xshift=5mm, yshift=-5mm] {\textbf{COMBINE} };
    \end{tikzpicture}
\end{center}

\section{Analisi Complessità D\&I}

\begin{definition}[Analisi Complessità D\&I]
    Sia $T(n)$ il costo per risolvere un problema di dimensione $n$.
    Sia $D(n)$ il costo della fase DIVIDE.
    Sia $C(n)$ il costo della fase COMBINE.
    L'equazione di ricorrenza generale è:
    $$ T(n) = \sum_{i=1}^{a} T(n_i) + D(n) + C(n) $$

    Caso particolare: \textbf{Divisione Bilanciata}.
    Il problema è diviso in $a$ sottoproblemi, ognuno di dimensione $n/b$.
    Sia $f(n) = D(n) + C(n)$ il costo di divide e combine.
    $$ T(n) = aT(n/b) + f(n) $$
\end{definition}


\section{Esempio: Ricerca Binaria (D\&I) }

\begin{example}[Ricerca Binaria: Setup]
    \begin{itemize}
        \item \textbf{Input}: $A[p..r]$ ordinato, chiave $k$.
        \item \textbf{Output}: Indice $i$ t.c. $A[i]=k$, o $-1$.
    \end{itemize}
\end{example}

\begin{algorithmic}[1]
    \Procedure{BinarySearch}{A, p, r, k}
        \If{$p > r$} \Comment{Caso Base 1: array vuoto }
            \State \Return $-1$
        \EndIf

        \State $q = \lfloor (p+r)/2 \rfloor$ \Comment{DIVIDE }

        \If{$A[q] == k$} \Comment{IMPERA (Caso Base 2) }
            \State \Return $q$
        \ElsIf{$A[q] > k$} \Comment{IMPERA (Ricorsione) }
            \State \Return \Call{BinarySearch}{A, p, q-1, k}
        \Else
            \State \Return \Call{BinarySearch}{A, q+1, r, k}
        \EndIf

        \Comment{COMBINE: non necessario, costo $\Theta(1)$ }
    \EndProcedure
\end{algorithmic}

\begin{explanation}{Ricerca Binaria}
Ad ogni passo dimezza lo spazio di ricerca ($n \to n/2 \to n/4 \dots$).
\begin{itemize}
    \item Se $A[q] > k$, cerco a sinistra.
    \item Se $A[q] < k$, cerco a destra.
    \item Costo logaritmico $\Theta(\log n)$.
\end{itemize}
\end{explanation}


\begin{observation}[Analisi: Ricerca Binaria]
    \textbf{Analisi Ricorrenza BS: }
    C'è $a=1$ sottoproblema di dimensione $n/b = n/2$.
    $f(n) = D(n) + C(n) = \Theta(1) + \Theta(1) = \Theta(1)$.
    $$ T(n) = \begin{cases} \Theta(1) & \text{se } n \le 1 \\ T(n/2) + \Theta(1) & \text{se } n > 1 \end{cases} $$

    \textbf{Soluzione (Metodo Iterativo)}:
    $T(n) = T(n/2) + c$
    $T(n) = (T(n/4) + c) + c = T(n/4) + 2c$
    $T(n) = (T(n/8) + c) + 2c = T(n/8) + 3c$
    ... dopo $i$ passi...
    $T(n) = T(n/2^i) + i \cdot c$
    Ci si ferma al caso base quando $n/2^i = 1 \implies i = \log_2 n$.
    $T(n) = T(1) + c \cdot \log_2 n = \Theta(1) + \Theta(\log n) = \Theta(\log n)$.
\end{observation}

\section{Esempio: Minimo/Massimo (D\&I) }

\begin{example}[Minimo/Massimo: Setup]
    \begin{itemize}
        \item \textbf{Input}: $A[1..n]$.
        \item \textbf{Output}: Coppia $\langle min, max \rangle$ di A.
    \end{itemize}
\end{example}

\begin{algorithmic}[1]
    \Procedure{MinMax}{A, p, r}
        \If{$r - p \le 1$} \Comment{Caso Base: 1 o 2 elementi }
            \If{$A[p] \le A[r]$}
                \State \Return $\langle A[p], A[r] \rangle$
            \Else
                \State \Return $\langle A[r], A[p] \rangle$
            \EndIf
        \Else
            \State $q = \lfloor (p+r)/2 \rfloor$ \Comment{DIVIDE}
            \State $\langle min_1, max_1 \rangle = \Call{MinMax}{A, p, q}$ \Comment{IMPERA }
            \State $\langle min_2, max_2 \rangle = \Call{MinMax}{A, q+1, r}$ \Comment{IMPERA }
            \State $min = \min(min_1, min_2)$ \Comment{COMBINE }
            \State $max = \max(max_1, max_2)$
            \State \Return $\langle min, max \rangle$
        \EndIf
    \EndProcedure
\end{algorithmic}

\begin{explanation}{Minimo e Massimo Simultanei}
Per trovare min e max con meno confronti ($3 \lfloor n/2 \rfloor$ invece di $2n$):
\begin{itemize}
    \item Divide l'array in due metà.
    \item Risolve ricorsivamente.
    \item Combina confrontando i minimi tra loro e i massimi tra loro.
\end{itemize}
\end{explanation}


\begin{observation}[Analisi: Minimo/Massimo]
    \textbf{Analisi Ricorrenza MinMax:}
    $a=2$ sottoproblemi, $n/b = n/2$.
    $f(n) = D(n) \text{ (cost.)} + C(n) \text{ (2 confr.)} = \Theta(1)$.
    $$ T(n) = \begin{cases} \Theta(1) & \text{se } n \le 2 \\ 2T(n/2) + \Theta(1) & \text{se } n \ge 3 \end{cases} $$
    (Questa ricorrenza si risolve in $T(n) = \Theta(n)$).
\end{observation}

\newpage

\newpage % ===================================================================
% FILE: Lezione6.tex
% ===================================================================

\part{Lezione 22 (10/11/2025) }

\section{Mergesort }
Mergesort è un algoritmo di ordinamento basato su Divide et Impera .
\begin{definition}[Mergesort: Paradigma D\&I]
\begin{itemize}
    \item \textbf{DIVIDE} : Divide l'array $A[p..r]$ in due metà, $A[p..q]$ e $A[q+1..r]$, dove $q = \lfloor (p+r)/2 \rfloor$ .
    \item \textbf{IMPERA} : Ordina ricorsivamente le due metà chiamando `Mergesort(A, p, q)` e `Mergesort(A, q+1, r)` .
    \item \textbf{COMBINE} : Combina (fonde) i due sottoarray ordinati $A[p..q]$ e $A[q+1..r]$ in un unico array ordinato $A[p..r]$ tramite la procedura `Merge(A, p, q, r)` .
\end{itemize}
\end{definition}

\subsection{Pseudocodice Mergesort}
\begin{algorithmic}[1]
\Procedure{Mergesort}{A, p, r} 
    \If{$p < r$} 
        \State $q = \lfloor (p+r)/2 \rfloor$ \Comment{DIVIDE }
        \State \Call{Mergesort}{A, p, q} \Comment{IMPERA }
        \State \Call{Mergesort}{A, q+1, r} \Comment{IMPERA }
        \State \Call{Merge}{A, p, q, r} \Comment{COMBINE }
    \EndIf
\EndProcedure
\end{algorithmic}

\subsection{Procedura Merge }

\begin{observation}[Procedura Merge]
La procedura `Merge` fonde due sottoarray contigui $A[p..q]$ e $A[q+1..r]$, che si assumono \textbf{già ordinati}.
Ha complessità \textbf{Lineare} $T(n) = \Theta(n)$ , dove $n = r-p+1$ .
Utilizza due array di appoggio, $L$ e $R$ , e due "sentinelle" ($\infty$) per evitare controlli sull'indice .
\end{observation}

\begin{algorithmic}[1]
\Procedure{Merge}{A, p, q, r} 
    \State $n_1 = q - p + 1$ \Comment{Dim. primo sottoarray }
    \State $n_2 = r - q$ \Comment{Dim. secondo sottoarray }
    \State Crea array $L[1..n_1+1]$ e $R[1..n_2+1]$ 
    
    \Comment{Copia i dati negli array di appoggio}
    \For{$i = 1 \to n_1$}
        \State $L[i] = A[p + i - 1]$ 
    \EndFor
    \For{$j = 1 \to n_2$}
        \State $R[j] = A[q + j]$ 
    \EndFor
    
    \State $L[n_1 + 1] = +\infty$ \Comment{Sentinella }
    \State $R[n_2 + 1] = +\infty$ \Comment{Sentinella }
    
    \State $i = 1$ \Comment{Indice per $L$ }
    \State $j = 1$ \Comment{Indice per $R$ }
    
    \Comment{Fondi L e R nell'array A}
    \For{$k = p \to r$} 
        \If{$L[i] \le R[j]$} 
            \State $A[k] = L[i]$ 
            \State $i = i + 1$ 
        \Else
            \State $A[k] = R[j]$
            \State $j = j + 1$
        \EndIf
    \EndFor
\EndProcedure
\end{algorithmic}

\subsection{Analisi Complessità Mergesort }

\begin{definition}[Analisi Mergesort: Ricorrenza]
\textbf{Equazione di Ricorrenza:}
$a=2$ sottoproblemi, $n/b = n/2$.
$f(n) = D(n) \text{ (cost.)} + C(n) \text{ (Merge)} = \Theta(1) + \Theta(n) = \Theta(n)$.
$$ T(n) = \begin{cases} \Theta(1) & \text{se } n = 1 \text{ } \\ 2T(n/2) + \Theta(n) & \text{se } n > 1 \text{ } \end{cases} $$
\end{definition}

\textbf{Soluzione 1: Albero di Ricorsione} 
L'albero di ricorsione mostra il costo $f(n_i)$ ad ogni livello.
\begin{center}
\begin{tikzpicture}[
    level distance=1.5cm, 
    level 1/.style={sibling distance=5cm},
    level 2/.style={sibling distance=2.5cm},
    level 3/.style={sibling distance=1.2cm},
    level 4/.style={sibling distance=0.8cm},
    every node/.style={draw, circle, inner sep=1pt, minimum size=6mm, font=\small},
    ]
  % Albero
  \node (z){$n$}
    child {node {$n/2$} 
      child {node {$n/4$} 
        child {node[draw=none] {$\vdots$}
            child{node {$1$} }
        }
      } 
      child {node {$n/4$}
        child {node[draw=none] {$\vdots$}
            child{node {$1$}}
        }
      }
    }
    child {node {$n/2$} 
      child {node {$n/4$} 
         child {node[draw=none] {$\vdots$}
            child{node {$1$}}
         }
      } 
      child {node {$n/4$}
         child {node[draw=none] {$\vdots$}
            child{node {$1$}}
         }
      }
    };
  % Costi
    \node[right=of z, xshift=5cm, draw=none, font=\small] (l0) {Costo: $cn$ };
    \node[right=of z, xshift=5cm, yshift=-1.5cm, draw=none, font=\small] (l1) {Costo: $2 \cdot c(n/2) = cn$ };
    \node[right=of z, xshift=5cm, yshift=-3cm, draw=none, font=\small] (l2) {Costo: $4 \cdot c(n/4) = cn$ };
    \node[right=of z, xshift=5cm, yshift=-4.5cm, draw=none, font=\small] (l3) {$\vdots$};
    \node[right=of z, xshift=5cm, yshift=-6cm, draw=none, font=\small] (l4) {Costo (foglie): $n \cdot c(1) = \Theta(n)$ };
  % Altezza
    \draw[decorate, decoration={brace, amplitude=10pt}] (l0.north west) ++ (-12, 0.5) -- (l4.south west) ++ (-12, -0.5) node [midway, left, xshift=-10pt, font=\small] {Altezza $\log_2 n$};
    \node[below=of l4, yshift=-1cm, draw=none, font=\small] {Totale: $\sum_{i=0}^{\log_2 n - 1} cn + \Theta(n) = cn \log_2 n + \Theta(n) = \Theta(n \log n)$};
\end{tikzpicture}
\end{center}
Il costo per ogni livello è $cn$ . L'albero ha $\log_2 n$ livelli.
Il costo totale è $cn \cdot \log_2 n = \Theta(n \log n)$ .
\begin{observation}[Analisi Mergesort: Metodo Iterativo]
\textbf{Soluzione 2: Metodo Iterativo (Sostituzione)} 
$T(n) = 2T(n/2) + cn$ 
$T(n) = 2(2T(n/4) + c(n/2)) + cn = 4T(n/4) + cn + cn = 4T(n/4) + 2cn$ 
$T(n) = 4(2T(n/8) + c(n/4)) + 2cn = 8T(n/8) + cn + 2cn = 8T(n/8) + 3cn$ 
... dopo $i$ passi ...
$T(n) = 2^i T(n/2^i) + i \cdot cn$ 
Ci si ferma al caso base $n/2^i = 1 \implies i = \log_2 n$.
$T(n) = 2^{\log_2 n} T(1) + (\log_2 n) \cdot cn$ 
$T(n) = n \cdot \Theta(1) + cn \log_2 n = \Theta(n \log n)$ .
\end{observation}


\begin{observation}[Complessità in Spazio: Mergesort]
Mergesort \textbf{non} ordina "in loco" , poiché richiede $\Theta(n)$ spazio ausiliario per gli array $L$ e $R$ ad ogni chiamata di `Merge` .
\end{observation}

\subsection{Esempio: Albero delle Chiamate }
Per $A[1..7]$ , l'ordine delle chiamate ricorsive è :
\begin{center}
\begin{tikzpicture}[
    level distance=1.3cm,
    level 1/.style={sibling distance=4cm},
    level 2/.style={sibling distance=2cm},
    level 3/.style={sibling distance=1.5cm},
    every node/.style={align=center, font=\small}
    ]
  \node {$MS(1,7)$ \\ $n=7$ }
    child {node {$MS(1,4)$ \\ $n=4$ }
      child {node {$MS(1,2)$ \\ $n=2$ }
        child {node {$MS(1,1)$ \\ $n=1$ }}
        child {node {$MS(2,2)$ \\ $n=1$ }}
      } 
      child {node {$MS(3,4)$ \\ $n=2$ }
        child {node {$MS(3,3)$ \\ $n=1$ }}
        child {node {$MS(4,4)$ \\ $n=1$ }}
      }
    }
    child {node {$MS(5,7)$ \\ $n=3$ }
      child {node {$MS(5,6)$ \\ $n=2$ }
        child {node {$MS(5,5)$ \\ $n=1$ }}
        child {node {$MS(6,6)$ \\ $n=1$ }}
      } 
      child {node {$MS(7,7)$ \\ $n=1$ }}
    };
\end{tikzpicture}
\end{center}

\newpage
\newpage % ===================================================================
% FILE: Lezione7.tex
% ===================================================================

\part*{Spiegazione della Lezione 23 (12/10/2025)}
\section*{Introduzione: Relazioni di Ricorrenza}

Questi appunti della Lezione 23 affrontano un argomento cruciale nell'analisi degli algoritmi: le \textbf{Relazioni di Ricorrenza}.
In breve, queste sono equazioni matematiche usate per descrivere il tempo di esecuzione, $T(n)$, di un algoritmo che chiama sé stesso (cioè un algoritmo ricorsivo).

\subsection*{Tipi di Relazioni di Ricorrenza}
Gli appunti ne identificano tre tipi principali.


\begin{definition}[Relazioni Bilanciate (Divide et Impera)]
    Sono le più comuni negli algoritmi "Divide et Impera" (come Mergesort).
    Hanno una forma specifica:
    \[
    T(n) = aT\left(\frac{n}{b}\right) + f(n)
    \]
    \begin{itemize}
        \item \textbf{$a$} è il numero di sotto-problemi in cui dividiamo il problema principale.
        \item \textbf{$n/b$} è la dimensione di ciascun sotto-problema.
        \item \textbf{$f(n)$} è chiamata la "forzante" e rappresenta il lavoro "extra" fatto per dividere e ricombinare i risultati.
    \end{itemize}
\end{definition}

\begin{enumerate}
    \setcounter{enumi}{1}
    \item \textbf{Relazioni di Ordine K:} Queste dipendono dai valori immediatamente precedenti, come $T(n-1)$, $T(n-2)$, ecc. (es. Fibonacci).
    \item \textbf{Caso Generale:} Una forma più complessa dove i sotto-problemi potrebbero non avere dimensioni uguali.
\end{enumerate}

\subsection*{Esempi Concreti di Relazioni Bilanciate}

\begin{example}[Mergesort]
    Per ordinare un array, lo divide in 2 metà ($a=2$), le ordina ricorsivamente (ciascuna di dimensione $n/2$, quindi $b=2$) e poi le fonde (un'operazione che costa $\Theta(n)$).
    La sua relazione è: $T(n) = 2T\left(\frac{n}{2}\right) + \Theta(n)$.
\end{example}

\begin{example}[Ricerca Binaria]
    Per cercare in un array ordinato, fa un confronto, poi chiama ricorsivamente sé stessa su \emph{una} sola metà ($a=1$) di dimensione $n/2$ ($b=2$).
    Il costo del confronto è costante, $\Theta(1)$. La sua relazione è: $T(n) \le T\left(\frac{n}{2}\right) + \Theta(1)$.
\end{example}

\subsection*{Come Risolvere Queste Relazioni?}
Una volta che abbiamo l'equazione, come troviamo la complessità finale?
Gli appunti elencano quattro metodi:

\begin{enumerate}
    \item \textbf{Metodo Iterativo}
    \item \textbf{Metodo di Sostituzione} (Induzione)
    \item \textbf{Albero di Ricorsione} (Metodo grafico)
    \item \textbf{Teorema Principale (Master Theorem)}
\end{enumerate}


\section*{Il Cuore della Lezione: Il Master Theorem}
Il Teorema Principale (Master Theorem) è una "ricetta" che funziona solo per le relazioni bilanciate $T(n) = aT\left(\frac{n}{b}\right) + f(n)$.
L'idea centrale è \textbf{confrontare due "forze"}:
\begin{enumerate}
    \item Il costo della \textbf{ricorsione} (quanti sotto-problemi si creano).
    \item Il costo del \textbf{lavoro extra} $f(n)$ (la "forzante").
\end{enumerate}

\begin{definition}[Il Master Theorem]
    Data $T(n) = aT\left(\frac{n}{b}\right) + f(n)$, si calcola la \textbf{"Funzione Spartiacque"}: \textbf{$n^{\log_b a}$}.
    Confrontando $f(n)$ con $n^{\log_b a}$ si ricade in uno dei tre casi:

    \begin{itemize}
        \item \textbf{Caso 1: $f(n)$ polinomialmente minore} ($f(n) = O(n^{\log_b a - \epsilon})$)
        \begin{itemize}
            \item \textbf{Logica:} Il costo è dominato dalla ricorsione (dalle foglie).
            \item \textbf{Soluzione: $T(n) = \Theta(n^{\log_b a})$}.
        \end{itemize}

        \item \textbf{Caso 2: $f(n)$ circa uguale} ($f(n) = \Theta(n^{\log_b a} \cdot \log^k n)$)
        \begin{itemize}
            \item \textbf{Logica:} Le forze sono bilanciate; il costo è lo stesso ad ogni livello.
            \item \textbf{Soluzione: $T(n) = \Theta(n^{\log_b a} \cdot \log^{k+1} n)$}. (Se $k=0$, la soluzione è $\Theta(n^{\log_b a} \cdot \log n)$).
        \end{itemize}

        \item \textbf{Caso 3: $f(n)$ polinomialmente maggiore} ($f(n) = \Omega(n^{\log_b a + \epsilon})$)
        \begin{itemize}
            \item \textbf{Logica:} Il costo è dominato dal lavoro extra $f(n)$ (il collo di bottiglia).
            \item \textbf{Controllo:} Richiede la "Condizione di Regolarità" ($a f(n/b) \le c f(n)$).
            \item \textbf{Soluzione: $T(n) = \Theta(f(n))$}.
        \end{itemize}
    \end{itemize}
\end{definition}


\subsection*{Applicazioni del Master Theorem}

\begin{example}[Mergesort]
    $T(n) = 2T\left(\frac{n}{2}\right) + \Theta(n)$
    \begin{itemize}
        \item $a=2, b=2$. Spartiacque: $n^{\log_2 2} = n$.
        \item Confronto: $f(n) = \Theta(n)$ è \emph{uguale} allo spartiacque (Caso 2 con $k=0$).
        \item \textbf{Soluzione: $\Theta(n \log n)$}.
    \end{itemize}
\end{example}

\begin{example}[Ricerca Binaria]
    $T(n) = T\left(\frac{n}{2}\right) + \Theta(1)$
    \begin{itemize}
        \item $a=1, b=2$. Spartiacque: $n^{\log_2 1} = n^0 = 1$.
        \item Confronto: $f(n) = \Theta(1)$ è \emph{uguale} allo spartiacque (Caso 2 con $k=0$).
        \item \textbf{Soluzione: $\Theta(1 \cdot \log n) = \Theta(\log n)$}.
    \end{itemize}
\end{example}

\begin{example}[Esempio (Min/Max)]
    $T(n) = 2T\left(\frac{n}{2}\right) + \Theta(1)$
    \begin{itemize}
        \item $a=2, b=2$. Spartiacque: $n^{\log_2 2} = n$.
        \item Confronto: $f(n) = \Theta(1)$ è \emph{polinomialmente minore} di $n$ (Caso 1).
        \item \textbf{Soluzione: $\Theta(n)$}.
    \end{itemize}
\end{example}

\begin{example}[Esempio 1 (dal testo)]
    $T(n) = 9T\left(\frac{n}{3}\right) + n$
    \begin{itemize}
        \item $a=9, b=3$. Spartiacque: $n^{\log_3 9} = n^2$.
        \item Confronto: $f(n) = n$ è \emph{polinomialmente minore} di $n^2$ (Caso 1).
        \item \textbf{Soluzione: $\Theta(n^2)$}.
    \end{itemize}
\end{example}

\begin{example}[Esempio 2 (dal testo)]
    $T(n) \le T\left(\frac{2n}{3}\right) + 1$
    \begin{itemize}
        \item $a=1, b=3/2$. Spartiacque: $n^{\log_{3/2} 1} = n^0 = 1$.
        \item Confronto: $f(n) = 1$ è \emph{uguale} allo spartiacque (Caso 2 con $k=0$).
        \item \textbf{Soluzione: $\Theta(\log n)$}.
    \end{itemize}
\end{example}

\begin{example}[Esempio 3 (dal testo)]
    $T(n) = 3T\left(\frac{n}{4}\right) + n \log n$
    \begin{itemize}
        \item $a=3, b=4$. Spartiacque: $n^{\log_4 3} \approx n^{0.792}$.
        \item Confronto: $f(n) = n \log n$ è \emph{polinomialmente maggiore} (Caso 3).
        \item (Si verifica la condizione di regolarità).
        \item \textbf{Soluzione: $\Theta(f(n)) = \Theta(n \log n)$}.
    \end{itemize}
\end{example}



\begin{observation}[Riepilogo della Lezione]
    Gli appunti della Lezione 23 introducono le \textbf{Relazioni di Ricorrenza} per analizzare $T(n)$ degli algoritmi ricorsivi. Si concentrano sulle \textbf{Relazioni Bilanciate} ($T(n) = aT(n/b) + f(n)$). Dopo aver elencato quattro metodi di risoluzione, si focalizzano sul \textbf{Master Theorem}.

    Il teorema confronta $f(n)$ con lo "spartiacque" $n^{\log_b a}$ e definisce tre casi:
    \begin{enumerate}
        \item \textbf{Caso 1 ($f(n)$ minore):} Soluzione: $\Theta(n^{\log_b a})$.
        \item \textbf{Caso 2 ($f(n)$ uguale):} Soluzione: $\Theta(n^{\log_b a} \cdot \log n)$ (o $\log^{k+1} n$).
        \item \textbf{Caso 3 ($f(n)$ maggiore):} Soluzione: $\Theta(f(n))$ (con C.R.).
    \end{enumerate}
\end{observation}


\section*{ASA (Esercizi per casa)}

\begin{example}[Esercizi ASA]
    Risolvere le seguenti relazioni di ricorrenza:
    \begin{enumerate}
        \item $T(n) = 3T(\frac{n}{2}) + n^2$
        \item $T(n) = 3T(\frac{n}{4}) + \frac{n}{5} \log n$
    \end{enumerate}
\end{example}

\newpage

\newpage % ===================================================================
% FILE: Lezione24.tex
% ===================================================================

\part{Lezione 24 (13/11/2025)}

\section{Dimostrazione del Teorema Principale}
L'obiettivo è risolvere la relazione di ricorrenza $T(n) = aT(n/b) + f(n)$.
Si può derivare la formula generale usando l'albero di ricorsione o il metodo iterativo.
\subsection{Metodo Iterativo (Derivazione della Formula)}
Si espande la ricorrenza sostituendo $T(n)$ dentro sé stessa.
% --- APPLICAZIONE STILE ---
\begin{definition}[Formula Generale (Metodo Iterativo)]
    Partiamo dalla ricorrenza:
    \[ T(n) = aT(n/b) + f(n) \]
    Sostituiamo $T(n/b)$ nell'equazione:
    \[ T(n) = a \left[ aT(n/b^2) + f(n/b) \right] + f(n) = a^2 T(n/b^2) + af(n/b) + f(n) \]
    Sostituiamo $T(n/b^2)$ nell'equazione:
    \[ T(n) = a^2 \left[ aT(n/b^3) + f(n/b^2) \right] + af(n/b) + f(n) = a^3 T(n/b^3) + a^2 f(n/b^2) + af(n/b) + f(n) \]

    Dopo $i$ passi, la formula generale è:
    \[ T(n) = a^i T(n/b^i) + \sum_{j=0}^{i-1} a^j f(n/b^j) \]
    Ci si ferma al caso base quando la dimensione del problema è 1, cioè $n/b^i = 1$, che avviene quando $i = \log_b n$.
    Sostituendo $i = \log_b n$:
    \[ T(n) = a^{\log_b n} T(1) + \sum_{j=0}^{\log_b n - 1} a^j f(n/b^j) \]
    Usando l'identità $a^{\log_b n} = n^{\log_b a}$ (dimostrata sotto) e sapendo che $T(1) = \Theta(1)$, la formula finale del costo è:
    \[ T(n) = \Theta(n^{\log_b a}) + \sum_{j=0}^{\log_b n - 1} a^j f(n/b^j) \]
    Questo costo totale è la somma di due parti:
    \begin{itemize}
        \item \textbf{$\Theta(n^{\log_b a})$}: Il costo per la soluzione dei casi base (le foglie dell'albero).
        \item \textbf{$\sum a^j f(n/b^j)$}: Il costo totale del lavoro di "Divide" e "Combine" speso a tutti i livelli della ricorsione.
    \end{itemize}
\end{definition}

% --- APPLICAZIONE STILE ---
\begin{observation}[Identità delle Foglie: $a^{\log_b n} = n^{\log_b a}$]
    \textbf{Dimostrazione:}
    Si parte da $a^{\log_b n}$.
    Si applica la proprietà $x = n^{\log_n x}$:
    \[ a^{\log_b n} = (n^{\log_n a})^{\log_b n} \]
    Si applica la formula del cambio di base $\log_n a = \frac{\log_b a}{\log_b n}$:
    \[ a^{\log_b n} = \left( n^{\frac{\log_b a}{\log_b n}} \right)^{\log_b n} \]
    Moltiplicando gli esponenti:
    \[ a^{\log_b n} = n^{\frac{\log_b a}{\log_b n} \cdot \log_b n} = n^{\log_b a} \]
\end{observation}

\subsection{Analisi dei Casi del Teorema}
L'analisi consiste nel determinare quale dei due termini della formula $T(n) = \Theta(n^{\log_b a}) + \sum...$ domina.
% --- APPLICAZIONE STILE ---
\begin{definition}[Caso 1: $f(n)$ polinomialmente minore]
    \begin{itemize}
        \item \textbf{Condizione:} $f(n) \in O(n^{\log_b a - \epsilon})$ per $\epsilon > 0$.
        \item \textbf{Analisi:} La sommatoria $\sum_{j=0}^{\log_b n - 1} a^j f(n/b^j)$ può essere analizzata come una serie geometrica.
        Sostituendo la condizione, si dimostra che la somma cresce più lentamente del primo termine (la ragione della serie è $r = b^\epsilon > 1$).
        \item \textbf{Logica:} Il costo è dominato dal lavoro svolto nei casi base (le foglie).
        \item \textbf{Soluzione:} $T(n) \in \Theta(n^{\log_b a})$.
    \end{itemize}
\end{definition}

% --- APPLICAZIONE STILE ---
\begin{definition}[Caso 2: $f(n)$ bilanciato (caso $k=0$)]
    \begin{itemize}
        \item \textbf{Condizione:} $f(n) = \Theta(n^{\log_b a})$.
        \item \textbf{Analisi:} Partiamo dalla formula $T(n) = \Theta(n^{\log_b a}) + \sum_{j=0}^{\log_b n - 1} a^j f(n/b^j)$.
        Sostituiamo la condizione $f(n/b^j) = \Theta((n/b^j)^{\log_b a})$ nella sommatoria:
        \[ \sum_{j=0}^{\log_b n - 1} a^j \cdot \left(\frac{n}{b^j}\right)^{\log_b a} = \sum_{j=0}^{\log_b n - 1} a^j \cdot \frac{n^{\log_b a}}{(b^{\log_b a})^j} = \sum_{j=0}^{\log_b n - 1} a^j \cdot \frac{n^{\log_b a}}{a^j} \]
        Semplificando $a^j$, otteniamo:
        \[ \sum_{j=0}^{\log_b n - 1} n^{\log_b a} = n^{\log_b a} \sum_{j=0}^{\log_b n - 1} 1 = n^{\log_b a} \cdot (\log_b n) \]
        \item \textbf{Logica:} Il costo del lavoro extra è bilanciato con il costo delle foglie.
        Il costo totale è il costo di un livello ($n^{\log_b a}$) moltiplicato per il numero di livelli ($\log n$).
        \item \textbf{Soluzione:} $T(n) = \Theta(n^{\log_b a}) + \Theta(n^{\log_b a} \cdot \log n) = \Theta(n^{\log_b a} \cdot \log n)$.
    \end{itemize}
\end{definition}

% --- INIZIO GRAFICO RICOSTRUITO ---
\begin{figure}[h!]
    \centering
    \begin{tikzpicture}[
        level distance=2cm,
        sibling distance=4cm,
        level 1/.style={sibling distance=5cm},
        level 2/.style={sibling distance=2.5cm},
        level 3/.style={sibling distance=1.5cm, font=\small},
        level 4/.style={font=\tiny},
        every node/.style={align=center}
        ]

        % --- L'ALBERO ---
        % Livello 0 (Radice)
        \node (root) {$cn^2$}
        % Livello 1
        child { node (l1_1) {$c(\frac{n}{4})^2$}
        % Livello 2
        child { node (l2_1) {$c(\frac{n}{16})^2$}
        % Livello 3 (Puntini)
        child { node[draw=none] (l3_1) {$\vdots$}
        % Livello 4 (Foglie)
        child { node (l4_1) {$T(1)$} }
        }
        }
        child { node (l2_2) {$c(\frac{n}{16})^2$}
        child { node[draw=none] (l3_2) {$\vdots$}
        child { node (l4_2) {$T(1)$} }
        }
        }
        child { node (l2_3) {$c(\frac{n}{16})^2$}
        child { node[draw=none] (l3_3) {$\vdots$}
        child { node (l4_3) {$T(1)$} }
        }
        }
        }
        child { node (l1_2) {$c(\frac{n}{4})^2$}
        child { node (l2_4) {$c(\frac{n}{16})^2$}
        child { node[draw=none] (l3_4) {$\vdots$}
        child { node (l4_4) {$T(1)$} }
        }
        }
        child { node (l2_5) {$c(\frac{n}{16})^2$}
        child { node[draw=none] (l3_5) {$\vdots$}
        child { node (l4_5) {$T(1)$} }
        }
        }
        child { node (l2_6) {$c(\frac{n}{16})^2$}
        child { node[draw=none] (l3_6) {$\vdots$}
        child { node (l4_6) {$T(1)$} }
        }
        }
        }
        child { node (l1_3) {$c(\frac{n}{4})^2$}
        child { node (l2_7) {$c(\frac{n}{16})^2$}
        child { node[draw=none] (l3_7) {$\vdots$}
        child { node (l4_7) {$T(1)$} }
        }
        }
        child { node (l2_8) {$c(\frac{n}{16})^2$}
        child { node[draw=none] (l3_8) {$\vdots$}
        child { node (l4_8) {$T(1)$} }
        }
        }
        child { node (l2_9) {$c(\frac{n}{16})^2$}
        child { node[draw=none] (l3_9) {$\vdots$}
        child { node (l4_9) {$T(1)$} }
        }
        }
        };
        % --- PUNTINI CENTRALI TRA LE FOGLIE ---
        \node[right=of l4_5, node distance=1.5cm, draw=none] (dots_mid) {$\dots$};
        \node[right=of l4_6, node distance=1.5cm, draw=none] (dots_mid2) {$\dots$};

        % --- ANNOTAZIONI A DESTRA (COSTI PER LIVELLO) ---
        \node[right=of root, node distance=6cm, draw=none] (cost0) {$cn^2$};
        \node[right=of l1_3, node distance=3.5cm, draw=none] (cost1) {$\frac{3}{16} cn^2$};
        \node[right=of l2_9, node distance=2.5cm, draw=none] (cost2) {$(\frac{3}{16})^2 cn^2$};
        \node[below=of cost2, draw=none, node distance=1cm] (cost_dots) {$\vdots$};

        \draw[dotted, thick] (root.east) -- (cost0.west);
        \draw[dotted, thick] (l1_3.east) -- (cost1.west);
        \draw[dotted, thick] (l2_9.east) -- (cost2.west);

        % --- ANNOTAZIONI IN BASSO (COSTO FOGLIE E CONTEGGIO) ---
        \node[below=of l4_5, node distance=1.5cm, draw=none] (leaves_cost) {$\Theta(n^{\log_4 3})$};
        \draw [decorate, decoration={brace, amplitude=10pt, mirror}] (l4_1.south west) -- (l4_9.south east)
        node [midway, below, yshift=-10pt] {$n^{\log_4 3}$ foglie};
        % --- ANNOTAZIONE A SINISTRA (ALTEZZA) ---
        \draw[<->, thick] (root.north west) ++ (-6.5cm, 0.5cm) -- (l4_1.south west) ++ (-1.5cm, -0.5cm)
        node [midway, left, xshift=-5pt] {$\log_4 n$};
        % --- TOTALE ---
        \node[below=of leaves_cost, node distance=1.5cm, right=of leaves_cost, xshift=6cm, draw=none, font=\Large] (total) {Totale: $O(n^2)$};
    \end{tikzpicture}
    \caption{Visualizzazione dell'albero di ricorsione per $T(n) = 3T(n/4) + cn^2$.
    Questo è un esempio del \textbf{Caso 3} del Master Theorem, dove il costo è dominato dalla radice (root).}
    \label{fig:master-tree-case3}
\end{figure}
% --- FINE GRAFICO ---


\section{Esercizio (Compitino 24-25)}
Analizzare la complessità di un algoritmo la cui struttura (semplificata) è la seguente, ipotizzando diversi costi per il lavoro $f(n)$.
% --- APPLICAZIONE STILE ---
\begin{example}[Analisi Algoritmo Ricorsivo]
    Dato il seguente algoritmo:
    \begin{algorithmic}[1]
        \Procedure{ALGO}{A, p, r}
            \If{$p < r$}
                \State $q = \lfloor (p+r)/2 \rfloor$
                \State \Call{ALGO}{A, p, q} \Comment{Costo $T(n/2)$}
                \State \Call{ALGO}{A, q+1, r} \Comment{Costo $T(n/2)$}
                \State \Call{ALGO}{A, p, q} \Comment{Costo $T(n/2)$}
                \State \Call{ALGO}{A, q+1, r} \Comment{Costo $T(n/2)$}
                \State... (Lavoro extra con costo $f(n)$)...
            \EndIf
        \EndProcedure
    \end{algorithmic}

    \begin{explanation}{Analisi Ricorsiva}
    L'algoritmo divide il problema in 2 metà ($n/2$) ma effettua \textbf{4 chiamate ricorsive} ($a=4$).
    Questo porta a un costo elevato che domina il lavoro locale $f(n)$, a meno che $f(n)$ non sia molto pesante.
    \end{explanation}

    L'algoritmo fa $a=4$ chiamate ricorsive su sottoproblemi di dimensione $n/2$ (quindi $b=2$).
    % --- APPLICAZIONE STILE ---
    \begin{observation}[Caso Pessimo: $f(n) = n^2$]
        La relazione di ricorrenza è: $T(n) = 4T(n/2) + n^2$.
        \begin{itemize}
            \item $a = 4$, $b = 2$.
            \item \textbf{Spartiacque:} $n^{\log_b a} = n^{\log_2 4} = n^2$.
            \item \textbf{Confronto:} $f(n) = n^2$ è uguale allo spartiacque.
            \item Siamo nel \textbf{Caso 2} (con $k=0$).
            \item \textbf{Soluzione:} $T(n) = \Theta(n^{\log_b a} \cdot \log n) = \Theta(n^2 \cdot \log n)$.
        \end{itemize}
    \end{observation}

    % --- APPLICAZIONE STILE ---
    \begin{observation}[Caso Ottimo (ipotetico): $f(n) = n$]
        La relazione di ricorrenza è: $T(n) = 4T(n/2) + n$.
        \begin{itemize}
            \item $a = 4$, $b = 2$. \textbf{Spartiacque:} $n^2$.
            \item \textbf{Confronto:} $f(n) = n$ è polinomialmente minore di $n^2$ (poiché $n = O(n^{2-\epsilon})$ per $\epsilon=1$).
            \item Siamo nel \textbf{Caso 1}.
            \item \textbf{Soluzione:} $T(n) = \Theta(n^{\log_b a}) = \Theta(n^2)$.
        \end{itemize}
    \end{observation}

\end{example}
\newpage

\newpage \input{extra/precisazione_01}
\newpage % ===================================================================
% FILE: Precizazione1.tex
% ===================================================================

\part*{Lezione 9 (20/10/2025) }

\section*{Esercitazione Master's Theorem}
\subsection*{Domande}

\begin{itemize}
    \item[(A)] $T(n) = 7 T(n/2) + n^2 \quad \forall n \ge n_0$
    \item[(A')] $T'(n) = a T(n/4) + n^2 \quad \forall n \ge n_0'$
\end{itemize}

\paragraph{Domanda (A):} Qual è il più grande valore di $a$ per cui $T'$ è asintoticamente $<$ (\emph{minore del}) valore di $T$?

\paragraph{Domanda (A'):} Qual è il più grande valore di $a$ per cui $T'$ è asintoticamente uguale al valore di $T$?

\subsection*{Introduzione all'esercizio}
\paragraph{In questa parte dell'esercizio abbiamo come scopo riportare in una forma adeguata i nostri valori}
\subsubsection*{Costo in tempo di A}
\[ T(n) = 7 T(n/2) + n^2 \]
\begin{itemize}
    \item $a=7, b=2, f(n)=n^2$
    \item $\log_b a \implies \log_2 7$
    \item $f(n) = n^2 \in O(n^{\log_2 7 - \epsilon})$
    \item $0 < \epsilon \le \log_2 7 - 2$
    \item $1^\circ \text{ CASO} \implies T(n) = \Theta(n^{\log_2 7})$ \quad (\emph{circa $n^{2.8...}$})
\end{itemize}

\subsubsection*{Costo in tempo di A'}
\[ T'(n) = a T(n/4) + n^2 \]
\begin{itemize}
    \item $a=a, b=4, f(n)=n^2$
    \item $\log_b a \implies \log_4 a$
    \item Quale caso del Teorema?
    \begin{itemize}
        \item Ramo 1: $\log_4 a < 2 \implies a < 16$ \quad (\emph{Caso 3?})
        \item Ramo 2: $\log_4 a = 2 \implies a = 16$ \quad (\emph{Caso 2?})
        \item Ramo 3: $\log_4 a > 2 \implies a > 16$ \quad (\emph{Caso 1})
    \end{itemize}
\end{itemize}

\section*{Logica per risolvere}
L'obiettivo è usare il Teorema Master per risolvere le ricorrenze. Il teorema si applica a ricorrenze della forma:
Si calcola il valore critico $\log_b a$ e lo si confronta con l'esponente di $f(n)$ (supponendo $f(n) = n^k$).

\subsection*{Step 1: Analisi di T(n) (Ricorrenza A)}
La prima ricorrenza è $T(n) = 7 T(n/2) + n^2$.
\begin{itemize}
    \item \textbf{Identificazione parametri:}
    \begin{itemize}
        \item $a=7$
        \item $b=2$
        \item $f(n) = n^2$
    \end{itemize}
    \item \textbf{Calcolo esponente critico:}
    \begin{itemize}
        \item Calcoliamo $\log_b a = \log_2 7$.
        \item Sappiamo che $2^2 = 4$ e $2^3 = 8$, quindi $\log_2 7$ è un numero tra 2 e 3 (circa 2.81).
    \end{itemize}
    \item \textbf{Confronto e applicazione Teorema Master:}
    \begin{itemize}
        \item Confrontiamo $f(n) = n^2$ con $n^{\log_b a} = n^{\log_2 7}$.
        \item Poiché $2 < \log_2 7$, $f(n)$ è polinomialmente più piccola di $n^{\log_b a}$.
        \item Questo corrisponde al \textbf{Caso 1} del Teorema Master: $f(n) = O(n^{\log_b a - \epsilon})$, dove $\epsilon = \log_2 7 - 2 > 0$.
        \item La soluzione è quindi $T(n) = \Theta(n^{\log_b a})$.
    \end{itemize}
\end{itemize}

\begin{quote}
    \textbf{Risultato per T(n): $T(n) = \Theta(n^{\log_2 7})$}
\end{quote}

\subsection*{Step 2: Analisi di T'(n) (Ricorrenza A')}
La seconda ricorrenza è $T'(n) = a T(n/4) + n^2$.
\begin{itemize}
    \item \textbf{Identificazione parametri:}
    \begin{itemize}
        \item $a = a$ (sconosciuto)
        \item $b = 4$
        \item $f(n) = n^2$
    \end{itemize}
    \item \textbf{Calcolo esponente critico:}
    \begin{itemize}
        \item L'esponente critico è $\log_b a = \log_4 a$.
    \end{itemize}
    \item \textbf{Confronto e applicazione Teorema Master:}
    \begin{itemize}
        \item Dobbiamo confrontare $\log_4 a$ con l'esponente di $f(n)$, che è 2.
        \item Questo crea tre scenari possibili:
        \item \textbf{Scenario 1 (Caso 1 del Teorema): $\log_4 a > 2$}
        \begin{itemize}
            \item Questo succede quando $a > 4^2$, cioè $a > 16$.
            \item La soluzione è dominata dalla ricorsione: $T'(n) = \Theta(n^{\log_4 a})$.
        \end{itemize}
        \item \textbf{Scenario 2 (Caso 2 del Teorema): $\log_4 a = 2$}
        \begin{itemize}
            \item Questo succede quando $a = 4^2$, cioè $a = 16$.
            \item La soluzione è: $T'(n) = \Theta(n^{\log_b a} \log n) = \Theta(n^2 \log n)$.
        \end{itemize}
        \item \textbf{Scenario 3 (Caso 3 del Teorema): $\log_4 a < 2$}
        \begin{itemize}
            \item Questo succede quando $a < 16$.
            \item La soluzione è dominata dal costo $f(n)$: $T'(n) = \Theta(n^2)$ (assumendo la condizione di regolarità, che è soddisfatta).
        \end{itemize}
    \end{itemize}
\end{itemize}

\hrule

\subsection*{Step 3: Risposta alle Domande}
Ora usiamo i risultati degli Step 1 e 2 per rispondere alle domande.

\paragraph{Domanda (A'): Trovare $a$ t.c. $T'(n)$ è uguale a $T(n)$}
Vogliamo trovare il più grande $a$ per cui $T'(n)$ è asintoticamente uguale a $T(n)$.

Controlliamo quale dei nostri 3 scenari per $T'(n)$ può soddisfare questa uguaglianza:
\begin{itemize}
    \item \textbf{Se $a > 16$ (Scenario 1):} $T'(n) = \Theta(n^{\log_4 a})$.
    \begin{itemize}
        \item Dobbiamo avere $\Theta(n^{\log_4 a}) = \Theta(n^{\log_2 7})$.
        \item Questo richiede che gli esponenti siano uguali: $\log_4 a = \log_2 7$.
        \item Risolviamo per $a$ (usando il cambio di base: $\log_4 a = \frac{\log_2 a}{\log_2 4} = \frac{\log_2 a}{2}$):
        \[ \frac{\log_2 a}{2} = \log_2 7 \]
        \[ \log_2 a = 2 \log_2 7 \]
        \[ \log_2 a = \log_2 (7^2) \]
        \[ a = 49 \]
        \item Questo valore $a=49$ è coerente con la condizione $a > 16$.
    \end{itemize}
    \item \textbf{Se $a = 16$ (Scenario 2):} $T'(n) = \Theta(n^2 \log n)$.
    \begin{itemize}
        \item $n^2 \log n$ non è asintoticamente uguale a $n^{\log_2 7}$ (che è $\approx n^{2.81}$).
    \end{itemize}
    \item \textbf{Se $a < 16$ (Scenario 3):} $T'(n) = \Theta(n^2)$.
    \begin{itemize}
        \item $n^2$ non è asintoticamente uguale a $n^{\log_2 7}$.
    \end{itemize}
\end{itemize}
\textbf{Risposta (A'):} L'unico valore (e quindi il più grande) per cui $T'(n)$ è asintoticamente uguale a $T(n)$ è $a = 49$.

\paragraph{Domanda (A): Trovare $a$ t.c. $T'(n)$ è minore di $T(n)$}
Vogliamo trovare il più grande $a$ per cui $T'(n)$ è asintoticamente minore di $T(n)$ (cioè $T'(n) = o(T(n))$).

Controlliamo di nuovo i nostri 3 scenari:
\begin{itemize}
    \item \textbf{Se $a > 16$ (Scenario 1):} $T'(n) = \Theta(n^{\log_4 a})$.
    \begin{itemize}
        \item Vogliamo $n^{\log_4 a} = o(n^{\log_2 7})$.
        \item Questo è vero se l'esponente $\log_4 a$ è strettamente minore di $\log_2 7$.
        \item $\log_4 a < \log_2 7 \implies a < 49$ (dal calcolo precedente).
        \item Questo scenario è valido per l'intervallo $16 < a < 49$.
    \end{itemize}
    \item \textbf{Se $a = 16$ (Scenario 2):} $T'(n) = \Theta(n^2 \log n)$.
    \begin{itemize}
        \item Vogliamo $n^2 \log n = o(n^{\log_2 7})$.
        \item Poiché $2 < \log_2 7 \approx 2.81$, $n^2 \log n$ cresce più lentamente di $n^{\log_2 7}$. L'affermazione è vera.
        \item Quindi $a = 16$ è una soluzione.
    \end{itemize}
    \item \textbf{Se $a < 16$ (Scenario 3):} $T'(n) = \Theta(n^2)$.
    \begin{itemize}
        \item Vogliamo $n^2 = o(n^{\log_2 7})$.
        \item Poiché $2 < \log_2 7$, $n^2$ cresce più lentamente di $n^{\log_2 7}$. L'affermazione è vera.
        \item Questo scenario è valido per $1 \le a < 16$.
    \end{itemize}
\end{itemize}
Unendo tutti i casi validi, $T'(n)$ è asintoticamente minore di $T(n)$ per ogni $a$ nell'intervallo $1 \le a < 49$.

\textbf{Risposta (A):} La domanda chiede il più grande valore di $a$. Se $a$ può essere un numero reale, non esiste un "più grande" valore (il limite è 49). Se si intende il più grande valore intero, la risposta è $a = 48$.

\textbf{Ricorda bene} che $\log n$ va sempre più veloce di qualsiasi polinomio di $n$. Se dovessimo cercare qualcosa, $\log n$ va più veloce di qualunque polinomio di $n$.

\newpage

\newpage \input{extra/approfondimento_02}
\newpage % ===================================================================
% FILE: Lezione25.tex
% ===================================================================

\part{Lezione 25 (17/11/2025)}

\section{Esercizio su Teorema Master (Confronto Asintotico)}

Consideriamo due algoritmi caratterizzati dalle seguenti ricorrenze:
\begin{itemize}
    \item[(A)] $T(n) = 7 T\left(\frac{n}{2}\right) + n^2$
    \item[(A')] $T'(n) = a T'\left(\frac{n}{4}\right) + n^2$
\end{itemize}

\textbf{Domanda:} Qual è il più grande valore di $a$ per cui $T'$ è asintoticamente più veloce di $T$?

\subsection{Costo in Tempo di A}
Analizziamo $T(n) = 7 T(n/2) + n^2$ con il Teorema Master.
\begin{itemize}
    \item $a=7, b=2, f(n)=n^2$.
    \item Calcoliamo lo spartiacque: $n^{\log_b a} = n^{\log_2 7} \approx n^{2.8}$.
    \item Confrontiamo con $f(n)$: $n^2 = O(n^{\log_2 7 - \epsilon})$ (con $\epsilon \approx 0.8$).
    \item Siamo nel \textbf{Caso 1}.
\end{itemize}
\[ T(n) = \Theta(n^{\log_2 7}) \]

\subsection{Costo in Tempo di A'}
Analizziamo $T'(n) = a T'(n/4) + n^2$.
\begin{itemize}
    \item $a=a, b=4, f(n)=n^2$.
    \item Spartiacque: $n^{\log_4 a}$.
\end{itemize}
Dobbiamo confrontare $\log_4 a$ con l'esponente di $f(n)$ (che è 2). Ci sono 3 casi possibili per $a$:

\begin{enumerate}
    \item \textbf{Caso 3 ($\log_4 a < 2 \iff a < 16$):}
    La forzante $n^2$ domina. $T'(n) = \Theta(n^2)$.
    Verifica condizione regolarità: $a(n/4)^2 \le c n^2 \implies a/16 \le c$. Vera per $a < 16$.
    In questo caso $T'(n) = \Theta(n^2)$, che è sicuramente più veloce di $\Theta(n^{2.8})$.

    \item \textbf{Caso 2 ($\log_4 a = 2 \iff a = 16$):}
    Equilibrio. $T'(n) = \Theta(n^2 \log n)$.
    Anche questo è più veloce di $\Theta(n^{2.8})$.

    \item \textbf{Caso 1 ($\log_4 a > 2 \iff a > 16$):}
    Le foglie dominano. $T'(n) = \Theta(n^{\log_4 a})$.
    Affinché $T'$ sia più veloce di $T$, deve valere:
    \[ n^{\log_4 a} < n^{\log_2 7} \implies \log_4 a < \log_2 7 \]
    Usando il cambio di base ($\log_4 a = \frac{\log_2 a}{2}$):
    \[ \frac{\log_2 a}{2} < \log_2 7 \implies \log_2 a < 2 \log_2 7 \implies \log_2 a < \log_2 49 \implies a < 49 \]
\end{enumerate}

\textbf{Soluzione:} $A'$ è più veloce di $A$ per $a < 49$. Il valore intero massimo è \textbf{48}.

\section{Analisi di Algoritmi (Esercizi Vari)}

\subsection{Esercizio "Mistero"}
\begin{algorithmic}[1]
    \Procedure{Mistero}{n}
        \If{$n < 10$} \Return 1 \EndIf
        \State $x = \Call{Mistero}{\lfloor n/4 \rfloor} + \Call{Mistero}{\lfloor n/4 \rfloor}$ \Comment{2 chiamate ricorsive}
        \State $L=1$
        \While{$i < n$} \Comment{Ciclo esterno}
            \State $j=1$
            \While{$j < n$} \Comment{Ciclo interno}
                \State...
                \State $j = j+1$
            \EndWhile
            \State $i = i \cdot 3$
        \EndWhile
        \State $y = \Call{Mistero}{\lfloor n/4 \rfloor}$ \Comment{1 chiamata ricorsiva}
        \State \Return $x+y$
    \EndProcedure
    \end{algorithmic}

\begin{explanation}{Analisi Ricorsiva}
La funzione combina chiamate ricorsive e cicli annidati:
\begin{itemize}
    \item \textbf{Ricorsione}: 3 chiamate su dimensione $n/4$ ($x$ ne fa 2, $y$ ne fa 1). $T(n) = 3T(n/4) + f(n)$.
    \item \textbf{Costo locale ($f(n)$)}: I cicli annidati. Il ciclo interno lavora su $j$, l'esterno su $i$ che cresce esponenzialmente ($3^k$).
\end{itemize}
\end{explanation}

    \textbf{Analisi:}
    \begin{itemize}
        \item Chiamate ricorsive: 3 chiamate su $n/4$. Quindi $a=3, b=4$.
        \item Costo f(n): Il ciclo interno è $\Theta(n)$, quello esterno è logaritmico? (Dagli appunti sembra indicato come $\Theta(n \log n)$).
        \item Ricorrenza: $T(n) = 3T(n/4) + \Theta(n \log n)$.
        \item Master Theorem:
        \begin{itemize}
            \item $n^{\log_4 3} \approx n^{0.79}$.
            \item $f(n) = n \log n$ è polinomialmente maggiore ($n^1 > n^{0.79}$).
            \item \textbf{Caso 3}.
        \end{itemize}
        \item Soluzione: $T(n) = \Theta(n \log n)$.
    \end{itemize}

    \subsection{Algo 1 (Radice Quadrata)}
    \begin{itemize}
        \item Ricorrenza data: $T(n) = 2T(n/4) + \sqrt{n}$.
        \item $a=2, b=4 \implies n^{\log_4 2} = n^{0.5} = \sqrt{n}$.
        \item $f(n) = \sqrt{n}$.
        \item Siamo nel \textbf{Caso 2} (uguali).
        \item Soluzione: $T(n) = \Theta(\sqrt{n} \log n)$.
    \end{itemize}

    \subsection{Algo 2 (Somma Ricorsiva)}
    \begin{itemize}
        \item Divide l'array in due parti ($p, q$ e $q+1, r$).
        \item Fa 2 chiamate ricorsive: $a=2, b=2$.
        \item Costo di combinazione (somma): $\Theta(1)$.
        \item Ricorrenza: $T(n) = 2T(n/2) + \Theta(1)$.
        \item Master Theorem: $n^{\log_2 2} = n^1$. $f(n) = n^0$.
        \item \textbf{Caso 1}.
        \item Soluzione: $T(n) = \Theta(n)$.
    \end{itemize}

    \subsection{Confronto Finale}
    Confrontiamo:
    \begin{enumerate}
        \item $T_A(n) = 9T(n/3) + 2n^2$.
        \item $T_{A'}(n) = 3T(n/2) + n^2 \log^2 n$.
    \end{enumerate}

    \textbf{Analisi A:}
    $a=9, b=3 \implies n^{\log_3 9} = n^2$. $f(n) = 2n^2$.
    Caso 2. $T_A(n) = \Theta(n^2 \log n)$.

    \textbf{Analisi A':}
    $a=3, b=2 \implies n^{\log_2 3} \approx n^{1.58}$. $f(n) = n^2 \log^2 n$.
    $f(n)$ è maggiore dello spartiacque. Caso 3.
    $T_{A'}(n) = \Theta(n^2 \log^2 n)$.

    \textbf{Conclusione:} $T_A$ è asintoticamente migliore (più veloce) di $T_{A'}$.

    \newpage

\newpage % ===================================================================
% FILE: Lezione26.tex
% ===================================================================

\part{Lezione 26 (19/11/2025)}

\section{Limiti Inferiori alla Difficoltà di un Problema}

Vogliamo stabilire quanto è "difficile" un problema $\mathcal{P}$ intrinsecamente, indipendentemente dall'algoritmo usato.

\begin{definition}[Limite Inferiore]
    Il \textbf{Limite Inferiore} $L(n)$ misura la difficoltà di un problema $\mathcal{P}$ in funzione della dimensione $n$ dell'input.
    Rappresenta la complessità al \textbf{caso pessimo} del \textbf{miglior algoritmo possibile} che risolve $\mathcal{P}$.
    \[ \forall \text{ algoritmo } A \text{ che risolve } \mathcal{P}, \quad T_A(n) \ge L(n) \quad (\text{nel caso pessimo}) \]
    Ovvero, $L(n)$ è il minimo numero di operazioni necessarie per risolvere il caso pessimo.
\end{definition}

\subsection{Esempio Introduttivo: Ricerca}
Consideriamo il problema $\mathcal{P}$: Ricerca di una chiave in un vettore \textbf{ordinato}.

\begin{itemize}
    \item \textbf{Approccio 1: Scansione Lineare}. Complessità $O(n)$.
    Un algoritmo che risolve $\mathcal{P}$ fornisce un \textbf{Limite Superiore} alla difficoltà di $\mathcal{P}$.
    \item \textbf{Approccio 2: Ricerca Binaria}. Complessità $O(\log n)$.
    Migliora il limite superiore.
    \item \textbf{Domanda:} Posso fare di meglio? Qual è il \textbf{Limite Inferiore} (la "pavimentazione") sotto il quale non posso scendere?
\end{itemize}

\section{Criteri per Stabilire i Limiti Inferiori}

\subsection{1° Criterio: Dimensione dell'Input}
Se la soluzione di un problema richiede, nel caso pessimo, l'esame di tutti i dati in ingresso, allora la dimensione dell'input $n$ è un limite inferiore.
\[ L(n) = \Omega(n) \]

\begin{example}[Ricerca MAX in Vettore Non Ordinato]
    Per trovare il massimo in un vettore non ordinato, devo necessariamente analizzare tutti gli $n$ elementi (altrimenti il massimo potrebbe essere proprio l'elemento saltato).
    \begin{itemize}
        \item \textbf{Limite Inferiore:} $L(n) = n$.
        \item \textbf{Algoritmo noto:} Scansione Lineare, costo $O(n)$.
    \end{itemize}
    Poiché il costo dell'algoritmo ($n$) coincide con il limite inferiore ($n$), l'algoritmo di \textbf{Scansione Lineare è OTTIMO}.
\end{example}

\subsection{2° Criterio: Albero di Decisione}
Questo criterio si applica a problemi risolvibili attraverso una sequenza di "decisioni" (es. confronti tra valori) che riducono via via lo spazio delle soluzioni possibili.

\subsubsection{Struttura dell'Albero di Decisione}
Possiamo modellare l'esecuzione di un algoritmo basato su confronti come un albero:
\begin{itemize}
    \item \textbf{Nodo Interno:} Rappresenta un confronto/decisione (es. $x > y$?).
    \item \textbf{Foglia:} Rappresenta una possibile \textbf{soluzione} finale.
    \item \textbf{Cammino Radice-Foglia:} Rappresenta una specifica esecuzione dell'algoritmo su un dato input.
\end{itemize}

\subsubsection{Relazione con la Complessità}
\begin{itemize}
    \item \textbf{Caso Ottimo:} Cammino più breve dalla radice a una foglia.
    \item \textbf{Caso Pessimo:} Cammino più lungo dalla radice a una foglia, ovvero l'\textbf{Altezza dell'Albero}.
\end{itemize}
Per minimizzare il caso pessimo, vogliamo che l'albero sia il più \textbf{bilanciato} possibile (altezza minima per un dato numero di foglie).

\begin{theorem}[Limite Inferiore basato sulle Soluzioni]
    Sia $SOL(n)$ il numero di possibili soluzioni distinte per un problema di dimensione $n$ (ovvero il numero di foglie dell'albero).
    In un albero binario (o ternario), l'altezza $h$ deve soddisfare:
    \[ h \ge \log_2(\#\text{foglie}) \]
    Pertanto, il limite inferiore è dato dal logaritmo del numero delle possibili soluzioni:
    \[ L(n) = \Omega(\log_2(SOL(n))) \]
\end{theorem}

\begin{observation}
    L'algoritmo migliore al caso pessimo è quello che minimizza l'altezza dell'albero di decisione, ovvero quello che ha altezza logaritmica rispetto al numero di foglie.
\end{observation}

\subsubsection{Applicazione: Ricerca in Vettore Ordinato}
Analizziamo il problema della ricerca di una chiave $k$ in un array ordinato $A[1..n]$ usando il criterio dell'Albero di Decisione.
\begin{itemize}
    \item \textbf{Possibili Soluzioni ($SOL(n)$):} L'elemento può trovarsi in una delle $n$ posizioni, oppure non esserci.
    \[ \#\text{Soluzioni} = n + 1 \]
    \item \textbf{Limite Inferiore:}
    \[ L(n) = \log_2(n+1) \approx \log_2 n \]
    \item \textbf{Confronto:} L'algoritmo \textbf{Ricerca Binaria} ha costo $O(\log n)$.
\end{itemize}
Poiché il costo dell'algoritmo coincide con il limite inferiore, la \textbf{Ricerca Binaria è OTTIMA}. Non è necessario (né possibile) fare di meglio basandosi sui confronti.

\subsection{3° Criterio: Eventi Contabili (Avversario)}
Se la ripetizione di un certo evento è indispensabile per risolvere il problema, allora:
\[ L(n) = (\#\text{volte che si deve ripetere}) \times (\text{costo evento}) \]

\begin{example}[Ricerca MAX in Array con Confronti]
    Evento necessario: Un elemento, per non essere il massimo, deve "uscire perdente" da un confronto con un altro valore.
    \begin{itemize}
        \item Abbiamo $n$ candidati al massimo.
        \item Alla fine deve rimanere 1 solo vincitore.
        \item Devono esserci quindi $n-1$ "perdenti".
        \item Ogni confronto elimina al massimo 1 candidato (il perdente).
    \end{itemize}
    Necessari almeno \textbf{$n-1$} confronti.
    \[ L(n) = \Omega(n) \]
\end{example}

\section{Osservazione Finale: Confronto tra Criteri}
Consideriamo il problema: \textbf{Ricerca di $k$ in Vettore NON Ordinato}.
\begin{itemize}
    \item \textbf{Criterio Albero di Decisione:}
    \[ \#\text{Soluzioni} = n+1 \implies L(n) = \Omega(\log n) \]
    Questo è un limite inferiore valido, ma è troppo basso ("largo").

    \item \textbf{Criterio Dimensione Input:}
    Bisogna guardare tutti gli elementi.
    \[ L(n) = \Omega(n) \]
\end{itemize}

Il limite inferiore "vero" (o più significativo) è il più alto tra quelli trovati. In questo caso $\Omega(n)$.
Quindi, per la ricerca non ordinata, la \textbf{Scansione Lineare} (costo $n$) è ottima, mentre un ipotetico algoritmo logaritmico (suggerito dal criterio dell'albero) non è realizzabile.

\newpage

% \input{lezioni/Lezione12}
\newpage 

\part{Confronto tra Algoritmi di Ordinamento}

\section{Confronto tra Algoritmi di Ordinamento}

Viene presentato un confronto sulla complessità temporale degli algoritmi principali studiati.

\begin{table}[h]
    \centering
    \renewcommand{\arraystretch}{1.5}
    \begin{tabular}{|l|c|c|c|}
        \hline
        \textbf{Algoritmo} & \textbf{Caso Ottimo} & \textbf{Caso Medio} & \textbf{Caso Pessimo} \\
        \hline
        \textbf{Merge Sort} & $\Theta(n \log n)$ & $\Theta(n \log n)$ & $\Theta(n \log n)$ \\
        \hline
        \textbf{Insertion Sort} & $\Theta(n)$ & $\Theta(n^2)$ & $\Theta(n^2)$ \\
        \hline
        \textbf{Quick Sort} & $\Theta(n \log n)$ & $\Theta(n \log n)$ & $\Theta(n^2)$ \\
        \hline
    \end{tabular}
    \caption{Confronto complessità temporale}
    \label{tab:confronto_algo}
\end{table}

\section{Quick Sort: L'Idea}

L'algoritmo segue l'approccio \textit{"Divide et Impera"} in tre fasi:

\begin{enumerate}
    \item \textbf{Scelta del Perno (Pivot):} Si sceglie un elemento, ad esempio l'ultimo elemento dell'array: $P = A[r]$.
    \item \textbf{Partizione (Partition):} L'array $A$ viene diviso in due metà (non necessariamente uguali) rispetto al pivot:
    \[ A_1 = \{x \in A \mid x \le P\} \]
    \[ A_2 = \{x \in A \mid x > P\} \]
    Il pivot $P$ viene posizionato tra $A_1$ e $A_2$.
    \item \textbf{Ricorsione:} Si ordinano ricorsivamente i sotto-array $A_1$ e $A_2$.
\end{enumerate}

\begin{algorithm}
    \caption{QuickSort(A, p, r)}
    \begin{algorithmic}[1]
        \State \Comment{Goal: Ordina A[p...r]. Prima chiamata: p=1, r=n}
        \If{$p < r$}
            \State $q \gets \textsc{Partition}(A, p, r)$
            \State $\textsc{QuickSort}(A, p, q-1)$
            \State $\textsc{QuickSort}(A, q+1, r)$
        \EndIf
        \State \Comment{\#elementi = $r - p + 1$}
    \end{algorithmic}
\end{algorithm}

\begin{explanation}{Logica QuickSort}
Algoritmo ricorsivo basato su \textit{Divide et Impera}:
\begin{enumerate}
    \item \textbf{Partition}: Trova la posizione finale del pivot $q$.
    \item \textbf{Ricorsione}: Ordina i sotto-array a sinistra e a destra di $q$.
    \item \textbf{Base}: Se $p \ge r$, l'array ha 0 o 1 elemento ed è già ordinato.
\end{enumerate}
\end{explanation}

\section{Confronto: Merge Sort vs Quick Sort}

Entrambi usano la strategia \textit{Divide et Impera}, ma in modo opposto:

\begin{table}[h]
    \centering
    \renewcommand{\arraystretch}{1.5}
    \begin{tabular}{|p{0.15\textwidth}|p{0.38\textwidth}|p{0.38\textwidth}|}
        \hline
        \textbf{Fase} & \textbf{Merge Sort} & \textbf{Quick Sort} \\
        \hline
        \textbf{Divide} & \textbf{Banale:} $q = \lfloor(p+r)/2\rfloor$. Divide a metà perfetta. & \textbf{Complesso:} $q = \textsc{Partition}(\dots)$. Il lavoro "pesante" viene fatto qui. \\
        \hline
        \textbf{Impera} & $\textsc{MS}(A, p, q)$, $\textsc{MS}(A, q+1, r)$ & $\textsc{QS}(A, p, q-1)$, $\textsc{QS}(A, q+1, r)$ \\
        \hline
        \textbf{Combine} & \textbf{Complesso:} $\textsc{Merge}(\dots)$. Necessaria per unire i risultati. & \textbf{Banale:} Non necessaria (l'array è ordinato "in loco"). \\
        \hline
    \end{tabular}
\end{table}

\section{La Procedura Partition}

La funzione \texttt{Partition} riorganizza l'array in loco (senza array di appoggio) in modo lineare $\Theta(n)$.

\begin{algorithm}
    \caption{PARTITION(A, p, r)}
    \begin{algorithmic}[1]
        \State $x \gets A[r]$ \Comment{Pivot}
        \State $i \gets p - 1$
        \For{$j \gets p$ \textbf{to} $r - 1$}
            \If{$A[j] \le x$}
                \State $i \gets i + 1$
                \State \textbf{Swap}($A[i], A[j]$)
            \EndIf
        \EndFor
        \State \textbf{Swap}($A[i+1], A[r]$) \Comment{Posiziona il pivot}
        \State \textbf{return} $i + 1$
    \end{algorithmic}
\end{algorithm}

\begin{explanation}{Partition (Lomuto)}
Scansiona l'array con un indice $j$. Se $A[j] \le Pivot$, incrementa la regione dei "minori" ($i$) e scambia.
Alla fine, il pivot viene messo esattamente tra i minori e i maggiori.
\end{explanation}

\begin{observation}[Invariante del ciclo]
    Durante l'esecuzione, l'array è diviso in regioni:
    \begin{itemize}
        \item $A[p \dots i]$: Elementi $\le$ Pivot.
        \item $A[i+1 \dots j-1]$: Elementi $>$ Pivot.
        \item $A[j \dots r-1]$: Elementi ancora da esaminare.
        \item $A[r]$: Pivot.
    \end{itemize}
\end{observation}

\begin{example}[Traccia Partition]
    \textbf{Array iniziale:} 2 8 7 1 3 5 6 4 (Pivot = 4). \\
    Durante la partizione, gli elementi minori o uguali a 4 vengono spostati a sinistra. Alla fine, il 4 si troverà nella sua posizione corretta definitiva.
\end{example}

\section{Variante: Hoare Partition}

Viene presentata una variante dell'algoritmo di partizione (Partizione di Hoare) che usa due indici che convergono dagli estremi.

\begin{algorithm}
    \caption{HOARE-PARTITION(A, p, r)}
    \begin{algorithmic}[1]
        \State $x \gets A[p]$ \Comment{Pivot (qui preso come primo elemento)}
        \State $i \gets p - 1$
        \State $j \gets r + 1$
        \While{\textbf{true}}
            \Repeat
                \State $j \gets j - 1$
            \Until{$A[j] \le x$}
            \Repeat
                \State $i \gets i + 1$
            \Until{$A[i] \ge x$}
            \If{$i < j$}
                \State \textbf{exchange} $A[i]$ with $A[j]$
            \Else
                \State \textbf{return} $j$
            \EndIf
        \EndWhile
    \end{algorithmic}
\end{algorithm}

\begin{explanation}{Hoare Partition}
Due indici ($i, j$) partono dagli estremi e convergono.
\begin{itemize}
    \item $i$ avanza finché trova elementi "piccoli".
    \item $j$ indietreggia finché trova elementi "grandi".
    \item Quando si bloccano entrambi (trovati elementi fuori posto), si scambiano.
\end{itemize}
È spesso più efficiente di Lomuto in pratica (meno swap).
\end{explanation}
% \end{algorithm} removed duplicate

\section{Analisi della Complessità}

Sia $n = r - p + 1$. Il tempo di esecuzione $T(n)$ dipende da come il pivot divide l'array ($q$).
La relazione di ricorrenza generale è:
$$T(n) = T(q-1) + T(n-q) + \Theta(n)$$
dove $\Theta(n)$ è il costo della Partition.

\subsection{Caso Pessimo}
Si verifica quando l'array è \textbf{già ordinato} (o ordinato al contrario).
In questo caso, il pivot (essendo il massimo o il minimo) divide l'array in un sotto-problema di dimensione $n-1$ e uno di dimensione $0$.
L'albero di ricorsione diventa una lista lunga $n$.

\textbf{Calcolo:}
$$T(n) = T(n-1) + T(0) + c \cdot n$$
Sviluppando la somma:
$$T(n) = \sum_{i=1}^{n} i = \Theta(n^2)$$

\subsection{Caso Ottimo}
Si verifica quando il pivot divide l'array sempre a metà (come nel Merge Sort), ovvero $q = n/2$.
\[ \textbf{Complessità: } O(n \log n) \]

\subsection{Caso Medio}
Anche nel caso medio la complessità si dimostra essere:
\[ O(n \log n) \]

\begin{observation}
    Questo è il motivo per cui il Quick Sort viene utilizzato nella pratica: spesso è più veloce del Merge Sort grazie alle costanti nascoste minori e all'uso della memoria in loco (non richiede array ausiliari per il merge).
\end{observation}



\section*{Quick Sort: Rappresentazione Grafica}

\subsection*{1. L'Idea della Partizione}
L'array viene diviso in due parti (non necessariamente uguali) rispetto a un elemento "perno" (Pivot) $P$.
\begin{itemize}
    \item $A_1$: elementi $\le P$
    \item $A_2$: elementi $> P$
\end{itemize}

\vspace{0.5cm}

\begin{center}
    \begin{tikzpicture}
        % Array intero A
        \node at (-1, 1) {\Large $A$};
        \draw[thick] (0, 0.5) rectangle (8, 1.5);
        \node at (4, 1) {Intero Array};

        % Freccia divisione
        \draw[->, thick] (4, 0.4) -- (4, -0.4);

        % Array partizionato
        \node at (-1, -1.5) {\Large $A'$};

        % A1
        \draw[thick, fill=blue!10] (0, -2) rectangle (3, -1);
        \node at (1.5, -1.5) {$A_1 (\le P)$};

        % Pivot
        \draw[thick, fill=red!20] (3, -2) rectangle (4, -1);
        \node at (3.5, -1.5) {$P$};

        % A2
        \draw[thick, fill=green!10] (4, -2) rectangle (8, -1);
        \node at (6, -1.5) {$A_2 (> P)$};

        % Graffe descrizione
        \draw[decorate, decoration={brace, amplitude=5pt, mirror}] (0,-2.2) -- (3,-2.2) node[midway, below=0.2cm] {$\forall x \in A_1, x \le P$};
        \draw[decorate, decoration={brace, amplitude=5pt, mirror}] (4,-2.2) -- (8,-2.2) node[midway, below=0.2cm] {$\forall x \in A_2, x > P$};

    \end{tikzpicture}
\end{center}

\newpage

\subsection*{2. Esempio di Traccia (Partition)}
Esecuzione della partizione sull'array: \texttt{2 8 7 1 3 5 6 4}.
Il Pivot è l'ultimo elemento ($4$).

\vspace{0.5cm}

\begin{center}
    \begin{tikzpicture}[
        cell/.style={rectangle, draw, minimum size=0.8cm, font=\large},
        pivot/.style={rectangle, draw, minimum size=0.8cm, fill=gray!30, font=\large, thick},
        scan/.style={rectangle, draw, minimum size=0.8cm, fill=yellow!20, font=\large},
        swap/.style={rectangle, draw, minimum size=0.8cm, fill=red!20, font=\large}
        ]

        % Funzione per disegnare un passo
        \newcommand{\drawarray}[4]{
        \node[anchor=west] at (-2, 0) {\textbf{#1}};
        \foreach \val [count=\i] in {#2} {
        \pgfmathsetmacro{\x}{(\i-1)*0.8}

        % Stile di default
        \node[cell] (n\i) at (\x, 0) {\val};

        % Evidenzia Pivot (ultima posizione visualizzata come tale)
        \ifnum\i=#3
        \node[pivot] at (\x, 0) {\val};
        \fi
        }
        % Etichette i e j se fornite
        #4
        }

        % Passo 1: Inizio
        \begin{scope}[yshift=0cm]
            \drawarray{Inizio:}{2, 8, 7, 1, 3, 5, 6, 4}{8}{
            \node[below=0.1cm of n8] {\small Pivot};
            \node[above=0.1cm of n1] {$j$};
            \node[above=0.1cm of n1, xshift=-0.8cm] {$i$};
            }
        \end{scope}

        % Passo 2: Trovato 2 <= 4
        \begin{scope}[yshift=-2cm]
            \drawarray{Step 1:}{2, 8, 7, 1, 3, 5, 6, 4}{8}{
            \node[below=0.1cm of n1] {\small $i \gets i+1$};
            \node[draw=none] at (8, 0) {\small (2 $\le$ 4: Swap non nec.)};
            }
        \end{scope}

        % Passo 3: Scansione 8, 7 (maggiori, i fermo)
        \begin{scope}[yshift=-4cm]
            \drawarray{Step 2-3:}{2, 8, 7, 1, 3, 5, 6, 4}{8}{
            \node[above=0.1cm of n1] {$i$};
            \node[above=0.1cm of n3] {$j$};
            \node[draw=none] at (8, 0) {\small (8, 7 $>$ 4: $i$ fermo)};
            }
        \end{scope}

        % Passo 4: Trovato 1 <= 4 -> Swap con A[i+1] (8)
        \begin{scope}[yshift=-6cm]
            \drawarray{Step 4:}{2, 1, 7, 8, 3, 5, 6, 4}{8}{
            \node[swap] at (0.8, 0) {1}; % New pos of 1
            \node[swap] at (2.4, 0) {8}; % New pos of 8
            \node[above=0.1cm of n2] {$i$};
            \node[above=0.1cm of n4] {$j$};
            \draw[<->, thick, red, bend left] (0.8, 0.5) to (2.4, 0.5);
            \node[draw=none] at (8, 0) {\small (1 $\le$ 4: Swap $A[i+1] \leftrightarrow A[j]$)};
            }
        \end{scope}

        % Passo 5: Trovato 3 <= 4 -> Swap con A[i+1] (7)
        \begin{scope}[yshift=-8cm]
            \drawarray{Step 5:}{2, 1, 3, 8, 7, 5, 6, 4}{8}{
            \node[swap] at (1.6, 0) {3};
            \node[swap] at (3.2, 0) {7};
            \node[above=0.1cm of n3] {$i$};
            \node[above=0.1cm of n5] {$j$};
            \draw[<->, thick, red, bend left] (1.6, 0.5) to (3.2, 0.5);
            }
        \end{scope}

        % Passo Finale: Posiziona Pivot
        \begin{scope}[yshift=-10cm]
            \drawarray{Finale:}{2, 1, 3, 4, 7, 5, 6, 8}{4}{
            \node[pivot] at (2.4, 0) {4}; % Pivot in posizione corretta
            \node[above=0.1cm of n4] {Pivot};
            \draw[<->, thick, blue, bend left] (2.4, 0.5) to (5.6, 0.5);
            \node[draw=none] at (8, 0) {\small (Swap $A[i+1] \leftrightarrow A[r]$)};
            }
        \end{scope}

    \end{tikzpicture}
\end{center}

\newpage

\subsection*{3. Invariante della Procedura Partition}
Durante la scansione, l'array è diviso in 4 regioni dinamiche gestite dagli indici $i$ e $j$.

\vspace{1cm}

\begin{center}
    \begin{tikzpicture}[scale=1.0]
        % Definizioni dimensioni
        \def\h{1} % Altezza celle

        % Regioni
        % Regione <= x
        \draw[fill=green!10] (0,0) rectangle (3,\h);
        \node at (1.5, 0.5) {$\le x$};

        % Regione > x
        \draw[fill=red!10] (3,0) rectangle (6,\h);
        \node at (4.5, 0.5) {$> x$};

        % Regione da esaminare
        \draw[fill=gray!10] (6,0) rectangle (9,\h);
        \node at (7.5, 0.5) {? (ignoto)};

        % Pivot
        \draw[thick] (9,0) rectangle (10,\h);
        \node at (9.5, 0.5) {$x$};

        % Indici sotto
        \draw[thick] (0,0) -- (0,-0.2) node[below] {$p$};
        \draw[thick] (3,0) -- (3,-0.2) node[below] {$i$};
        \draw[thick] (6,0) -- (6,-0.2) node[below] {$j$};
        \draw[thick] (10,0) -- (10,-0.2) node[below] {$r$};

        % Descrizione sopra
        \node[above, font=\small, text=green!50!black] at (1.5, \h) {Già processati (minori)};
        \node[above, font=\small, text=red!50!black] at (4.5, \h) {Già processati (maggiori)};
        \node[above, font=\small, text=gray] at (7.5, \h) {Da processare};

        % Loop
        \draw[->, thick, blue] (6.5, -0.8) -- (7.5, -0.8) node[midway, below] {$j$ avanza};

    \end{tikzpicture}
\end{center}

\newpage


\newpage 

\part*{Lezione (27/11/2025)}

\section{Confronto Ordinamenti}
\subsection*{Riepilogo Complessità}

\begin{center}
    \textbf{Ordine di grandezza del:}

    \vspace{0.2cm}
    $\swarrow \quad \searrow$
    \vspace{0.2cm}

    \begin{tabular}{|l|c|c|c|c|l|}
        \hline
        \multirow{3}{*}{\textbf{Algoritmo}} & \multicolumn{3}{c|}{\textbf{COSTO IN TEMPO}} & \multirow{2}{*}{\textbf{COSTO}} & \multirow{3}{*}{\textbf{Commenti}} \\
        & \multicolumn{3}{c|}{} & \multirow{2}{*}{\textbf{IN SPAZIO}} & \\ \cline{2-4}
        & \shortstack{CASO \\ OTTIMO} & \shortstack{CASO \\ MEDIO} & \shortstack{CASO \\ PESSIMO} & & \\
        \hline
        Insertion Sort & $n$ & $n^2$ & $n^2$ & in loco & \\
        \hline
        Selection Sort & $n^2$ & $n^2$ & $n^2$ & in loco & \\
        \hline
        Merge Sort & $n \log n$ & $n \log n$ & $n \log n$ & $n$ & OTTIMO in tempo \\
        \hline
        Quick Sort & $n \log n$ & $n \log n$ & $n^2$ & \shortstack{in loco \\ + gestione \\ RICORSIONE} & \shortstack{OTTIMO in tempo \\ al caso medio} \\
        \hline
        \textbf{Heapsort} & $n \log n$ & $n \log n$ & $n \log n$ & in loco & \shortstack{OTTIMO in tempo \\ e spazio} \\
        \hline
    \end{tabular}
\end{center}

\vspace{0.3cm}
\noindent
$\llcorner \!\! \rightarrow$ \textbf{Heapsort} utilizza una struttura dati specifica: lo \textbf{HEAP}.

\hrulefill

\section{Struttura Dati: HEAP (di Massimo)}

\subsection{Definizione}
L'Heap rappresenta un \textbf{albero binario quasi completo}.
\begin{itemize}
    \item \textbf{Quasi completo} significa che l'albero è riempito completamente in tutti i livelli, tranne eventualmente l'ultimo.
    \item Sull'ultimo livello, i nodi sono tutti \textbf{accumulati a sinistra}.
\end{itemize}

\subsection{Rappresentazione in Array}
Anche se concettualmente è un albero, l'Heap viene solitamente rappresentato utilizzando un \textbf{array}.
\begin{itemize}
    \item Non serve memorizzare puntatori espliciti a padre e figli.
    \item La mappatura dall'albero all'array avviene tramite una \textbf{visita per livelli} (breadth-first).
    \item La radice si trova in $A[0]$.
    \item Gli elementi successivi seguono l'ordine della visita.
\end{itemize}

\noindent
Sia $A.heap\_size$ il numero di elementi dell'heap memorizzati in $A$. Gli elementi validi dell'heap si trovano negli indici da $0$ a $A.heap\_size - 1$.

\subsection{Regole di Posizionamento (Indici)}
Dato un nodo $x$ che corrisponde all'elemento di indice $i$ nell'array $A$:

\begin{center}
    \begin{tikzpicture}[
        node distance=1cm,
        level distance=1.2cm,
        sibling distance=2cm,
        every node/.style={circle, draw, minimum size=0.6cm, inner sep=0pt},
        arr_cell/.style={rectangle, draw, minimum width=1cm, minimum height=0.6cm, inner sep=0pt}
        ]

        % Disegno Albero
        \node[dashed] (t) {$t$}
        child {
        node (x) {$x$} edge from parent[dashed]
        child { node (y) {$y$} edge from parent[solid]}
        child { node (z) {$z$} edge from parent[solid]}
        };

        % Etichetta nodo x
        \node[draw=none, right=0.1cm of x] {\footnotesize \textit{indice $i$}};

        % Disegno Array sotto
        \node[arr_cell, below=2cm of t, xshift=-2cm] (arr_t) {$t$};
        \node[arr_cell, right=0cm of arr_t] (arr_x) {$x$};
        \node[arr_cell, right=0cm of arr_x, minimum width=1.5cm] (dots) {$\dots$};
        \node[arr_cell, right=0cm of dots] (arr_y) {$y$};
        \node[arr_cell, right=0cm of arr_y] (arr_z) {$z$};
        \node[arr_cell, right=0cm of arr_z, minimum width=1cm] (end) {};

        % Etichette indici array
        \node[below=0.1cm of arr_t, draw=none] {\footnotesize $\lfloor \frac{i-1}{2} \rfloor$};
        \node[below=0.1cm of arr_x, draw=none] {\footnotesize $i$};
        \node[below=0.1cm of arr_y, draw=none] {\footnotesize $2i+1$};
        \node[below=0.1cm of arr_z, draw=none] {\footnotesize $2i+2$};

        % Frecce o linee di collegamento logico (opzionale)
        \draw[->, gray, dotted] (x) -- (arr_x);
    \end{tikzpicture}
\end{center}

\vspace{0.5cm}

\noindent
Le formule per navigare l'albero muovendosi tra gli indici dell'array sono:

\begin{align*}
    \textbf{Parent}(i) & \quad \longrightarrow \quad \text{return } \left\lfloor \frac{i-1}{2} \right\rfloor \\
    \textbf{Left}(i)   & \quad \longrightarrow \quad \text{return } 2i + 1 \\
    \textbf{Right}(i)  & \quad \longrightarrow \quad \text{return } 2i + 2
\end{align*}

\section{Proprietà degli Heap: Verifica ed Efficienza}

\subsection{Correttezza delle Regole di Posizionamento}
Possiamo verificare che le formule per navigare l'array siano coerenti. In particolare, applicando la funzione \texttt{Parent} al risultato di \texttt{Left} o \texttt{Right}, dobbiamo tornare al nodo di partenza $i$.

\begin{proof}[Verifica]
    Ricordiamo che $\text{Parent}(k) = \lfloor \frac{k-1}{2} \rfloor$.
    \begin{itemize}
        \item \textbf{Figlio Sinistro:} $k = 2i + 1$.
        \[ \text{Parent}(\text{Left}(i)) = \left\lfloor \frac{(2i + 1) - 1}{2} \right\rfloor = \left\lfloor \frac{2i}{2} \right\rfloor = i \]
        \item \textbf{Figlio Destro:} $k = 2i + 2$.
        \[ \text{Parent}(\text{Right}(i)) = \left\lfloor \frac{(2i + 2) - 1}{2} \right\rfloor = \left\lfloor \frac{2i + 1}{2} \right\rfloor = \left\lfloor i + 0.5 \right\rfloor = i \]
    \end{itemize}
    Le regole sono corrette: si "risale" esattamente al genitore.
\end{proof}

\subsection{Efficienza in Memoria (Spazio)}
Memorizzare l'heap in un vettore di dimensione $n$ è molto più efficiente rispetto a una rappresentazione esplicita con nodi e puntatori.
\begin{itemize}
    \item \textbf{Rappresentazione Array:} Richiede spazio $n$ (solo i dati).
    \item \textbf{Rappresentazione a Puntatori:} Richiederebbe spazio $3n$ (per ogni nodo: il dato + puntatore left + puntatore right + eventuale puntatore parent).
\end{itemize}

\section{Proprietà fondamentali}

\subsection{Proprietà di Max-Heap}
Un \textbf{Max-Heap} è un albero binario quasi completo che soddisfa la seguente invariante:

\begin{definition}[Max-Heap Property]
    Per ogni nodo $i$ diverso dalla radice ($i > 0$):
    \[ A[\text{Parent}(i)] \ge A[i] \]
    Ovvero: il valore di un nodo è sempre minore o uguale al valore del padre.
\end{definition}

\begin{observation}
    Da questa proprietà deriva che:
    \begin{enumerate}
        \item L'elemento massimo dell'intero heap si trova sempre nella radice ($A[0]$).
        \item In ogni sotto-albero, la radice del sotto-albero contiene il valore massimo tra tutti i nodi di quel sotto-albero.
    \end{enumerate}
\end{observation}

\subsection{Altezza e Profondità}
È fondamentale distinguere tra altezza e profondità dei nodi:
\begin{itemize}
    \item \textbf{Profondità (Depth) di un nodo:} La lunghezza del cammino (numero di archi) dalla radice al nodo. (La radice ha profondità 0).
    \item \textbf{Altezza (Height) di un nodo:} La lunghezza del cammino più lungo dal nodo a una foglia. (Le foglie hanno altezza 0).
    \item \textbf{Altezza dell'Heap ($h$):} Corrisponde all'altezza della radice, ovvero la massima distanza dalla radice a una foglia.
\end{itemize}

\part*{Proprietà Matematiche degli Heap}

Analizziamo le tre proprietà fondamentali che legano la dimensione dell'input $n$ alla struttura dell'albero.

\section*{1. Altezza dell'Heap}

\begin{property}
    Un heap di $n$ elementi ha altezza $h = \lfloor \log_2 n \rfloor$.
    In notazione asintotica:
    \[ h = O(\log n) \]
\end{property}

\begin{proof}[Dimostrazione]
    Consideriamo i limiti sul numero di nodi $n$ per un albero binario di altezza $h$:
    \begin{itemize}
        \item \textbf{Caso minimo:} L'albero è completo fino al livello $h-1$ e ha una sola foglia al livello $h$.
        \[ n \ge 2^h \]
        \item \textbf{Caso massimo:} L'albero è pieno (tutti i livelli completi fino ad $h$).
        \[ n \le 2^{h+1} - 1 < 2^{h+1} \]
    \end{itemize}
    Combinando le disuguaglianze otteniamo:
    \[ 2^h \le n < 2^{h+1} \]
    Applicando il logaritmo in base 2:
    \[ h \le \log_2 n < h+1 \]
    Poiché $h$ deve essere un intero, l'unica soluzione è:
    \[ h = \lfloor \log_2 n \rfloor \]
\end{proof}

\section*{2. Numero di Foglie}

\begin{property}
    Un heap di $n$ nodi contiene esattamente $\lceil n/2 \rceil$ foglie.
\end{property}

\begin{proof}[Dimostrazione]
    Possiamo derivare il numero di foglie sottraendo il numero di \textbf{nodi interni} dal totale $n$.
    Un nodo $i$ è un nodo interno se ha almeno un figlio (il sinistro). La condizione di esistenza del figlio sinistro è:
    \[ \text{Left}(i) < n \implies 2i + 1 \le n - 1 \]
    Risolvendo per $i$:
    \[ 2i \le n - 2 \implies i \le \frac{n}{2} - 1 \]
    Essendo $i$ un intero:
    \[ i \le \left\lfloor \frac{n}{2} \right\rfloor - 1 \]
    I nodi interni sono quindi quelli con indice da $0$ a $\lfloor n/2 \rfloor - 1$. Il loro numero è $\lfloor n/2 \rfloor$.
    Il numero di foglie è:
    \[ \#\text{foglie} = n - \#\text{interni} = n - \left\lfloor \frac{n}{2} \right\rfloor = \left\lceil \frac{n}{2} \right\rceil \]
\end{proof}

\section*{3. Nodi di Altezza $h$}

\begin{property}
    In un heap di $n$ nodi, ci sono al più:
    \[ \left\lceil \frac{n}{2^{h+1}} \right\rceil \]
    nodi di altezza $h$.
\end{property}

\begin{intuition}[Caso dell'Albero Pieno - ABCB]
    Consideriamo un \textbf{ABCB} (Albero Binario Completamente Bilanciato), ovvero un heap "pieno" su tutti i livelli.
    Sia $H$ l'altezza totale dell'albero. Il numero totale di nodi è:
    \[ n = 2^{H+1} - 1 \]
    Analizziamo il numero di nodi per ogni altezza $h$:
    \begin{itemize}
        \item \textbf{Altezza $h=0$ (Foglie):} Circa metà dei nodi sono foglie ($n/2$).
        \item \textbf{Altezza $h=1$ (Padri delle foglie):} Sopra le foglie c'è un livello con la metà dei nodi rispetto al livello 0 ($n/4$).
        \item \textbf{Altezza generica $h$:} Generalizzando, il numero di nodi decresce esponenzialmente con l'altezza: $\approx n/2^{h+1}$.
    \end{itemize}
\end{intuition}

\part*{Manutenzione dell'Heap}

\section{Procedura MAX-Heapify}

\subsection{Definizione e Scopo}
\begin{definition}[MAX-Heapify]
    La procedura \texttt{MAX-Heapify} è un algoritmo fondamentale utilizzato per ripristinare la proprietà di Max-Heap in un nodo specifico che potrebbe violarla.
\end{definition}

\noindent
\textbf{Precondizioni (Ipotesi):}
Affinché la procedura funzioni correttamente, assumiamo che gli alberi binari radicati in $\text{Left}(i)$ e $\text{Right}(i)$ siano già dei \textbf{Max-Heap}, mentre $A[i]$ potrebbe essere minore dei suoi figli.

\subsection{Esempio Grafico}
L'indice $i=1$ (valore $4$) viola la proprietà perché è minore del figlio sinistro ($14$).

\vspace{0.5cm}

\begin{center}
    \begin{tikzpicture}[
        level distance=1.5cm,
        sibling distance=2cm,
        every node/.style={circle, draw, minimum size=0.8cm},
        highlight/.style={fill=red!20, draw=red, thick},
        arrow_line/.style={->, thick, red, bend right=45}
        ]
        % --- DISEGNO DELL'ALBERO ---
        \node (n0) {16}
        child { node[highlight] (n1) {4} % Nodo che viola
        child { node[highlight] (n3) {14} % Figlio maggiore
        child { node (n7) {2} }
        child { node (n8) {8} }
        }
        child { node (n4) {7}
        child { node (n9) {1} }
        child[missing]
        }
        }
        child { node (n2) {10}
        child { node (n5) {9} }
        child { node (n6) {3} }
        };
        % Frecce e label
        \node[draw=none, right=0.1cm of n1, red] {\footnotesize $i=1$};
        \node[draw=none, left=0.1cm of n3, red] {\footnotesize max};
        \draw[<->, dashed, thick, red] (n1) -- (n3) node[midway, right] {\footnotesize swap};

        % --- DISEGNO DELL'ARRAY ---
        \begin{scope}[yshift=-6cm, xshift=-4cm]
            \node[draw=none, rectangle] at (0, 0.8) {\textbf{Rappresentazione Array:}};
            \foreach \x/\val in {0/16, 1/4, 2/10, 3/14, 4/7, 5/9, 6/3, 7/2, 8/8, 9/1} {
            \node[draw, rectangle, minimum size=0.8cm] (arr\x) at (\x, 0) {\val};
            \node[draw=none, below=0.1cm of arr\x] {\footnotesize \x};
            }
            \node[draw=red, thick, rectangle, minimum size=0.8cm] at (1, 0) {4};
            \node[draw=red, thick, rectangle, minimum size=0.8cm] at (3, 0) {14};
            \draw[->, thick, red] (arr1.south) to[out=-45, in=-135] (arr3.south);
        \end{scope}
    \end{tikzpicture}
\end{center}

\subsection{Pseudocodice}

\begin{algorithm}
    \caption{MAX-Heapify(A, i)}
    \begin{algorithmic}[1]
        \State $l \gets \text{Left}(i)$
        \State $r \gets \text{Right}(i)$
        \State $max \gets i$
        \State \Comment{Controlla se il figlio sinistro esiste ed è maggiore del corrente massimo}
        \If{$l < A.heap\_size \textbf{ and } A[l] > A[max]$}
            \State $max \gets l$
        \EndIf
        \State \Comment{Controlla se il figlio destro esiste ed è maggiore del corrente massimo}
        \If{$r < A.heap\_size \textbf{ and } A[r] > A[max]$}
            \State $max \gets r$
        \EndIf
        \State \Comment{Se il massimo non è la radice $i$, scambia e ricorri}
        \If{$max \neq i$}
            \State \textbf{swap} $A[i] \leftrightarrow A[max]$
            \State \textsc{Max-Heapify}$(A, max)$
        \EndIf
    \end{algorithmic}
\end{algorithm}

\subsection{Analisi della Complessità}

\begin{analysis}[Costo Temporale]
    Il costo è proporzionale all'altezza del nodo $i$, poiché nel caso peggiore il valore scende fino alle foglie.
    \[ T(n) = O(h) = O(\log n) \]
\end{analysis}

\hrulefill

\section{Costruzione dell'Heap (Build-Max-Heap)}

\subsection{Strategia Bottom-Up}
La procedura trasforma un array disordinato in un Max-Heap chiamando \texttt{Max-Heapify} a ritroso, dai nodi interni fino alla radice.
Le foglie (da $\lfloor n/2 \rfloor$ a $n-1$) sono già heap validi.

\begin{center}
    \begin{tikzpicture}[scale=1]
        \draw[thick] (0,0) rectangle (10, 1);
        \draw (1,0) -- (1,1); \node at (0.5, 0.5) {$A[0]$}; \node at (1.5, 0.5) {$\dots$}; \draw (2,0) -- (2,1);
        \draw[thick] (5,0) -- (5,1);
        \draw (9,0) -- (9,1); \node at (9.5, 0.5) {$A[n\!-\!1]$};
        \node[below=0.2cm] at (0.5,0) {$0$};
        \node[below=0.2cm] at (5,0) {$\lfloor \frac{n}{2} \rfloor$};
        \node[below=0.2cm] at (9.5,0) {$n-1$};
        \draw[decorate, decoration={brace, amplitude=10pt, mirror}, thick] (0,-0.8) -- (5,-0.8) node[midway, below=15pt] {\textbf{Nodi Interni} (da processare)};
        \draw[decorate, decoration={brace, amplitude=10pt, mirror}, thick] (5.1,-0.8) -- (10,-0.8) node[midway, below=15pt] {\textbf{Foglie} (già Heap)};
    \end{tikzpicture}
\end{center}

\subsection{Pseudocodice}

\begin{algorithm}
    \caption{Build-Max-Heap(A)}
    \begin{algorithmic}[1]
        \State $A.heap\_size \gets A.length$
        \For{$i \gets \lfloor \frac{A.length}{2} \rfloor - 1 \textbf{ downto } 0$}
            \State \textsc{Max-Heapify}$(A, i)$
        \EndFor
    \end{algorithmic}
\end{algorithm}

\section{Analisi della Complessità di Build-Max-Heap}

\subsection{Analisi Accurata}
Il costo totale non è $O(n \log n)$, ma \textbf{lineare} $O(n)$.
Il costo totale $T(n)$ è la somma dei costi per ogni nodo, che dipendono dall'altezza $h$.
\[ T(n) = \sum_{h=0}^{\lfloor \log n \rfloor} (\text{nodi di altezza } h) \times O(h) \]
Sostituendo il numero massimo di nodi $\lceil n/2^{h+1} \rceil$:
\[ T(n) \le \frac{c \cdot n}{2} \sum_{h=0}^{\lfloor \log n \rfloor} \frac{h}{2^h} \]
La serie $\sum \frac{h}{2^h}$ converge a 2. Pertanto:
\[ T(n) \le \frac{c \cdot n}{2} \cdot 2 = O(n) \]


\section*{Correttezza}

\subsection{Invariante di ciclo}

All'inizio dell'iterazione del ciclo for...

\section{Analisi del Costo in Tempo}

Limite superiore:
$n/2$ chiamate di \textsc{Max-Heapify}...

\newpage






















\newpage % ===================================================================
% FILE: Lezione15.tex
% ===================================================================

\part{Lezione 29/1 - 3/2: Pile e Code}

\section{Pile e Code: Insiemi Dinamici}
Sono insiemi dinamici in cui l'elemento rimosso dall'operazione di cancellazione, o inserito dall'operazione di inserimento, è \textbf{PREDETERMINATO}.

\begin{definition}[Organizzazione Logica]
    \begin{itemize}
        \item \textbf{PILA (Stack)}: Politica \textbf{LIFO} (Last In First Out).
        \item \textbf{CODA (Queue)}: Politica \textbf{FIFO} (First In First Out).
    \end{itemize}
    \textbf{Implementazione}: ARRAY o LISTE.
\end{definition}

\section{Pile (Stacks)}

Le operazioni possibili (Query e Modifica) sono:
\begin{itemize}
    \item \texttt{ISEMPTY}: dice se la pila è vuota.
    \item \texttt{TOP}: lettura dell'elemento in cima alla pila (immutata).
    \item \texttt{PUSH}: inserimento (in cima).
    \item \texttt{POP}: cancellazione (dalla cima).
\end{itemize}

\subsection{Implementazione su Array}


\begin{algorithmic}[1]
    \Procedure{IsEmpty}{PILA, top}
        \If{$top < 1$} \Return \textbf{TRUE}
        \Else \Return \textbf{FALSE}
        \EndIf
        \Comment{Complessità Costante $\Theta(1)$}
    \EndProcedure
    \Statex
    \Procedure{Top}{PILA, top}
        \If{\Call{IsEmpty}{PILA, top}} \Return \textbf{error}
        \Else \Return $PILA[top]$
        \EndIf
    \EndProcedure
    \Statex
    \Procedure{Push}{PILA, top, x}
        \State $top \gets top + 1$
        \If{$top > PILA.length$} \Return \textbf{error}
        \Else \State $PILA[top] \gets x$
        \EndIf
    \EndProcedure
    \Statex
    \Procedure{Pop}{PILA, top}
        \If{\Call{IsEmpty}{PILA, top}} \Return \textbf{error}
        \Else
            \State $x \gets PILA[top]$
            \State $top \gets top - 1$
            \Return $x$
        \EndIf
        \Comment{Complessità Costante $\Theta(1)$}
    \EndProcedure
\end{algorithmic}

\begin{explanation}{Gestione Stack (Array)}
Tutte le operazioni lavorano sull'indice \texttt{top} in tempo costante $\Theta(1)$.
\begin{itemize}
    \item \textbf{Push}: Prima incrementa, poi scrive (Prefix).
    \item \textbf{Pop}: Legge, poi decrementa (Postfix logico).
\end{itemize}
\end{explanation}


\begin{center}
    \begin{tikzpicture}[
        element/.style={draw=paletteIndigo, fill=white, minimum width=1.5cm, minimum height=0.8cm, font=\small},
        stack/.style={draw=paletteIndigo, thick, minimum width=1.7cm, minimum height=4cm}
    ]
        % Stack Container
        \draw[thick, draw=paletteIndigo] (0,0) -- (0,4) (2,4) -- (2,0) -- (0,0);
        
        % Elements
        \node[element] at (1, 0.4) {Elemento 1};
        \node[element] at (1, 1.2) {Elemento 2};
        \node[element] at (1, 2.0) {Elemento 3};
        
        % Top Pointer
        \draw[->, thick, paletteRegalia] (3, 2.0) -- (2.1, 2.0) node[right, at start] {\texttt{top}};
        
        % Labels
        \node[below, paletteWenge] at (1,-0.2) {Stack (LIFO)};
        \node[above, paletteRegalia] at (1, 4.2) {PUSH $\downarrow$ / POP $\uparrow$};
    \end{tikzpicture}
\end{center}

\subsection{Implementazione su Lista}

\begin{algorithmic}[1]
    \Procedure{IsEmpty}{topEl}
        \If{$topEl == \textbf{nil}$} \Return \textbf{TRUE}
        \Else \Return \textbf{FALSE}
        \EndIf
    \EndProcedure
    \Statex
    \Procedure{Top}{topEl}
        \If{\Call{IsEmpty}{topEl}} \Return \textbf{error}
        \Else \Return $topEl.key$
        \EndIf
    \EndProcedure
    \Statex
    \Procedure{Push}{topEl, x}
        \State $x.next \gets topEl$
        \State $topEl \gets x$
    \EndProcedure
    \Statex
    \Procedure{Pop}{topEl}
        \If{\Call{IsEmpty}{topEl}} \Return \textbf{error}
        \EndIf
        \State $VAL \gets topEl.key$
        \State $topEl \gets topEl.next$
        \Return $VAL$
    \EndProcedure
\end{algorithmic}

\begin{explanation}{Gestione Stack (Lista)}
Le operazioni avvengono sempre sulla \textbf{testa} della lista (\texttt{topEl}):
\begin{itemize}
    \item \textbf{Push}: \texttt{x.next = topEl; topEl = x} (Inserimento in testa).
    \item \textbf{Pop}: \texttt{topEl = topEl.next} (Rimozione in testa).
\end{itemize}
\end{explanation}

\textbf{Complessità}: Sempre Costante.

\newpage
\section{Code (Queues)}

\subsection{Possibili Query e Operazioni}
\begin{itemize}
    \item \texttt{ISFULL} (ARRAY ONLY).
    \item \texttt{ISEMPTY}: dice se la coda è vuota.
    \item \texttt{FIRST}: lettura dell'elemento in testa alla coda (immutata).
    \item \texttt{ENQUEUE}: inserimento.
    \item \texttt{DEQUEUE}: cancellazione.
\end{itemize}

\subsection{Implementazione su Array (Gestione Circolare)}
\begin{itemize}
    \item \texttt{head}: indice dell'elemento in testa.
    \item \texttt{tail}: indice della locazione in cui inserire il prossimo elemento.
\end{itemize}

\begin{center}
    \begin{tikzpicture}[scale=0.8]
        % Circular Array
        \foreach \i in {1,...,8} {
            \draw[thick, draw=paletteIndigo, fill=white] ( {45*(\i-1)}:2cm ) arc ({45*(\i-1)}:{45*\i}:2cm) -- ({45*\i}:3cm) arc ({45*\i}:{45*(\i-1)}:3cm) -- cycle;
            \node at ({45*(\i-0.5)}:2.5cm) {$Q[\i]$};
        }
        
        % Pointers
        \draw[->, ultra thick, paletteRegalia] (0:3.5cm) -- (0:3.1cm) node[right, at start] {\texttt{head}};
        \draw[->, ultra thick, paletteLenurple] (135:3.5cm) -- (135:3.1cm) node[left, at start] {\texttt{tail}};
    \end{tikzpicture}
\end{center}


\begin{algorithmic}[1]
    \Function{IsEmpty}{A, head, tail}
        \Return $(head == tail)$
    \EndFunction
    \Statex
    \Function{IsFull}{A, head, tail}
        \Return $(head == tail + 1)$
        \Comment{N.B. array circolare}
    \EndFunction
    \Statex
    \Function{First}{A, head, tail}
        \If{\Call{IsEmpty}{A, head, tail}} \Return \textbf{error}
        \Else \Return $A[head]$
        \EndIf
    \EndFunction
    \Statex
    \Procedure{Enqueue}{A, head, tail, x}
        \If{\Call{IsFull}{A, head, tail}} \Return \textbf{error}
        \EndIf
        \State $A[tail] \gets x$
        \State $tail \gets (tail + 1) \% A.length$
        \Comment{$\Theta(1)$}
    \EndProcedure
    \Statex
    \Procedure{Dequeue}{A, head, tail}
        \If{\Call{IsEmpty}{A, head, tail}} \Return \textbf{error}
        \EndIf
        \State $x \gets A[head]$
        \State $head \gets (head + 1) \% A.length$
        \Return $x$
        \Comment{$\Theta(1)$}
    \EndProcedure
\end{algorithmic}

\begin{explanation}{Coda Circolare}
L'array "si morde la coda" grazie all'operatore modulo:
$$ next\_idx = (curr\_idx + 1) \% capacity $$
Questo evita di dover shiftare gli elementi dopo una Dequeue.
\end{explanation}


\subsection{Implementazione su Lista}


\begin{algorithmic}[1]
    \Function{IsEmpty}{head}
        \Return $(head == \textbf{nil})$
    \EndFunction
    \Statex
    \Function{First}{head}
        \If{\Call{IsEmpty}{head}} \Return \textbf{error}
        \Else \Return $head.key$
        \EndIf
    \EndFunction
    \Statex
    \Procedure{Enqueue}{head, tail, x}
        \If{\Call{IsEmpty}{head}}
            \State $head \gets x$
        \Else
            \State $tail.next \gets x$
        \EndIf
        \State $tail \gets x$
        \State $x.next \gets \textbf{nil}$
    \EndProcedure
    \Statex
    \Procedure{Dequeue}{head, tail}
        \If{\Call{IsEmpty}{head}} \Return \textbf{error}
        \EndIf
        \State $VAL \gets head.key$
        \If{$head == tail$}
            \State $tail \gets \textbf{nil}$
        \EndIf
        \State $head \gets head.next$
        \Return $VAL$
        \Comment{$\Theta(1)$}
    \EndProcedure
\end{algorithmic}

\begin{explanation}{Coda su Lista}
\begin{itemize}
    \item \textbf{Enqueue}: Avviene su \texttt{tail}. Necessario aggiornare il puntatore \texttt{next} della vecchia coda e spostare \texttt{tail}.
    \item \textbf{Dequeue}: Avviene su \texttt{head}. Caso speciale: se la coda diventa vuota, bisogna mettere a NULL anche \texttt{tail}.
\end{itemize}
\end{explanation}


\textbf{Entrambe Complessità Costante}.

\newpage % ===================================================================
% FILE: Lezione16.tex
% ===================================================================

\part{Lezione 1/12: Heapsort e Code di Priorità}

\section{Build-Max-Heap}


\begin{algorithmic}[1]
    \Procedure{Build-Max-Heap}{A, n}
        \State $A.hs \gets n$
        \For{$i = \lfloor n/2 \rfloor - 1 \textbf{ downto } 0$}
            \State \Call{Max-Heapify}{A, i}
        \EndFor
    \EndProcedure
\end{algorithmic}

\begin{explanation}{Logica di Costruzione (Bottom-Up)}
Partendo dall'ultimo nodo interno ($\lfloor n/2 \rfloor - 1$) fino alla radice, chiamiamo \texttt{Max-Heapify}.
Le foglie sono già heap validi, quindi non serve processarle.
\end{explanation}


\subsection{Analisi di Complessità}
\begin{itemize}
    \item \textbf{Limite Superiore}: $n/2$ chiamate di \texttt{Max-Heapify} (costo $O(\log n)$) $\rightarrow T(n) = O(n \log n)$.
    \item \textbf{Limite Stretto (Corretto)}: $T(n) = O(n)$.
\end{itemize}

\subsection{Correttezza}
\textbf{Invariante}: All'inizio di ogni iterazione del ciclo for, ogni nodo $i+1, i+2, \dots, n-1$ è radice di un max-heap.

\section{Heapsort}


\begin{algorithmic}[1]
    \Procedure{Heapsort}{A}
        \State \Call{Build-Max-Heap}{A}
        \For{$i = n-1 \textbf{ downto } 1$}
            \State scambia $A[0]$ con $A[i]$
            \State $A.hs \gets A.hs - 1$
            \State \Call{Max-Heapify}{A, 0}
        \EndFor
    \EndProcedure
\end{algorithmic}

\begin{explanation}{Fasi dell'Heapsort}
\begin{enumerate}
    \item \textbf{Costruzione}: Si trasforma l'array in un Max-Heap.
    \item \textbf{Estrazione}: Si scambia la radice (massimo) con l'ultimo elemento e si riduce la dimensione dell'heap.
    \item \textbf{Ripristino}: Si chiama \texttt{Max-Heapify} sulla nuova radice per far "affondare" il valore scambiato.
\end{enumerate}
\end{explanation}


\textbf{Costo Totale}: $T(n) = O(n \log n)$.

\subsection{Esempio Grafico (Heap)}
Esempio su Array: $A = [16, 14, 10, 8, 7, 9, 3]$.

\begin{center}
    \begin{tikzpicture}[
        treenode/.style={draw=paletteIndigo, circle, fill=white, minimum size=8mm, font=\small},
        level 1/.style={sibling distance=3cm},
        level 2/.style={sibling distance=1.5cm},
        edge from parent/.style={draw=paletteRegalia, thick, ->}
    ]
        \node[treenode] {16}
            child {node[treenode] {14}
                child {node[treenode] {8}}
                child {node[treenode] {7}}
            }
            child {node[treenode] {10}
                child {node[treenode] {9}}
                child {node[treenode] {3}}
            };
            
        \node[right, align=left, paletteWenge] at (4, -1) {\textbf{Max-Heap}:\\Ogni padre $\ge$ figli.\\Radice = Max Assoluto.};
    \end{tikzpicture}
\end{center}

\section{Code di Priorità}
Mantiene un insieme di elementi con chiavi (key, priorità).

\subsection{Operazioni}
\begin{itemize}
    \item \texttt{Insert(S, x)}: $S = S \cup \{x\}$.
    \item \texttt{Heap-Max(A)}: restituisce l'elemento massimo, $O(1)$.
    \item \texttt{Heap-Extract-Max(A)}: rimuove e restituisce il massimo, $O(\log n)$.
    \item \texttt{Heap-Increase-Key(A, i, k)}: aumenta il valore della chiave del nodo $i$ a $k$, $O(\log n)$.
    \item \texttt{Max-Heap-Insert(A, key)}: inserisce una nuova chiave, $O(\log n)$.
\end{itemize}

\newpage 
% ===================================================================
% FILE: Lezione17.tex
% ===================================================================

% --- Definizioni Locali (Compatibilità) ---

% (Rimosso codebox locale)

% Adattamento ambiente 'proposition' usando 'concept' (stile Bibbia)
\newenvironment{proposition}[1][Proposizione]
  {\begin{concept}{#1}}
  {\end{concept}}

\part{Lezione 17: Alberi Binari: Bilanciamento e Proprietà Avanzate}

\section{Schema Generale di Ricorsione su Alberi}

Quando si affrontano problemi su alberi binari che richiedono di calcolare una proprietà o un valore basato sulla struttura dell'albero, si utilizza spesso uno schema ricorsivo standard basato sulla proprietà di \textbf{decomponibilità} del problema.

\begin{concept}{Schema di Risoluzione Decomponibile}
L'algoritmo segue questi passi fondamentali:
\begin{enumerate}
    \item \textbf{Caso Base:} Si gestisce l'albero vuoto (o foglia), restituendo un valore neutro o specifico per la chiusura della ricorrenza (es. 0, null, true).
    \item \textbf{Passo Induttivo (Divide):} Si effettuano le chiamate ricorsive sui sottoalberi sinistro (\texttt{u.left}) e destro (\texttt{u.right}).
    \item \textbf{Combinazione (Conquer):} Si combinano i risultati ottenuti dai sottoalberi con le informazioni del nodo corrente (\texttt{u}) per ottenere il risultato finale.
    \item \textbf{Restituzione:} Si restituisce l'esito al chiamante (caso generale).
\end{enumerate}
\end{concept}

\subsection{Implementazione Generica e Complessità}

Di seguito lo pseudocodice generico per una funzione \texttt{DECOMP(u)} che opera su un nodo \texttt{u}.

\begin{codebox}{Pseudocodice: DECOMP(u)}
\begin{verbatim}
Funzione DECOMP(u)
    // 1. Caso Base
    IF (u == NULL) THEN 
        RETURN ValoreBase; 
    
    // 2. Passo Induttivo
    RisSx = DECOMP(u.left);  // Ricorsione a sinistra
    RisDx = DECOMP(u.right); // Ricorsione a destra
    
    // 3. Combinazione
    RETURN RICOMBINA(RisSx, RisDx, u.info);
\end{verbatim}
\end{codebox}

\begin{explanation}{Analisi dello Schema Ricorsivo}
Questo "template" è universale per problemi decomponibili:
\begin{itemize}
    \item \textbf{Discesa (Pre-order)}: Si scende fino alle foglie (null).
    \item \textbf{Risalita (Post-order)}: I risultati parziali (\texttt{RisSx}, \texttt{RisDx}) tornano dai figli e vengono combinati nel nodo corrente.
\end{itemize}
\end{explanation}

\begin{center}
\begin{tikzpicture}[
    node/.style={circle, draw, minimum size=0.8cm},
    arrow/.style={->, thick, blue}
    ]
    \node[node] (u) {u};
    \node[node, below left=of u] (l) {sx};
    \node[node, below right=of u] (r) {dx};
    
    \draw[->] (u) -- (l) node[midway, left, font=\tiny] {1. Call};
    \draw[->] (u) -- (r) node[midway, right, font=\tiny] {2. Call};
    \draw[dashed, ->, red] (l) to[bend left] node[midway, left, font=\tiny] {3. Return} (u);
    \draw[dashed, ->, red] (r) to[bend right] node[midway, right, font=\tiny] {4. Return} (u);
    
    \node[right=of u] {Combine (u.info + sx + dx)};
\end{tikzpicture}
\end{center}

\paragraph{Complessità Computazionale.}
Poiché l'algoritmo visita ogni nodo esattamente una volta (come una visita post-order), la complessità temporale è lineare rispetto al numero di nodi $n$:
$$ T(n) = \Theta(n) $$

---

\section{Alberi Binari Completamente Bilanciati (ABCB)}

Un concetto fondamentale è la distinzione tra un albero completo e uno completamente bilanciato.

\begin{concept}{Definizioni Tassonomiche}
\begin{itemize}
    \item \textbf{Albero Binario Completo:} Un albero binario è \textit{completo} se ogni nodo ha esattamente 0 o 2 figli (nessun nodo ha grado 1).
    \item \textbf{Albero Binario Completamente Bilanciato (ABCB):} È un albero binario che è \textbf{sia completo}, sia ha tutte le \textbf{foglie allo stesso livello}.
\end{itemize}
\end{concept}

\subsection{Proprietà Matematiche degli ABCB}
Dato un ABCB di altezza $h$ (dove l'altezza è il numero di archi dalla radice alla foglia più profonda, oppure definita in base ai nodi come $h_{nodi}$), valgono le seguenti relazioni notevoli:

\begin{itemize}
    \item \textbf{Numero di foglie:} $2^h$.
    \item \textbf{Numero di nodi interni:} $2^h - 1$.
    \item \textbf{Numero totale di nodi ($n$):} Somma di foglie e interni:
    $$ n = 2^h + (2^h - 1) = 2^{h+1} - 1 $$
    \item \textbf{Relazione Altezza-Nodi:} Invertendo la formula:
    $$ n + 1 = 2^{h+1} \implies \log_2(n+1) = h + 1 \implies h = \log_2(n+1) - 1 $$
    Pertanto, l'altezza è logaritmica: $h(n) = \Theta(\log n)$.
\end{itemize}

\subsection{Algoritmo di Verifica ABCB}
Vogliamo scrivere un algoritmo che restituisca \texttt{TRUE} se un albero è ABCB, \texttt{FALSE} altrimenti.

\subsubsection{Approccio 1: Naive (Inefficiente)}
Un primo approccio controlla ricorsivamente se i sottoalberi sono ABCB e se hanno la stessa altezza.
$$ \text{Check}(u) = \text{ABCB}(u.left) \land \text{ABCB}(u.right) \land (H(u.left) == H(u.right)) $$
Questo approccio è inefficiente perché ricalcola l'altezza $H(u)$ ripetutamente per ogni nodo.

\subsubsection{Approccio 2: Ottimizzato (Lineare)}
Per ottenere una complessità $\Theta(n)$, dobbiamo calcolare il bilanciamento e l'altezza in un'unica visita (bottom-up). La funzione restituisce una coppia di valori \texttt{<bool bilanciato, int altezza>}.

\begin{codebox}{Algoritmo: ABCB2(u) $\rightarrow <bool, int>$}
\begin{verbatim}
ABCB2(u):
    // Caso Base: albero vuoto è bilanciato, altezza -1
    IF (u == NULL) THEN 
        RETURN <TRUE, -1>;

    // Chiamate Ricorsive
    <bil_s, alt_s> = ABCB2(u.left);
    <bil_d, alt_d> = ABCB2(u.right);

    // Logica di Combinazione:
    // 1. Sottoalberi devono essere bilanciati (bil_s && bil_d)
    // 2. Altezze devono essere uguali (alt_s == alt_d)
    is_balanced = bil_s AND bil_d AND (alt_s == alt_d);
    
    // Calcolo nuova altezza corrente
    current_height = 1 + max(alt_s, alt_d);

    RETURN <is_balanced, current_height>;
\end{verbatim}
\end{codebox}

\begin{explanation}{Strategia Bottom-Up $O(n)$}
Invece di ricalcolare l'altezza separatamente ($O(n^2)$), la funzione restituisce \textbf{due valori}:
1.  Il booleano di bilanciamento (AND logico dei figli).
2.  L'altezza corrente (aggiornata in risalita).
Questo permette di visitare ogni nodo una sola volta.
\end{explanation}

\textbf{Complessità:} Lineare, $\Theta(n)$, poiché esegue un attraversamento post-order costante per nodo.

---

\section{Nodi Cardine}

Un problema algoritmico interessante riguarda l'identificazione di nodi che soddisfano una specifica proprietà geometrica all'interno dell'albero.

\begin{concept}{Definizione: Nodo Cardine}
Un nodo $u$ di un albero binario si dice \textbf{CARDINE} se e solo se la sua profondità ($P_u$) è uguale alla sua altezza ($h_u$).
$$ u \text{ è Cardine} \iff P_u = h_u $$
\end{concept}
\textit{Nota:} $P_u$ è la distanza dalla radice, $h_u$ è la distanza dalla foglia più profonda nel suo sottoalbero.

\subsection{Strategia Risolutiva}
Per verificare questa condizione in modo efficiente ($\Theta(n)$), dobbiamo:
\begin{enumerate}
    \item Passare la profondità $p$ \textbf{scendendo} nell'albero (parametro in input).
    \item Calcolare l'altezza \textbf{risalendo} dall'albero (valore di ritorno).
    \item Verificare la condizione $p == altezza$ nel post-order.
\end{enumerate}

\begin{codebox}{Algoritmo: CARDINE(u, p)}
\begin{verbatim}
// Input: u (nodo corrente), p (profondità corrente, inizialmente 0)
// Output: altezza del sottoalbero radicato in u

INT CARDINE(Node u, int p)
    // 1. Caso Base
    IF (u == NULL) THEN 
        RETURN -1;

    // 2. Discesa ricorsiva (aumento profondità)
    int alt_s = CARDINE(u.left, p + 1);
    int alt_d = CARDINE(u.right, p + 1);

    // 3. Calcolo Altezza (fase di risalita)
    int \texttt{my\_alt} = max(alt_s, alt_d) + 1;

    // 4. Verifica proprietà Cardine
    IF (p == \texttt{my\_alt}) THEN 
        PRINT(u.key);

    RETURN \texttt{my\_alt};
\end{verbatim}
\end{codebox}

\begin{explanation}{Doppio Flusso Informativo}
Qui combiniamo due direzioni:
\begin{itemize}
    \item \textbf{Input (Top-down)}: La profondità \texttt{p} viene passata dal padre al figlio incrementandola.
    \item \textbf{Output (Bottom-up)}: L'altezza \texttt{\texttt{my\_alt}} viene calcolata dalle foglie risalendo.
    \item La verifica \texttt{p == \texttt{my\_alt}} avviene nel momento di incontro (post-order).
\end{itemize}
\end{explanation}

\subsection{Esempio di Traccia (Trace)}
Consideriamo l'albero tracciato nei sorgenti.
\begin{center}
\begin{tikzpicture}[level distance=1.5cm,
  level 1/.style={sibling distance=3cm},
  level 2/.style={sibling distance=1.5cm}]
  \node[circle, draw] {7}
    child {node[circle, draw] {4}
      child {node[circle, draw] {2}}
      child {node[circle, draw] {6}}
    }
    child {node[circle, draw] {12}
      child {node[circle, draw] {10}}
      child {node[circle, draw] {14}}
    };
\end{tikzpicture}
\end{center}

Analisi per alcuni nodi (supponendo struttura bilanciata dall'esempio):
\begin{itemize}
    \item \textbf{Nodo 7 (Radice):} Profondità $P=0$, Altezza $h=2$. ($0 \neq 2$) $\rightarrow$ No.
    \item \textbf{Nodo 4:} Profondità $P=1$, Altezza $h=1$. ($1 = 1$) $\rightarrow$ \textbf{CARDINE}.
    \item \textbf{Nodo 12:} Profondità $P=1$, Altezza $h=1$. ($1 = 1$) $\rightarrow$ \textbf{CARDINE}.
    \item \textbf{Foglie (2, 6, 10, 14):} Profondità $P=2$, Altezza $h=0$. ($2 \neq 0$) $\rightarrow$ No.
\end{itemize}

---

\section{Esercizi per Casa}

\subsection{Nodi Centrali}
\textbf{Problema:} Progettare un algoritmo che, dato un Albero Binario, stampi le chiavi dei suoi \textbf{Nodi Centrali}.

\begin{concept}{Definizione: Nodo Centrale}
Un nodo $u$ è detto \textbf{CENTRALE} se la dimensione del sottoalbero di cui è radice (numero di nodi) è uguale alla somma delle chiavi dei nodi che appartengono al percorso dalla radice al nodo stesso.
$$ Size(u) = \sum_{v \in \text{Path}(root \to u)} v.key $$
\end{concept}

\paragraph{Suggerimento per la soluzione:}
L'algoritmo deve combinare due flussi di informazioni, similmente all'esercizio sui Nodi Cardine:
1. \textbf{Top-down (Parametro):} La somma delle chiavi dalla radice al padre + chiave corrente.
2. \textbf{Bottom-up (Return):} La dimensione del sottoalbero (1 + size(sx) + size(dx)).
3. \textbf{Verifica:} Nel passo di post-order, confrontare i due valori.

\begin{codebox}{Abbozzo Soluzione: CENTRALE(u, somma\_cammino)}
\begin{verbatim}
CENTRALE(u, somma_corrente):
    IF u == NULL RETURN 0
    
    nuova_somma = somma_corrente + u.key
    
    size_sx = CENTRALE(u.left, nuova_somma)
    size_dx = CENTRALE(u.right, nuova_somma)
    
    my_size = 1 + size_sx + size_dx
    
    IF (my_size == nuova_somma) PRINT u.key
    
    RETURN my_size
\end{verbatim}
\end{codebox}

\begin{explanation}{Logica Nodi Centrali}
Simile ai nodi cardine, ma accumulando la somma delle chiavi:
\begin{itemize}
    \item \texttt{somma\_corrente} scende (Pre-order).
    \item \texttt{my\_size} (dimensione sottoalbero) risale (Post-order).
\end{itemize}
\end{explanation}

\newpage 
% ===================================================================
% FILE: Lezione18.tex
% ===================================================================

% --- Definizioni Locali (Compatibilità) ---
% (Rimosso codebox locale poiché definito in main.tex)

\part{Lezione 18: Strutture Dati Lineari}

\section{Liste (Linked Lists)}

Le liste rappresentano una struttura dati fondamentale per la gestione di collezioni dinamiche di elementi.

\begin{concept}{Definizione e Struttura}
Una lista è una sequenza di nodi dove ogni elemento è \textbf{virtualmente concatenato} ma non necessariamente contiguo in memoria. Non è necessario conoscere la dimensione a priori (numero di elementi necessari).

La struttura di base di un nodo comprende:
\begin{itemize}
    \item Un campo dati (informazione).
    \item Un campo \texttt{NEXT} che punta all'elemento successivo.
\end{itemize}
La lista è accessibile tramite un puntatore alla testa (\texttt{HEAD}).
\end{concept}

\subsection{Tipologie e Vantaggi}
\begin{itemize}
    \item \textbf{Liste Semplici:} Collegamento unidirezionale (\texttt{NEXT}).
    \item \textbf{Liste Doppie:} Ogni nodo possiede due puntatori: uno al successivo (\texttt{NEXT}) e uno al precedente (\texttt{PREC}).
\end{itemize}

\textbf{Vantaggi rispetto agli Array:}
Le liste sono molto comode per effettuare inserimenti ed eliminazioni in posizioni intermedie, operazioni che in un array risulterebbero scomode (richiedendo lo shift degli elementi).

\begin{center}
\begin{tikzpicture}[list/.style={rectangle split, rectangle split parts=2,
    draw, rectangle split horizontal}, >=stealth, start chain]

  \node[list, on chain] (A) {Dati \nodepart{second} Next};
  \node[list, on chain] (B) {Dati \nodepart{second} Next};
  \node[list, on chain] (C) {Dati \nodepart{second} $\emptyset$};

  \draw[->] let \p1 = (A.two), \p2 = (A.center) in (\x1,\y2) -- (B);
  \draw[->] let \p1 = (B.two), \p2 = (B.center) in (\x1,\y2) -- (C);
  
  \node[left=of A, anchor=east] {HEAD};
  \draw[->] ($(A.west)+(-0.5,0)$) -- (A.west);
\end{tikzpicture}
\end{center}

---

\section{Pile (Stack)}

La Pila è un insieme dinamico gestito con politica \textbf{LIFO (Last In First Out)}: si entra e si esce solo dalla "cima".

\begin{concept}{Operazioni Fondamentali}
Una pila supporta le seguenti operazioni, tutte con complessità $\Theta(1)$:
\begin{itemize}
    \item \textbf{PUSH:} Aggiunge un elemento in cima alla pila.
    \item \textbf{POP:} Rimuove l'elemento in cima alla pila.
    \item \textbf{QUERY:}
    \begin{itemize}
        \item \texttt{ISEMPTY}: Verifica se la pila è vuota.
        \item \texttt{TOP}: Restituisce il valore dell'elemento in cima senza rimuoverlo.
    \end{itemize}
\end{itemize}
\end{concept}

\subsection{Implementazione con Array}
Struttura: $\text{PILA} = \langle A, \text{TOP} \rangle$.
\begin{itemize}
    \item \texttt{TOP}: indice che varia tra $0$ e $n-1$. Vale $-1$ se la pila è vuota.
    \item \textbf{Svantaggio:} Bisogna definire a priori la dimensione massima ($A.length$).
\end{itemize}

\begin{codebox}{Algoritmi Pila (Array)}
\textbf{ISEMPTY(PILA)}
\begin{verbatim}
IF (TOP < 0) THEN RETURN TRUE
ELSE RETURN FALSE
\end{verbatim}

\textbf{TOP(PILA)}
\begin{verbatim}
IF ISEMPTY(PILA) THEN error
ELSE RETURN A[TOP]
\end{verbatim}

\textbf{PUSH(PILA, X)}
\begin{verbatim}
TOP = TOP + 1
IF (TOP >= A.length) THEN error (Overflow)
A[TOP] = X
\end{verbatim}

\textbf{POP(PILA)}
\begin{verbatim}
IF ISEMPTY(PILA) THEN error (Underflow)
ELSE
    X = A[TOP]
    TOP = TOP - 1
    RETURN X
\end{verbatim}
\end{codebox}

\begin{explanation}{Dettagli Implementativi (Stack Array)}
\begin{enumerate}
    \item \textbf{TOP}: Mantiene l'indice dell'ultimo elemento inserito.
    \item \textbf{PUSH}: Incrementa prima \texttt{TOP}, poi scrive. Se \texttt{TOP} supera la capacità, si ha \textit{Overflow}.
    \item \textbf{POP}: Legge l'elemento a \texttt{TOP} e poi decrementa. Se \texttt{TOP} diventa negativo, lo stack è vuoto (Underflow).
\end{enumerate}
\end{explanation}

\begin{center}
\begin{tikzpicture}[scale=0.8]
    \node[draw, minimum width=2cm, minimum height=0.6cm] (c1) at (0,0) {Data A};
    \node[draw, minimum width=2cm, minimum height=0.6cm, above=0cm of c1] (c2) {Data B};
    \node[draw, minimum width=2cm, minimum height=0.6cm, above=0cm of c2] (c3) {};
    \node[right=0.5cm of c2] {$\leftarrow$ TOP};
    \node[above=0.5cm of c3] {Stack (Array)};
\end{tikzpicture}
\end{center}

\subsection{Implementazione con Lista}
Struttura gestita tramite un puntatore \texttt{TOPER} che indica la testa della lista (cima della pila). Non ha limiti di dimensione fissa.

\begin{codebox}{Algoritmi Pila (Lista)}
\textbf{ISEMPTY(TOPER)}
\begin{verbatim}
IF (TOPER == NULL) THEN RETURN TRUE
ELSE RETURN FALSE
\end{verbatim}

\textbf{PUSH(TOPER, X)} (X è il nuovo nodo)
\begin{verbatim}
X.next = TOPER
TOPER = X
\end{verbatim}

\textbf{POP(TOPER)}
\begin{verbatim}
IF ISEMPTY(TOPER) THEN error
val = TOPER.key
TOPER = TOPER.next
RETURN val
\end{verbatim}
\end{codebox}

\begin{explanation}{Dettagli Implementativi (Stack Lista)}
\begin{itemize}
    \item Non esiste limite di dimensione (se non la memoria).
    \item \textbf{PUSH}: Inserimento in testa ($O(1)$). Il nuovo nodo punta alla vecchia testa.
    \item \textbf{POP}: Rimozione in testa ($O(1)$). Si aggiorna \texttt{TOPER} al successivo.
\end{itemize}
\end{explanation}

---

\section{Code (Queue)}

La Coda è un insieme dinamico gestito con politica \textbf{FIFO (First In First Out)}: si entra solo dalla coda (\texttt{tail}) e si esce solo dalla testa (\texttt{head}).

\subsection{Implementazione con Array Circolare}
Struttura: $\text{CODA} = \langle A, \text{head}, \text{tail} \rangle$.
L'array è trattato come circolare usando l'aritmetica modulare.

\begin{concept}{Gestione Indici Circolari}
Gli indici avanzano modulo la lunghezza dell'array ($A.length$).
Esempio calcolo indici:
$$ (6+1) \pmod{7} = 0 $$
$$ (3+1) \pmod{7} = 4 $$
\end{concept}

\begin{codebox}{Algoritmi Coda (Array Circolare) - $\Theta(1)$}
\textbf{ISEMPTY(CODA)}
\begin{verbatim}
RETURN (head == tail)
\end{verbatim}

\textbf{FIRST(CODA)}
\begin{verbatim}
IF ISEMPTY(CODA) THEN error
ELSE RETURN A[head]
\end{verbatim}

\textbf{ENQUEUE(CODA, X)}
\begin{verbatim}
// Controllo Overflow (Coda piena se tail+1 tocca head)
IF (head == (tail + 1) % A.length) THEN error
A[tail] = X
tail = (tail + 1) % A.length
\end{verbatim}

\textbf{DEQUEUE(CODA)}
\begin{verbatim}
IF ISEMPTY(CODA) THEN error
ELSE
    X = A[head]
    head = (head + 1) % A.length
    RETURN X
\end{verbatim}
\end{codebox}

\begin{explanation}{Logica Coda Circolare}
L'uso del \textbf{modulo} (\texttt{\%}) permette di riutilizzare gli spazi liberi all'inizio dell'array quando la coda avanza.
\begin{itemize}
    \item \texttt{tail}: Punta alla prima cella \textit{libera}.
    \item \texttt{head}: Punta al primo elemento valido.
    \item \textbf{Coda Piena}: Se avanzando \texttt{tail} si raggiunge \texttt{head} ($tail+1 == head$).
\end{itemize}
\end{explanation}

\begin{center}
\begin{tikzpicture}[scale=0.7]
    \foreach \x in {0,1,...,7}
       \node[draw, circle, minimum size=0.8cm] (n\x) at (\x*45:2cm) {};
    \node[fill=blue!20, circle, minimum size=0.7cm] at (0:2cm) {H};
    \node[fill=red!20, circle, minimum size=0.7cm] at (135:2cm) {T};
    \node at (0,0) {Circular Buffer};
\end{tikzpicture}
\end{center}

\subsection{Implementazione con Lista}
Manteniamo due puntatori: \texttt{head} (per estrazione) e \texttt{tail} (per inserimento).

\begin{codebox}{Algoritmi Coda (Lista)}
\textbf{ISEMPTY(head, tail)}
\begin{verbatim}
IF (head == NULL) THEN RETURN TRUE
ELSE RETURN FALSE
\end{verbatim}

\textbf{ENQUEUE(head, tail, X)}
\begin{verbatim}
IF ISEMPTY(head, tail) THEN
    head = X
ELSE
    tail.next = X
tail = X  // Aggiornamento della coda (implicito ma necessario)
\end{verbatim}

\textbf{DEQUEUE(head, tail)}
\begin{verbatim}
IF ISEMPTY(head, tail) THEN error
ELSE
    val = head.key
    IF (head == tail) THEN tail = NULL // Caso svuotamento coda
    head = head.next
    RETURN val
\end{verbatim}
\end{codebox}

\begin{explanation}{Gestione coda con Lista}
Manteniamo due puntatori per garantire operazioni $O(1)$:
\begin{itemize}
    \item \textbf{ENQUEUE}: Inserisce in coda (\texttt{tail}). Aggiorna \texttt{tail.next} e poi \texttt{tail}.
    \item \textbf{DEQUEUE}: Rimuove dalla testa (\texttt{head}). Se la lista si svuota, bisogna aggiornare anche \texttt{tail} a NULL.
\end{itemize}
\end{explanation}




\newpage % FILE: Lezione19.tex
% ===================================================================

\part{Lezione 19: Alberi Binari (Approfondimenti)}

% ============================================================
% SEZIONE 1: NODO CENTRALE
% ============================================================
\section{Il Nodo Centrale}

Questa sezione analizza un problema specifico sugli Alberi Binari (AB).

\begin{concept}{Definizione: Nodo Centrale}
Dato un Albero Binario (AB) $T$, un nodo $u$ non vuoto è detto \textbf{CENTRALE} se:
\begin{center}
    \textit{La dimensione del sottoalbero di cui $u$ è radice (numero di nodi)}
    
    \textbf{È PARI A}
    
    \textit{La somma delle chiavi dei nodi che appartengono al percorso dalla radice dell'albero al nodo $u$ stesso.}
\end{center}
\end{concept}

\subsection{Algoritmo Risolutivo}
Il problema richiede di progettare un algoritmo che stampi le chiavi di tutti i nodi centrali.

\begin{figure}[h]
\centering
\begin{tikzpicture}[
  level distance=1.2cm,
  sibling distance=1.5cm,
  every node/.style={circle, draw=gray, fill=white, minimum size=6mm},
  central/.style={circle, draw=paletteRegalia, fill=paletteLenurple!30, very thick},
  path/.style={draw=paletteIndigo, line width=1.5pt, dashed}
]
  % Rappresentazione concettuale
  \node (root) {R}
    child { node {.} }
    child { node (u_padre) {.}
      child { node[central] (u) {u}
        child { node {.} }
        child { node {.} 
            child { node {.} }
        }
      }
      child { node {.} }
    };
  \draw[path, ->] (root) -- (u_padre);
  \draw[path, ->] (u_padre) -- (u);
  
  \node[right=3cm of root, text width=5cm, align=left, color=paletteIndigo] {
    \textbf{Concetto Visivo:}\\
    Percorso (Tratteggiato) vs\\
    Sottoalbero (Sotto $u$).
  };
\end{tikzpicture}
\caption{Rappresentazione logica del Nodo Centrale}
\end{figure}

\begin{codebox}{Pseudocodice: CENTRALI(u, SUM)}
\textbf{Idea:} Usiamo una visita \textit{posticipata} (post-order). La funzione restituisce la dimensione del sottoalbero al chiamante, ma internamente verifica la condizione di centralità usando il parametro accumulatore \texttt{SUM}.

\begin{algorithmic}[1]
\Function{Centrali}{$u, SUM$}
    \If{$u == \text{nil}$} \Comment{Caso Base: Albero vuoto}
        \State \Return 0
    \EndIf
    
    \State \textcolor{paletteIndigo}{\textbf{Visita Ricorsiva (Posticipata)}}
    \State $dim_{sx} \gets \Call{Centrali}{u.left, SUM + u.key}$
    \State $dim_{dx} \gets \Call{Centrali}{u.right, SUM + u.key}$
    
    \State \textcolor{paletteIndigo}{\textbf{Calcolo Dimensione Locale}}
    \State $dim_u \gets dim_{sx} + dim_{dx} + 1$
    
    \State \textcolor{paletteIndigo}{\textbf{Verifica Condizione}}
    \If{$dim_u == (SUM + u.key)$}
        \State \Call{Print}{u.key} \Comment{Stampa il nodo centrale}
    \EndIf
    
    \State \Return $dim_u$ \Comment{Restituisce la dimensione al padre}
\EndFunction
\end{algorithmic}
\end{codebox}

\newpage

% ============================================================
% SEZIONE 2: VISITE DEGLI ALBERI
% ============================================================
\section{Visite degli Alberi}
Analisi delle visite (Anticipata, Simmetrica, Posticipata) su un albero specifico tratto dagli appunti.

\subsection{Albero di Esempio e Tracce}
Ricostruzione dell'albero basata sulle tracce delle visite presenti nel manoscritto (Pag. 4).

\begin{figure}[h]
\centering
\begin{tikzpicture}[
  level distance=1.5cm,
  sibling distance=2.5cm,
  every node/.style={circle, draw=paletteIndigo, fill=white, thick, minimum size=8mm, drop shadow},
  edge from parent/.style={draw=black, thick, ->}
]
  % Radice 7
  \node {7}
    child { node {2}
      child { node {8}
        child { node {1} }
        child[missing] % Nodo invisibile per bilanciare
      }
      child { node {4} }
    }
    child { node {5}
      child { node {6} }
      child { node {9}
        child { node {3} }
        child[missing]
      }
    };
\end{tikzpicture}
\caption{\textbf{Struttura dell'Albero}: Radice 7, Figli 2 e 5.}
\end{figure}

\begin{description}
    \item[\textcolor{paletteIndigo}{1. Visita Anticipata (Pre-Order)}] \hfill \\
    Ordine: Radice $\to$ Sinistra $\to$ Destra. \\
    \textbf{Sequenza:} $7, 2, 8, 1, 4, 5, 6, 9, 3$

    \item[\textcolor{paletteIndigo}{2. Visita Simmetrica (In-Order)}] \hfill \\
    Ordine: Sinistra $\to$ Radice $\to$ Destra. \\
    \textbf{Sequenza:} $1, 8, 2, 4, 7, 6, 5, 3, 9$

    \item[\textcolor{paletteIndigo}{3. Visita Posticipata (Post-Order)}] \hfill \\
    Ordine: Sinistra $\to$ Destra $\to$ Radice. \\
    \textbf{Sequenza:} $1, 8, 4, 2, 6, 3, 9, 5, 7$
\end{description}

\vspace{1cm}

% ============================================================
% SEZIONE 3: CONTA FOGLIE
% ============================================================
\section{Conteggio Foglie}

Algoritmo ricorsivo per contare le foglie di un albero binario.

\begin{codebox}{Pseudocodice: CONTAFOGLIE(u)}
\begin{algorithmic}[1]
\Function{ContaFoglie}{$u$}
    \If{$u == \text{nil}$} \Comment{Albero vuoto}
        \State \Return 0
    \EndIf
    
    \Comment{È una foglia se entrambi i figli sono nil}
    \If{$(u.left == \text{nil}) \land (u.right == \text{nil})$}
        \State \Return 1
    \EndIf
    
    \Comment{Passo Ricorsivo: somma foglie sx + foglie dx}
    \State \Return \Call{ContaFoglie}{u.left} + \Call{ContaFoglie}{u.right}
\EndFunction
\end{algorithmic}
\end{codebox}

\begin{concept}{Analisi Complessità}
$$T(n) = O(n)$$
L'algoritmo visita ogni nodo esattamente una volta. La relazione è $n = l + r$ (nodi totali = foglie + nodi interni).
\end{concept}

\newpage

% ============================================================
% SEZIONE 4: ALBERO COMPLETO
% ============================================================
\section{Verifica Albero Completo}

L'algoritmo seguente verifica se un albero è "Completo" secondo la definizione data negli appunti (ogni nodo deve avere 0 o 2 figli). In letteratura questo è spesso noto come \textit{Albero Strettamente Binario}.

\begin{codebox}{Pseudocodice: COMPLETO(u)}
Restituisce \texttt{TRUE} se la proprietà è verificata, \texttt{FALSE} altrimenti.

\begin{algorithmic}[1]
\Function{Completo}{$u$}
    \If{$u == \text{nil}$} \Comment{Vuoto è completo}
        \State \Return \textbf{true}
    \EndIf
    
    \Comment{Se foglia, ok}
    \If{$(u.left == \text{nil}) \land (u.right == \text{nil})$}
        \State \Return \textbf{true}
    \EndIf
    
    \Comment{Controllo figli spaiati (XOR logico)}
    \If{$(u.left == \text{nil}) \neq (u.right == \text{nil})$}
        \State \Return \textbf{false} \Comment{Ha un solo figlio}
    \EndIf
    
    \State \Return \Call{Completo}{u.left} $\land$ \Call{Completo}{u.right}
\EndFunction
\end{algorithmic}
\end{codebox}

\vspace{1cm}

% ============================================================
% SEZIONE 5: ESERCIZI D'ESAME
% ============================================================
\section{Esercizi (Vecchio Compitino)}

\subsection{Problema: Chiave Doppia del Padre}
\textbf{Testo:} Scrivere un algoritmo che restituisca il numero delle foglie la cui chiave è il doppio di quella del genitore ($u.key == 2 \cdot u.p.key$).

\begin{figure}[h]
\centering
\begin{tikzpicture}[
  level distance=1.5cm,
  sibling distance=3cm,
  every node/.style={circle, draw=paletteIndigo, fill=white, thick, minimum size=10mm, drop shadow},
  match/.style={circle, draw=green!60!black, fill=green!10, very thick}
]
  % Esempio visuale basato su Pagina 7
  \node {2}
    child { node {7}
      child { node {1} }
      child { node[match, label=below:\small{$14=2\times7$}] {14} } 
    }
    child { node {6}
      child { node {3} }
      child { node {8} 
         child { node[match, label=below:\small{$16=2\times8$}] {16} }
         child[missing]
      }
    };
\end{tikzpicture}
\caption{Esempio: Nodi foglia (evidenziati) che soddisfano la condizione.}
\end{figure}

\begin{codebox}{Soluzione: CONTA(u)}
\begin{algorithmic}[1]
\Function{Conta}{$u$}
    \If{$u == \text{nil}$} \State \Return 0 \EndIf
    
    \If{$(u.left == \text{nil}) \land (u.right == \text{nil})$} \Comment{Se è foglia}
        \If{$(u.p \neq \text{nil}) \land (u.key == 2 * u.p.key)$}
            \State \Return 1
        \Else
            \State \Return 0
        \EndIf
    \EndIf
    
    \State \Return \Call{Conta}{u.left} + \Call{Conta}{u.right}
\EndFunction
\end{algorithmic}
\end{codebox}

\subsection{Problema: Stampa Chiave e Profondità}
Progettare un algoritmo che stampi chiave e profondità di ciascun nodo.
Si suggerisce una visita anticipata passando la profondità come parametro incrementale: `Stampa(u.left, depth+1)`.


\newpage
\part{Errori Comuni e Note}
% =======================================================
% FILE: errori-comuni-complessita.tex
% DA INCLUDERE NEL MAIN DOCUMENT
% =======================================================

\section{Errori Comuni da Evitare nell'Analisi}

Questa sezione riassume gli errori più frequenti che si possono commettere nell'analisi della complessità e nell'uso della notazione asintotica.

\subsection{Errore nel Calcolo della Complessità Totale}
Uno degli errori più comuni riguarda la combinazione delle complessità quando diverse fasi di un algoritmo vengono eseguite in sequenza.

\subsubsection{Moltiplicare Invece di Sommare}
L'errore comune consiste nel \textbf{moltiplicare} le complessità delle diverse fasi, anziché \textbf{sommarle}, quando queste sono eseguite in sequenza.

\begin{example}[Errore di Moltiplicazione]
    Si consideri un algoritmo composto da tre fasi eseguite consecutivamente:
    \begin{enumerate}
        \item Fase 1: $O(n^2)$
        \item Fase 2: $O(n \log n)$
        \item Fase 3: $O(n)$
    \end{enumerate}
    \textbf{Calcolo Errato}: Complessità totale $\neq O(n^2) \cdot O(n \log n) \cdot O(n) = O(n^4 \log n)$.
\end{example}

\begin{observation}[Regola Fondamentale]
    Quando le operazioni sono eseguite \textbf{in sequenza} (l'una dopo l'altra), il tempo totale è la somma dei tempi. La complessità finale è data dal termine dominante (quello con l'ordine di grandezza maggiore):
    $$ T_{totale}(n) = T_1(n) + T_2(n) + T_3(n) $$
    \textbf{Calcolo Corretto}:
    $$ O(n^2) + O(n \log n) + O(n) = \mathbf{O(n^2)} $$
    La moltiplicazione delle complessità si applica unicamente quando le operazioni sono \textbf{annidate} (es. un ciclo iterativo interno a un altro ciclo).
\end{observation}

\subsection{Imprecisioni Terminologiche sulle Strutture Dati}

\subsubsection{Definire un Array "Disordinato"}
È un errore usare l'aggettivo "disordinato" per descrivere lo stato di una struttura dati (come un array). La caratteristica dell'ordine è una proprietà booleana.

\begin{note}
    Non si deve dire che l'array è "disordinato".
    Si deve dire che l'array \textbf{non è ordinato}.
\end{note}

\subsection{Errori di Definizione sulle Notazioni Asintotiche (\texorpdfstring{$\mathbf{O}, \mathbf{\Theta}, \mathbf{\Omega}$}{O, Theta, Omega})}
Le notazioni asintotiche (O-grande, Theta, Omega) non definiscono un'uguaglianza tra funzioni, ma una \textbf{relazione di limitazione} del tasso di crescita asintotico.

\subsubsection{Confondere l'Appartenenza con l'Eguaglianza}
\begin{note}
    È un errore affermare che $\Theta = \text{equazione}$ (es. $\Theta = n^2$). La notazione non è un'equazione in senso stretto e non rappresenta una singola funzione.
\end{note}
La scrittura $f(n) = O(g(n))$ non indica un'uguaglianza, ma significa che la funzione $f(n)$ \textbf{appartiene all'insieme} delle funzioni che crescono al più come $g(n)$ (a meno di una costante per $n$ sufficientemente grande).

\begin{definition}[Sintesi Notazioni]
    Sia $g(n)$ una funzione di riferimento.
    \begin{itemize}
        \item $\mathbf{O(g(n))}$ (\textbf{O-grande}): Indica il \textbf{limite superiore} (Worst Case). Una funzione $f(n)$ è $O(g(n))$ se $f(n)$ cresce al più velocemente di $g(n)$.
        \item $\mathbf{\Omega(g(n))}$ (\textbf{Omega}): Indica il \textbf{limite inferiore} (Best Case). Una funzione $f(n)$ è $\Omega(g(n))$ se $f(n)$ cresce almeno tanto velocemente quanto $g(n)$.
        \item $\mathbf{\Theta(g(n))}$ (\textbf{Theta}): Indica l'\textbf{ordine esatto} (Average Case o limite sia superiore che inferiore). Una funzione $f(n)$ è $\Theta(g(n))$ se $f(n)$ cresce esattamente con lo stesso tasso di $g(n)$.
    \end{itemize}
\end{definition}

\subsubsection{Errore nella Spiegazione dei Simboli}
È un errore comune spiegare in modo confuso o invertito le limitazioni date dalle tre notazioni.

\begin{observation}
    Quando si spiega $f(n) = O(g(n))$, non significa che $f(n)$ sia limitata da $g(n)$ nel senso di una frazione con un risultato specifico. Significa che esiste una costante positiva $c$ e un $n_0$ tali che:
    $$ 0 \leq f(n) \leq c \cdot g(n) \quad \forall n \geq n_0 $$
    La definizione richiede che $f(n)$ sia \textbf{asintoticamente dominata} da $g(n)$, a meno di un fattore costante.
\end{observation}


\end{document}
