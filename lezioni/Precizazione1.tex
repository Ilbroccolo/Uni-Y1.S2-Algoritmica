% ===================================================================
% FILE: Precizazione1.tex
% ===================================================================

\part {Master's Theorem}
Quando si tratta di risolvere equazioni di ricorrenza \textbf{bilanciate}, è possibile utilizzare il Master's Theorem.
\begin{equation}
    T(n)=\begin{cases}
        \Theta(1) & n \leq k \\
        a \cdot T(\frac{n}{b}) + f(n) & n >k
    \end{cases}
\end{equation}
L'intuizione consiste nel fare un confronto tra $f(n)$ e $n^{\log_b{a}}$.
% --- APPLICAZIONE STILE ---
\begin{definition}[Master Theorem: I Tre Casi]
Ci sono tre casi possibili:
\begin{itemize}
    \item \textbf{Minore}: $f(n) = O(n^{\log_b{a}-\epsilon})$ per qualche costante $\epsilon > 0$.
    $f(n)$ cresce \textbf{polinomialmente} più lentamente di $n^{\log_b{a}}$.
    \emph{Soluzione}: $T(n) = \Theta(n^{\log_b{a}})$.
    
    \begin{example}
        Data la seguente equazione di ricorrenza:
        \begin{equation}
            T(n) = 9 \cdot T(\frac{n}{3}) + n
        \end{equation}
        Abbiamo che $a=9$, $b=3$, $f(n) = n$, $n^{\log_3 9} = n^2$.Possiamo dedurre quindi che, per un $\epsilon = 1$:
        \begin{equation}
            f(n)  = n = O(n^{\log_3 9 - \epsilon}) = O(n)
        \end{equation}
    \end{example}
    
    \item \textbf{Uguale}: $f(n) = \Theta(n^{\log_b{a}}\cdot \ln^k{n})$ per qualche costante $k \geq 0$.
    $f(n)$ e $n^{\log_b{a}}$ crescono allo stesso modo.
    \emph{Soluzione}: $T(n) = \Theta(n^{\log_b{a}} \cdot \ln^{k+1}{n})$.
    \item \textbf{Maggiore}: $f(n) = \Omega(n^{\log_b{a} + \epsilon})$ per qualche costante $\epsilon > 0$.
    $f(n)$ cresce \textbf{polinomialmente} più in fretta e rispetta la \textbf{condizione di regolarità}: $a \cdot f(\frac{n}{b}) \leq c \cdot f(n)$ per $c<1$.
    \emph{Soluzione}: $T(n) = \Theta(f(n))$.
\end{itemize}
\end{definition}


\begin{observation}
    Il Master's Theorem si può usare solamente quando $f(x)$ cresce \textbf{polinomialmente} più in fretta o lentamente di $n^{\log_b{a}}$.
\end{observation}


\newpage
\section{Guida Pratica all'Applicazione del Master's Theorem}

Il Teorema Master è uno strumento potente per risolvere equazioni di ricorrenza \textbf{bilanciate}.
La forma standard è:
\[
    T(n) = a \cdot T(\frac{n}{b}) + f(n)
\]

% --- APPLICAZIONE STILE ---
\begin{observation}[Logica: "Tradurre" l'Equazione]
Per usare il teorema, devi prima "tradurre" la tua equazione:
\begin{itemize}
    \item \textbf{$a$}: Il numero di \textbf{sotto-problemi} ($a \geq 1$).
    \item \textbf{$n/b$}: La dimensione di \textbf{ciascun sotto-problema} ($b > 1$).
    \item \textbf{$f(n)$}: Il costo "extra" per \textbf{dividere} e \textbf{combinare}.
\end{itemize}
L'idea centrale è confrontare $f(n)$ con $n^{\log_b{a}}$.
\end{observation}

\paragraph{Come Applicarlo in un Esercizio: Passo Passo}

\begin{example}[Guida Passo-Passo: $T(n) = 9 \cdot T(\frac{n}{3}) + n$]
    
    \subparagraph{Passo 1: Identificare $a$, $b$, e $f(n)$}
    Dall'equazione $T(n) = 9 \cdot T(\frac{n}{3}) + n$:
    \begin{itemize}
        \item $a = 9$ (sotto-problemi)
        \item $b = 3$ (dimensione 1/3)
        \item $f(n) = n$ (costo extra)
    \end{itemize}
    
    
    
    \subparagraph{Passo 2: Calcolare il "Confine" Ricorsivo}
    Calcola il valore $n^{\log_b{a}}$.
    \begin{itemize}
        \item $n^{\log_3 9} = n^2$
    \end{itemize}
    
    \subparagraph{Passo 3: Confrontare $f(n)$ con $n^{\log_b{a}}$}
    Confrontiamo $f(n) = n$ con $n^2$.
    \paragraph{I Tre Casi del Teorema}
    \subparagraph{Caso 1: $f(n)$ cresce più LENTAMENTE}
    \begin{itemize}
        \item \textbf{Logica:} Il costo è dominato dalla ricorsione.
        \item \textbf{Condizione:} $f(n) = O(n^{\log_b{a}-\epsilon})$ per $\epsilon > 0$.
        \item \textbf{Soluzione:} $T(n) = \Theta(n^{\log_b{a}})$.
    \end{itemize}
    
    \subparagraph{Verifica del nostro Esempio:}
    $f(n) = n$ e $n^{\log_b{a}} = n^2$.
    $f(n) = n$ è $O(n^{2-\epsilon})$ scegliendo $\epsilon=1$.
    Rientriamo nel \textbf{Caso 1}.
    \emph{Soluzione:} $T(n) = \Theta(n^2)$.
    
    \subparagraph{Caso 2: $f(n)$ cresce alla STESSA VELOCITÀ}
    \begin{itemize}
        \item \textbf{Logica:} Costo bilanciato.
        \item \textbf{Condizione:} $f(n) = \Theta(n^{\log_b{a}}\cdot \ln^k{n})$ per $k \geq 0$.
        \item \textbf{Soluzione:} $T(n) = \Theta(n^{\log_b{a}} \cdot \ln^{k+1}{n})$.
    \end{itemize}
    
    \subparagraph{Caso 3: $f(n)$ cresce più VELOCEMENTE}
    \begin{itemize}
        \item \textbf{Logica:} Costo dominato dal lavoro extra $f(n)$.
        \item \textbf{Condizione 1:} $f(n) = \Omega(n^{\log_b{a} + \epsilon})$ per $\epsilon > 0$.
        \item \textbf{Condizione 2 (Regolarità):} $a \cdot f(\frac{n}{b}) \leq c \cdot f(n)$ per $c < 1$.
        \item \textbf{Soluzione:} $T(n) = \Theta(f(n))$.
    \end{itemize}
\end{example}

\begin{observation}[Limiti del Teorema]
Il Teorema Master \textbf{non si può usare} se $f(n)$ non cresce \emph{polinomialmente} più velocemente o più lentamente di $n^{\log_b{a}}$.
Ad esempio, se $f(x) = \log{n}$ non potremmo utilizzarlo (perché non è polinomialmente diverso da $n^0=1$).
\end{observation}

\newpage