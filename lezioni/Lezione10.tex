% ===================================================================
% FILE: Lezione25.tex
% ===================================================================

\part{Lezione 25 (17/11/2025)}

\section{Esercizio su Teorema Master (Confronto Asintotico)}

Consideriamo due algoritmi caratterizzati dalle seguenti ricorrenze:
\begin{itemize}
    \item[(A)] $T(n) = 7 T\left(\frac{n}{2}\right) + n^2$
    \item[(A')] $T'(n) = a T'\left(\frac{n}{4}\right) + n^2$
\end{itemize}

\textbf{Domanda:} Qual è il più grande valore di $a$ per cui $T'$ è asintoticamente più veloce di $T$?

\subsection{Costo in Tempo di A}
Analizziamo $T(n) = 7 T(n/2) + n^2$ con il Teorema Master.
\begin{itemize}
    \item $a=7, b=2, f(n)=n^2$.
    \item Calcoliamo lo spartiacque: $n^{\log_b a} = n^{\log_2 7} \approx n^{2.8}$.
    \item Confrontiamo con $f(n)$: $n^2 = O(n^{\log_2 7 - \epsilon})$ (con $\epsilon \approx 0.8$).
    \item Siamo nel \textbf{Caso 1}.
\end{itemize}
\[ T(n) = \Theta(n^{\log_2 7}) \]

\subsection{Costo in Tempo di A'}
Analizziamo $T'(n) = a T'(n/4) + n^2$.
\begin{itemize}
    \item $a=a, b=4, f(n)=n^2$.
    \item Spartiacque: $n^{\log_4 a}$.
\end{itemize}
Dobbiamo confrontare $\log_4 a$ con l'esponente di $f(n)$ (che è 2). Ci sono 3 casi possibili per $a$:

\begin{enumerate}
    \item \textbf{Caso 3 ($\log_4 a < 2 \iff a < 16$):}
    La forzante $n^2$ domina. $T'(n) = \Theta(n^2)$.
    Verifica condizione regolarità: $a(n/4)^2 \le c n^2 \implies a/16 \le c$. Vera per $a < 16$.
    In questo caso $T'(n) = \Theta(n^2)$, che è sicuramente più veloce di $\Theta(n^{2.8})$.

    \item \textbf{Caso 2 ($\log_4 a = 2 \iff a = 16$):}
    Equilibrio. $T'(n) = \Theta(n^2 \log n)$.
    Anche questo è più veloce di $\Theta(n^{2.8})$.

    \item \textbf{Caso 1 ($\log_4 a > 2 \iff a > 16$):}
    Le foglie dominano. $T'(n) = \Theta(n^{\log_4 a})$.
    Affinché $T'$ sia più veloce di $T$, deve valere:
    \[ n^{\log_4 a} < n^{\log_2 7} \implies \log_4 a < \log_2 7 \]
    Usando il cambio di base ($\log_4 a = \frac{\log_2 a}{2}$):
    \[ \frac{\log_2 a}{2} < \log_2 7 \implies \log_2 a < 2 \log_2 7 \implies \log_2 a < \log_2 49 \implies a < 49 \]
\end{enumerate}

\textbf{Soluzione:} $A'$ è più veloce di $A$ per $a < 49$. Il valore intero massimo è \textbf{48}.

\section{Analisi di Algoritmi (Esercizi Vari)}

\subsection{Esercizio "Mistero"}
\begin{algorithmic}[1]
    \Procedure{Mistero}{n}
        \If{$n < 10$} \Return 1 \EndIf
        \State $x = \Call{Mistero}{\lfloor n/4 \rfloor} + \Call{Mistero}{\lfloor n/4 \rfloor}$ \Comment{2 chiamate ricorsive}
        \State $L=1$
        \While{$i < n$} \Comment{Ciclo esterno}
            \State $j=1$
            \While{$j < n$} \Comment{Ciclo interno}
                \State...
                \State $j = j+1$
            \EndWhile
            \State $i = i \cdot 3$
        \EndWhile
        \State $y = \Call{Mistero}{\lfloor n/4 \rfloor}$ \Comment{1 chiamata ricorsiva}
        \State \Return $x+y$
    \EndProcedure
    \end{algorithmic}

\begin{explanation}{Analisi Ricorsiva}
La funzione combina chiamate ricorsive e cicli annidati:
\begin{itemize}
    \item \textbf{Ricorsione}: 3 chiamate su dimensione $n/4$ ($x$ ne fa 2, $y$ ne fa 1). $T(n) = 3T(n/4) + f(n)$.
    \item \textbf{Costo locale ($f(n)$)}: I cicli annidati. Il ciclo interno lavora su $j$, l'esterno su $i$ che cresce esponenzialmente ($3^k$).
\end{itemize}
\end{explanation}

    \textbf{Analisi:}
    \begin{itemize}
        \item Chiamate ricorsive: 3 chiamate su $n/4$. Quindi $a=3, b=4$.
        \item Costo f(n): Il ciclo interno è $\Theta(n)$, quello esterno è logaritmico? (Dagli appunti sembra indicato come $\Theta(n \log n)$).
        \item Ricorrenza: $T(n) = 3T(n/4) + \Theta(n \log n)$.
        \item Master Theorem:
        \begin{itemize}
            \item $n^{\log_4 3} \approx n^{0.79}$.
            \item $f(n) = n \log n$ è polinomialmente maggiore ($n^1 > n^{0.79}$).
            \item \textbf{Caso 3}.
        \end{itemize}
        \item Soluzione: $T(n) = \Theta(n \log n)$.
    \end{itemize}

    \subsection{Algo 1 (Radice Quadrata)}
    \begin{itemize}
        \item Ricorrenza data: $T(n) = 2T(n/4) + \sqrt{n}$.
        \item $a=2, b=4 \implies n^{\log_4 2} = n^{0.5} = \sqrt{n}$.
        \item $f(n) = \sqrt{n}$.
        \item Siamo nel \textbf{Caso 2} (uguali).
        \item Soluzione: $T(n) = \Theta(\sqrt{n} \log n)$.
    \end{itemize}

    \subsection{Algo 2 (Somma Ricorsiva)}
    \begin{itemize}
        \item Divide l'array in due parti ($p, q$ e $q+1, r$).
        \item Fa 2 chiamate ricorsive: $a=2, b=2$.
        \item Costo di combinazione (somma): $\Theta(1)$.
        \item Ricorrenza: $T(n) = 2T(n/2) + \Theta(1)$.
        \item Master Theorem: $n^{\log_2 2} = n^1$. $f(n) = n^0$.
        \item \textbf{Caso 1}.
        \item Soluzione: $T(n) = \Theta(n)$.
    \end{itemize}

    \subsection{Confronto Finale}
    Confrontiamo:
    \begin{enumerate}
        \item $T_A(n) = 9T(n/3) + 2n^2$.
        \item $T_{A'}(n) = 3T(n/2) + n^2 \log^2 n$.
    \end{enumerate}

    \textbf{Analisi A:}
    $a=9, b=3 \implies n^{\log_3 9} = n^2$. $f(n) = 2n^2$.
    Caso 2. $T_A(n) = \Theta(n^2 \log n)$.

    \textbf{Analisi A':}
    $a=3, b=2 \implies n^{\log_2 3} \approx n^{1.58}$. $f(n) = n^2 \log^2 n$.
    $f(n)$ è maggiore dello spartiacque. Caso 3.
    $T_{A'}(n) = \Theta(n^2 \log^2 n)$.

    \textbf{Conclusione:} $T_A$ è asintoticamente migliore (più veloce) di $T_{A'}$.

    \newpage
