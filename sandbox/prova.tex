\documentclass[12pt,a4paper]{article}

% --- PAZCHETTI DI BASE E LINGUA ---
\usepackage[utf8]{inputenc}
\usepackage[T1]{fontenc}
\usepackage[italian]{babel}
\usepackage{lmodern} % Font moderno

% --- LAYOUT E GRAFICA ---
\usepackage[margin=2.5cm]{geometry} % Margini
\usepackage{xcolor} % Per i colori
\usepackage{graphicx}
\usepackage{float} % Per [H]
\usepackage{caption} % Per didascalie
\usepackage[hidelinks]{hyperref} % Link cliccabili (ma non visibili)

% --- MATEMATICA E ALGORITMI ---
\usepackage{amsmath}
\usepackage{amssymb}
\usepackage{algorithm}
\usepackage{algpseudocode} % Per lo pseudocodice

% --- PAZCHETTI PER I GRAFICI (TikZ e PGFPlots) ---
\usepackage{tikz}
\usetikzlibrary{shapes.geometric, arrows.meta, positioning, calc}
\usepackage{pgfplots}
\pgfplotsset{compat=1.17} % Imposta versione di pgfplots

% --- DEFINIZIONE COLORI PASTELLO ---
\definecolor{pastelyellow}{HTML}{FCF7DE}
\definecolor{pastelgreen}{HTML}{DEFDE0}
\definecolor{pastelred}{HTML}{FDE0E0}
\definecolor{pastelsalmon}{HTML}{FDDEDF}
\definecolor{pastelblue}{HTML}{DEF3FD}
\definecolor{pastelpurple}{HTML}{E0DEFd}
\definecolor{textdark}{HTML}{333333}
\definecolor{bordergray}{HTML}{AAAAAA}

% --- STILI GLOBALI PER TIKZ ---
\tikzset{
    % Stile per le celle dell'array
    cell/.style={
        draw=bordergray, 
        rounded corners=3pt, 
        minimum size=1cm, 
        font=\sffamily\color{textdark}, 
        fill=gray!10
    },
    % Stili per evidenziare
    cell_check/.style={cell, fill=pastelyellow},
    cell_found/.style={cell, fill=pastelgreen},
    cell_current/.style={cell, fill=pastelsalmon},
    cell_discard/.style={cell, fill=gray!20, draw=gray!40},
    % Stile per i nodi dell'albero
    tree_node/.style={
        draw=bordergray, 
        circle, 
        fill=pastelblue, 
        font=\sffamily\color{textdark},
        minimum size=8mm
    },
    tree_found/.style={tree_node, fill=pastelgreen},
    % Stili per le frecce
    arrow/.style={-Latex, thick, draw=bordergray},
    arrow_path/.style={-Latex, very thick, draw=pastelsalmon}
}


% --- INIZIO DOCUMENTO ---
\begin{document}

\title{Lezione 11: Algoritmi e Complessità\thanks{Documento generato e rivisto, unendo appunti e visualizzazioni.}}
\author{13 Ottobre 2025}
\date{}
\maketitle

\tableofcontents
\newpage

\section{Definizione di Algoritmo}
[cite_start]Un algoritmo è una sequenza finita di operazioni elementari (passi), univocamente determinata (non ambiguo), che, se eseguita su un calcolatore, porta alla risoluzione di un problema[cite: 1].

\subsection{Modello RAM (Random Access Machine)}
[cite_start]Nel modello RAM, si assume che le seguenti operazioni elementari abbiano costo "unitario" (costante)[cite: 2]:
\begin{itemize}
    [cite_start]\item \textbf{Operazioni aritmetiche}: +, -, *, /, \% [cite: 2]
    [cite_start]\item \textbf{Operazioni di confronto}: $<, >, ==, !=$ [cite: 2]
    [cite_start]\item \textbf{Operazioni logiche}: AND, OR, NOT [cite: 2]
    [cite_start]\item \textbf{Operazioni di trasferimento}: load/store/assegnamento [cite: 2]
    [cite_start]\item \textbf{Operazioni di controllo}: chiamata di funzione, RETURN [cite: 2]
\end{itemize}

\section{Analisi di Complessità}
[cite_start]Si analizza il costo computazionale (Tempo o Spazio) in funzione della dimensione dell'input, $n$[cite: 3, 4].
\begin{itemize}
    [cite_start]\item \textbf{Complessità in Tempo $T(n)$}: Numero di operazioni elementari eseguite[cite: 3].
    [cite_start]\item \textbf{Complessità in Spazio $S(n)$}: Numero di celle di memoria utilizzate (oltre a quelle dell'input)[cite: 4].
\end{itemize}

[cite_start]Ci si concentra sull'\textbf{ordine di grandezza} della funzione $T(n)$, ignorando costanti moltiplicative e termini di ordine inferiore[cite: 5].
[cite_start]Ad esempio, $T(n) = 3n + 2$ e $T(n) = 5n + \log n + 4$ sono entrambe considerate di complessità \textbf{Lineare}[cite: 6]. [cite_start]$T(n) = 8n^2$ è \textbf{Quadratica}[cite: 7].

\begin{figure}[H]
    \centering
    \begin{tikzpicture}
    \begin{axis}[
        width=12cm, height=8cm,
        axis lines=left,
        xlabel=$n$ (Dimensione input),
        ylabel=Operazioni (Costo),
        xmin=0, xmax=50,
        ymin=0, ymax=60,
        legend pos=north west,
        samples=100,
        domain=1:50, % Dominio da 1 per log(n)
        grid=major,
        grid style={dashed, gray!40}
    ]
        
    \addplot [very thick, color=pastelpurple] {1} node[right, pos=0.8, font=\small] {$O(1)$};
    \addlegendentry{Costante $O(1)$}
    
    \addplot [very thick, color=pastelblue] {log2(x)} node[right, pos=0.9, font=\small] {$O(\log n)$};
    \addlegendentry{Logaritmica $O(\log n)$}
    
    \addplot [very thick, color=pastelgreen] {x} node[right, pos=0.9, font=\small] {$O(n)$};
    \addlegendentry{Lineare $O(n)$}
    
    \addplot [very thick, color=pastelsalmon, domain=0:8] {x^2} node[right, pos=1, font=\small] {$O(n^2)$};
    \addlegendentry{Quadratica $O(n^2)$}
    
    \end{saxis}
    \end{tikzpicture}
    \caption{Confronto grafico degli ordini di grandezza}
\end{figure}

\subsection{Caso Ottimo, Pessimo, Medio}
\begin{itemize}
    [cite_start]\item \textbf{Caso Ottimo}: L'istanza di input che richiede il minor tempo[cite: 8].
    [cite_start]\item \textbf{Caso Pessimo}: L'istanza di input che richiede il maggior tempo[cite: 9].
    [cite_start]\item \textbf{Caso Medio}: Complessità media su tutte le possibili istanze[cite: 10].
\end{itemize}
[cite_start]Ci si concentra sul \textbf{caso pessimo} perché fornisce un limite superiore al costo: l'algoritmo non impiegherà mai più di $T(n)$[cite: 11].

\newpage % Separatore per gli esempi

%---------------------------------------------------------
\section{Esempio 1: Minimo in Vettore}
%---------------------------------------------------------
\begin{itemize}
    [cite_start]\item \textbf{Input}: Array $A[1..n]$ di interi[cite: 12].
    [cite_start]\item \textbf{Output}: Il valore minimo contenuto in $A$[cite: 12].
\end{itemize}

\begin{algorithmic}[1]
\Procedure{Minimo}{A, n} 
    \State $min = A[1]$ \Comment{Costo costante $c_1$}
    \For{$i = 2 \to n$} \Comment{Eseguito $n-1$ volte}
        \If{$A[i] < min$} \Comment{Costo $c_2$}
            \State $min = A[i]$ \Comment{Costo $c_3$}
        \EndIf
    \EndFor
    \State \Return $min$ \Comment{Costo costante $c_4$}
\EndProcedure
\end{algorithmic}

\textbf{Analisi:}
[cite_start]Il costo totale è $T(n) = c_1 + (n-1)(c_2 \text{ (confronto)} + c_3 \text{ (assegn. caso pessimo)}) + c_4$[cite: 13].
[cite_start]La complessità è \textbf{Lineare}, $T(n) \in \Theta(n)$, sia nel caso ottimo che in quello pessimo[cite: 14].

\begin{figure}[H]
\centering
\begin{tikzpicture}[scale=0.9]
\node[anchor=west] at (0, 1.5) {Array:};
\foreach \i/\val [count=\x from 0] in {7,2,9,1,5,8,3} {
    \node[cell] at (\x*1.2, 0) (cell-\x) {\val};
    \node at (\x*1.2, -0.7) {\tiny A[\x+1]};
}

% Passo 1: i=2
\begin{scope}[yshift=-3cm]
\node[anchor=west] at (-2, 0) {\textbf{i=2}: $min=7$. $A[2]=2 < 7$ \textbf{$\to$ Nuovo min=2}};
\foreach \i/\val [count=\x from 0] in {7,2,9,1,5,8,3} {
    \pgfmathtruncatemacro{\idx}{\x+1}
    \ifthenelse{\idx=1}{\def\style{cell_current}}{
        \ifthenelse{\idx=2}{\def\style{cell_found}}{\def\style{cell}}
    }
    \node[\style] at (\x*1.2, 0) (cell-\x) {\val};
}
\end{scope}

% Passo 2: i=3
\begin{scope}[yshift=-6cm]
\node[anchor=west] at (-2, 0) {\textbf{i=3}: $min=2$. $A[3]=9 > 2$};
\foreach \i/\val [count=\x from 0] in {7,2,9,1,5,8,3} {
    \pgfmathtruncatemacro{\idx}{\x+1}
    \ifthenelse{\idx=2}{\def\style{cell_current}}{
        \ifthenelse{\idx=3}{\def\style{cell_check}}{\def\style{cell}}
    }
    \node[\style] at (\x*1.2, 0) (cell-\x) {\val};
}
\end{scope}

% Passo 3: i=4
\begin{scope}[yshift=-9cm]
\node[anchor=west] at (-2, 0) {\textbf{i=4}: $min=2$. $A[4]=1 < 2$ \textbf{$\to$ Nuovo min=1}};
\foreach \i/\val [count=\x from 0] in {7,2,9,1,5,8,3} {
    \pgfmathtruncatemacro{\idx}{\x+1}
    \ifthenelse{\idx=2}{\def\style{cell_current}}{
        \ifthenelse{\idx=4}{\def\style{cell_found}}{\def\style{cell}}
    }
    \node[\style] at (\x*1.2, 0) (cell-\x) {\val};
}
\end{scope}

% ... (saltiamo i=5, i=6, i=7 per brevità) ...

% Passo finale
\begin{scope}[yshift=-12cm]
\node[anchor=west] at (-2, 0) {\textbf{Fine}: $min=1$.};
\foreach \i/\val [count=\x from 0] in {7,2,9,1,5,8,3} {
    \pgfmathtruncatemacro{\idx}{\x+1}
    \ifthenelse{\idx=4}{\def\style{cell_current}}{\def\style{cell}}
    \node[\style] at (\x*1.2, 0) (cell-\x) {\val};
}
\end{scope}
\end{tikzpicture}
\caption{Visualizzazione dell'algoritmo di ricerca del minimo}
\end{figure}

\newpage
%---------------------------------------------------------
\section{Esempio 2: Cerca K (Ricerca Sequenziale)}
%---------------------------------------------------------
\begin{itemize}
    [cite_start]\item \textbf{Input}: Array $A[1..n]$ di interi, $k$ intero[cite: 16].
    [cite_start]\item \textbf{Output}: $i$ tale che $A[i]=k$, o $-1$ se $k \notin A$[cite: 17].
\end{itemize}

\begin{algorithmic}[1]
\Procedure{Cerca-K}{A, n, k} 
    \State $i = 1$ 
    \State $trovato = \text{false}$ 
    \While{(\textbf{not} $trovato$) \textbf{and} ($i \le n$)} 
        \If{$A[i] == k$}
            \State $trovato = \text{true}$ 
        \Else
            \State $i = i + 1$ 
        \EndIf
    \EndWhile
    \If{$trovato$}
        [cite_start]\State \Return $i$ [cite: 18]
    \Else
        \State \Return $-1$ 
    \EndIf
\EndProcedure
\end{algorithmic}

\textbf{Analisi:}
\begin{itemize}
    \item \textbf{Caso Ottimo}: $k = A[1]$. Il ciclo `while` esegue 1 iterazione. [cite_start]$T(n) \in \Theta(1)$ (Costante)[cite: 19].
    \item \textbf{Caso Pessimo}: $k \notin A$ (o $k=A[n]$). Il ciclo `while` esegue $n$ iterazioni. [cite_start]$T(n) \in \Theta(n)$ (Lineare)[cite: 20, 21].
\end{itemize}

\begin{figure}[H]
\centering
\begin{tikzpicture}[scale=0.9]
\node[anchor=west] at (0, 1.5) {Array (Cerchiamo k=5):};
\foreach \i/\val [count=\x from 0] in {4,7,2,9,5,1,8} {
    \node[cell] at (\x*1.2, 0) (cell-\x) {\val};
    \node at (\x*1.2, -0.7) {\tiny A[\x+1]};
}

\begin{scope}[yshift=-3cm]
\node[anchor=west] at (-1, 0) {\textbf{i=1}: $A[1]=4 \neq 5$};
\foreach \i/\val [count=\x from 0] in {4,7,2,9,5,1,8} {
    \pgfmathtruncatemacro{\idx}{\x+1}
    \ifthenelse{\idx=1}{\def\style{cell_check}}{\def\style{cell}}
    \node[\style] at (\x*1.2, 0) (cell-\x) {\val};
}
\end{scope}

\begin{scope}[yshift=-6cm]
\node[anchor=west] at (-1, 0) {\textbf{i=2}: $A[2]=7 \neq 5$};
\foreach \i/\val [count=\x from 0] in {4,7,2,9,5,1,8} {
    \pgfmathtruncatemacro{\idx}{\x+1}
    \ifthenelse{\idx=2}{\def\style{cell_check}}{\def\style{cell}}
    \node[\style] at (\x*1.2, 0) (cell-\x) {\val};
}
\end{scope}

\begin{scope}[yshift=-9cm]
\node[anchor=west] at (-1, 0) {\textbf{i=3}: $A[3]=2 \neq 5$};
\foreach \i/\val [count=\x from 0] in {4,7,2,9,5,1,8} {
    \pgfmathtruncatemacro{\idx}{\x+1}
    \ifthenelse{\idx=3}{\def\style{cell_check}}{\def\style{cell}}
    \node[\style] at (\x*1.2, 0) (cell-\x) {\val};
}
\end{scope}

\begin{scope}[yshift=-12cm]
\node[anchor=west] at (-1, 0) {\textbf{i=4}: $A[4]=9 \neq 5$};
\foreach \i/\val [count=\x from 0] in {4,7,2,9,5,1,8} {
    \pgfmathtruncatemacro{\idx}{\x+1}
    \ifthenelse{\idx=4}{\def\style{cell_check}}{\def\style{cell}}
    \node[\style] at (\x*1.2, 0) (cell-\x) {\val};
}
\end{scope}

\begin{scope}[yshift=-15cm]
\node[anchor=west] at (-1, 0) {\textbf{i=5}: $A[5]=5 = 5$ \textbf{$\to$ TROVATO!}};
\foreach \i/\val [count=\x from 0] in {4,7,2,9,5,1,8} {
    \pgfmathtruncatemacro{\idx}{\x+1}
    \ifthenelse{\idx=5}{\def\style{cell_found}}{\def\style{cell}}
    \node[\style] at (\x*1.2, 0) (cell-\x) {\val};
}
\end{scope}
\end{tikzpicture}
\caption{Visualizzazione dell'algoritmo di ricerca sequenziale}
\end{figure}

\newpage
%---------------------------------------------------------
\section{Esempio 3: Minimo in Vettore Ordinato}
%---------------------------------------------------------
\begin{itemize}
    [cite_start]\item \textbf{Input}: Array $A[1..n]$ di interi, \textbf{ordinato}[cite: 22].
    [cite_start]\item \textbf{Output}: Il valore minimo contenuto in $A$[cite: 22].
\end{itemize}

\begin{algorithmic}[1]
\Procedure{Minimo-Ordinato}{A, n} 
    \State \Return $A[1]$ 
\EndProcedure
\end{algorithmic}
[cite_start]\textbf{Analisi}: $T(n) \in \Theta(1)$ (Costante)[cite: 23].

\vspace{1cm}
\hrule
\vspace{1cm}

%---------------------------------------------------------
\section{Esempio 4: Cerca K in Vettore Ordinato (Ricerca Binaria)}
%---------------------------------------------------------
\begin{itemize}
    [cite_start]\item \textbf{Input}: Array $A[1..n]$ di interi \textbf{ordinato}, $k$ intero[cite: 24].
    [cite_start]\item \textbf{Output}: $i$ tale che $A[i]=k$, o $-1$ se $k \notin A$[cite: 25].
\end{itemize}
[cite_start]L'idea è di confrontare $k$ con l'elemento centrale $A[q]$ e dimezzare lo spazio di ricerca[cite: 26].

\begin{algorithmic}[1]
\Procedure{BS-IT}{A, p, r, k} 
    \While{$p \le r$} 
        \State $q = \lfloor (p+r)/2 \rfloor$ 
        \If{$A[q] == k$}
            \State \Return $q$ 
        \ElsIf{$A[q] > k$}
            [cite_start]\State $r = q - 1$ [cite: 27]
        \Else
            \State $p = q + 1$ 
        \EndIf
    \EndWhile
    \State \Return $-1$ 
\EndProcedure
\end{algorithmic}

\textbf{Analisi:}
\begin{itemize}
    \item \textbf{Caso Ottimo}: $k = A[q]$ al primo ciclo. [cite_start]$T(n) \in \Theta(1)$ (Costante)[cite: 28].
    \item \textbf{Caso Pessimo}: $k \notin A$. Il numero di iterazioni è $\log_2 n$. [cite_start]$T(n) \in \Theta(\log n)$ (Logaritmica)[cite: 29].
\end{itemize}

\begin{figure}[H]
\centering
\begin{tikzpicture}[scale=0.9]
\node[anchor=west] at (0, 1.5) {Array ordinato (Cerchiamo k=9):};
\foreach \i/\val [count=\x from 0] in {1,3,5,7,9,11,13} {
    \node[cell] at (\x*1.2, 0) (cell-\x) {\val};
    \node at (\x*1.2, -0.7) {\tiny A[\x+1]};
}

% Passo 1
\begin{scope}[yshift=-3cm]
\node[anchor=west] at (-2.5, 0) {\textbf{Passo 1}: $p=1, r=7, q=4$. $A[4]=7 < 9 \to$ Cerca a destra};
% Array
\foreach \i/\val [count=\x from 0] in {1,3,5,7,9,11,13} {
    \pgfmathtruncatemacro{\idx}{\x+1}
    \ifthenelse{\idx<4}{\def\style{cell_discard}}{
        \ifthenelse{\idx=4}{\def\style{cell_check}}{\def\style{cell}}
    }
    \node[\style] at (\x*1.2, 0) (cell-\x) {\val};
}
% Sottolineo
\draw[thick, pastelpurple] (4*1.2, -1.2) -- (6*1.2, -1.2) node[midway, below, font=\small] {p=5, r=7};
\end{scope}

% Passo 2
\begin{scope}[yshift=-6cm]
\node[anchor=west] at (-2.5, 0) {\textbf{Passo 2}: $p=5, r=7, q=6$. $A[6]=11 > 9 \to$ Cerca a sinistra};
% Array
\foreach \i/\val [count=\x from 0] in {1,3,5,7,9,11,13} {
    \pgfmathtruncatemacro{\idx}{\x+1}
    \ifthenelse{\idx<5 \OR \idx>7}{\def\style{cell_discard}}{
        \ifthenelse{\idx=6}{\def\style{cell_check}}{
            \ifthenelse{\idx=7}{\def\style{cell_discard}}{\def\style{cell}}
        }
    }
    \node[\style] at (\x*1.2, 0) (cell-\x) {\val};
}
% Sottolineo
\draw[thick, pastelpurple] (4*1.2, -1.2) -- (4*1.2, -1.2) node[midway, below, font=\small] {p=5, r=5};
\end{scope}

% Passo 3
\begin{scope}[yshift=-9cm]
\node[anchor=west] at (-2.5, 0) {\textbf{Passo 3}: $p=5, r=5, q=5$. $A[5]=9 = 9 \to$ \textbf{TROVATO!}};
% Array
\foreach \i/\val [count=\x from 0] in {1,3,5,7,9,11,13} {
    \pgfmathtruncatemacro{\idx}{\x+1}
    \ifthenelse{\idx=5}{\def\style{cell_found}}{\def\style{cell_discard}}
    \node[\style] at (\x*1.2, 0) (cell-\x) {\val};
}
\end{scope}

% Albero di decisione
\begin{scope}[yshift=-13cm]
\node at (4*1.2, 1) {Albero di decisione:};
% Nodi
\node[tree_node] (root) at (3*1.2, 0) {A[4]=7};
\node[tree_node] (l) at (1*1.2, -2) {A[2]=3};
\node[tree_node] (r) at (5*1.2, -2) {A[6]=11};
\node[tree_node] (ll) at (0*1.2, -4) {A[1]=1};
\node[tree_node] (lr) at (2*1.2, -4) {A[3]=5};
\node[tree_node, tree_found] (rl) at (4*1.2, -4) {A[5]=9}; % Trovato
\node[tree_node] (rr) at (6*1.2, -4) {A[7]=13};
% Frecce
\draw[arrow] (root) -- (l) node[midway, above left, font=\tiny] {$<7$};
\draw[arrow] (root) -- (r) node[midway, above right, font=\tiny] {$>7$};
\draw[arrow] (l) -- (ll) node[midway, above left, font=\tiny] {$<3$};
\draw[arrow] (l) -- (lr) node[midway, above right, font=\tiny] {$>3$};
\draw[arrow] (r) -- (rl) node[midway, above left, font=\tiny] {$<11$};
\draw[arrow] (r) -- (rr) node[midway, above right, font=\tiny] {$>11$};
% Percorso
\draw[arrow_path] (root) -- (r);
\draw[arrow_path] (r) -- (rl);
\end{scope}

\end{tikzpicture}
\caption{Visualizzazione dell'algoritmo di ricerca binaria (passi e albero)}
\end{figure}


\newpage
\section*{Legenda Colori Visualizzazioni}
La seguente legenda si applica ai colori pastello usati nelle figure di questo documento.

\begin{itemize}
    \item \fcolorbox{bordergray}{pastelsalmon}{\phantom{spazio}} \textbf{Minimo Corrente / Pointer}: Elemento usato come riferimento (es. `min` attuale).
    \item \fcolorbox{bordergray}{pastelyellow}{\phantom{spazio}} \textbf{Elemento Confrontato}: Elemento in fase di ispezione (es. `A[i]`, `A[q]`).
    \item \fcolorbox{bordergray}{pastelgreen}{\phantom{spazio}} \textbf{Trovato / Nuovo Minimo}: Elemento che soddisfa la condizione (trovato, o un nuovo minimo).
    \item \fcolorbox{bordergray}{gray!20}{\phantom{spazio}} \textbf{Scartato}: Porzione dell'array non più considerata (es. nella ricerca binaria).
    \item \fcolorbox{bordergray}{pastelblue}{\phantom{spazio}} \textbf{Nodo Albero}: Elemento nell'albero di decisione della ricerca binaria.
\end{itemize}

\end{document}