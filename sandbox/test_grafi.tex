
\usepackage{tikz}
\usepackage{pgfplots}

% Impostazioni per TikZ
\usetikzlibrary{
    arrows.meta,     % Per frecce migliori
    positioning,     % Per posizionare i nodi
    shapes.geometric,% Per forme
    calc             % Per calcoli sulle coordinate
}

% Impostazioni per PGFPlots (per il grafico delle funzioni)
\pgfplotsset{compat=1.18}

\begin{document}



\begin{figure}[h!]
\centering
\begin{tikzpicture}[
    cell/.style={rectangle, draw=black, minimum width=1.5cm, minimum height=1cm, font=\Large},
    idx/.style={font=\tiny, below=0.1cm of #1},
    arr/.style={-Straight S[length=3mm], line width=1.5pt},
    code/.style={font=\small\ttfamily, anchor=west}
    ]
    
    % Array
    \node[cell] (c1) at (0, 0) {8};
    \node[cell] (c2) at (1.5, 0) {3};
    \node[cell] (c3) at (3, 0) {5};
    \node[cell] (c4) at (4.5, 0) {2};
    \node[cell] (c5) at (6, 0) {7};
    
    % Indici array
    \node[idx=c1] {A[1]};
    \node[idx=c2] {A[2]};
    \node[idx=c3] {A[3]};
    \node[idx=c4] {A[4]};
    \node[idx=c5] {A[5]};

    % Variabile Min
    \node[cell, fill=blue!10] (min) at (0, -2.5) {?};
    \node[left=0.2cm of min, font=\Large] {$min = $};
    
    % Stato 1: Inizializzazione
    \node[code] (s1) at (3, -1.5) {1. \Procedure{Minimo}{A
\usepackage{tikz}
\usepackage{pgfplots}

% Impostazioni per TikZ
\usetikzlibrary{
    arrows.meta,     % Per frecce migliori
    positioning,     % Per posizionare i nodi
    shapes.geometric,% Per forme
    calc             % Per calcoli sulle coordinate
}

% Impostazioni per PGFPlots (per il grafico delle funzioni)
\pgfplotsset{compat=1.18}

\begin{document}



\begin{figure}[h!]
\centering
\begin{tikzpicture}[
    cell/.style={rectangle, draw=black, minimum width=1.5cm, minimum height=1cm, font=\Large},
    idx/.style={font=\tiny, below=0.1cm of #1},
    arr/.style={-Straight S[length=3mm], line width=1.5pt},
    code/.style={font=\small\ttfamily, anchor=west}
    ]
    
    % Array
    \node[cell] (c1) at (0, 0) {8};
    \node[cell] (c2) at (1.5, 0) {3};
    \node[cell] (c3) at (3, 0) {5};
    \node[cell] (c4) at (4.5, 0) {2};
    \node[cell] (c5) at (6, 0) {7};
    
    % Indici array
    \node[idx=c1] {A[1]};
    \node[idx=c2] {A[2]};
    \node[idx=c3] {A[3]};
    \node[idx=c4] {A[4]};
    \node[idx=c5] {A[5]};

    % Variabile Min
    \node[cell, fill=blue!10] (min) at (0, -2.5) {?};
    \node[left=0.2cm of min, font=\Large] {$min = $};
    
    % Stato 1: Inizializzazione
    \node[code] (s1) at (3, -1.5) {1. \Procedure{Minimo}{A, 5}};
    \node[code] (s2) at (3, -2)   {2. \ \ $min = A[1]$};
    \draw[arr, blue!80!black] (s2.west) |- (min.east);
    \node[cell, fill=blue!10] (min1) at (0, -2.5) {8};
    
    % Stato 2: i = 2
    \node[code] (s3) at (3, -2.5) {3. \ \ \textbf{for} $i=2 \to 5$};
    \node[code] (s4) at (3, -3)   {4. \ \ \ \ \textbf{if} $A[2] < min$ (3 < 8) $\to$ \textbf{true}};
    \node[code] (s5) at (3, -3.5) {5. \ \ \ \ \ \ $min = A[2]$};
    \draw[arr, green!60!black] (s3.west) .. controls +(west:1) and +(north:1) .. (c2.north);
    \node[above=0.2cm of c2, font=\Large\bfseries] {$i$};
    \draw[arr, blue!80!black] (s5.west) |- (min.east);
    \node[cell, fill=blue!10] (min2) at (0, -2.5) {3};
    
    % Stato 3: i = 3
    \node[code] (s6) at (3, -4)   {6. \ \ \ \ \textbf{if} $A[3] < min$ (5 < 3) $\to$ \textbf{false}};
    \draw[arr, green!60!black, dashed] (s6.west) .. controls +(west:1) and +(north:1) .. (c3.north);
    
    % Stato 4: i = 4
    \node[code] (s7) at (3, -4.5) {7. \ \ \ \ \textbf{if} $A[4] < min$ (2 < 3) $\to$ \textbf{true}};
    \node[code] (s8) at (3, -5)   {8. \ \ \ \ \ \ $min = A[4]$};
    \draw[arr, green!60!black] (s7.west) .. controls +(west:1) and +(north:1) .. (c4.north);
    \draw[arr, blue!80!black] (s8.west) |- (min.east);
    \node[cell, fill=blue!10] (min3) at (0, -2.5) {2};

    % Stato 5: i = 5
    \node[code] (s9) at (3, -5.5) {9. \ \ \ \ \textbf{if} $A[5] < min$ (7 < 2) $\to$ \textbf{false}};
    \draw[arr, green!60!black, dashed] (s9.west) .. controls +(west:1) and +(north:1) .. (c5.north);
    
    % Stato 6: Ritorno
    \node[code] (s10) at (3, -6) {10.\ \ \textbf{end for}};
    \node[code] (s11) at (3, -6.5) {11.\ \textbf{Return} $min$};
    \node[cell, draw=red, line width=1.5pt] at (0, -2.5) {2};
    \draw[arr, red] (s11.west) |- (min.east);

\end{tikzpicture}
\caption{Tracciamento dell'algoritmo \texttt{Minimo}. La variabile \texttt{min} (blu) viene aggiornata solo quando \texttt{A[i]} (puntato dalla freccia verde) è più piccolo.}
\label{fig:minimo}
\end{figure}

\end{document}


, 5}};
    \node[code] (s2) at (3, -2)   {2. \ \ $min = A[1]$};
    \draw[arr, blue!80!black] (s2.west) |- (min.east);
    \node[cell, fill=blue!10] (min1) at (0, -2.5) {8};
    
    % Stato 2: i = 2
    \node[code] (s3) at (3, -2.5) {3. \ \ \textbf{for} $i=2 \to 5$};
    \node[code] (s4) at (3, -3)   {4. \ \ \ \ \textbf{if} $A[2] < min$ (3 < 8) $\to$ \textbf{true}};
    \node[code] (s5) at (3, -3.5) {5. \ \ \ \ \ \ $min = A[2]$};
    \draw[arr, green!60!black] (s3.west) .. controls +(west:1) and +(north:1) .. (c2.north);
    \node[above=0.2cm of c2, font=\Large\bfseries] {$i$};
    \draw[arr, blue!80!black] (s5.west) |- (min.east);
    \node[cell, fill=blue!10] (min2) at (0, -2.5) {3};
    
    % Stato 3: i = 3
    \node[code] (s6) at (3, -4)   {6. \ \ \ \ \textbf{if} $A[3] < min$ (5 < 3) $\to$ \textbf{false}};
    \draw[arr, green!60!black, dashed] (s6.west) .. controls +(west:1) and +(north:1) .. (c3.north);
    
    % Stato 4: i = 4
    \node[code] (s7) at (3, -4.5) {7. \ \ \ \ \textbf{if} $A[4] < min$ (2 < 3) $\to$ \textbf{true}};
    \node[code] (s8) at (3, -5)   {8. \ \ \ \ \ \ $min = A[4]$};
    \draw[arr, green!60!black] (s7.west) .. controls +(west:1) and +(north:1) .. (c4.north);
    \draw[arr, blue!80!black] (s8.west) |- (min.east);
    \node[cell, fill=blue!10] (min3) at (0, -2.5) {2};

    % Stato 5: i = 5
    \node[code] (s9) at (3, -5.5) {9. \ \ \ \ \textbf{if} $A[5] < min$ (7 < 2) $\to$ \textbf{false}};
    \draw[arr, green!60!black, dashed] (s9.west) .. controls +(west:1) and +(north:1) .. (c5.north);
    
    % Stato 6: Ritorno
    \node[code] (s10) at (3, -6) {10.\ \ \textbf{end for}};
    \node[code] (s11) at (3, -6.5) {11.\ \textbf{Return} $min$};
    \node[cell, draw=red, line width=1.5pt] at (0, -2.5) {2};
    \draw[arr, red] (s11.west) |- (min.east);

\end{tikzpicture}
\caption{Tracciamento dell'algoritmo \texttt{Minimo}. La variabile \texttt{min} (blu) viene aggiornata solo quando \texttt{A[i]} (puntato dalla freccia verde) è più piccolo.}
\label{fig:minimo}
\end{figure}

\end{document}


