% ===================================================================
% PREAMBOLO MIGLIORATO (A.A. 2024-2025)
% (Come da te fornito)
% ===================================================================
\documentclass[12pt, a4paper]{article}

% ---- 1. IMPOSTAZIONI FONDAMENTALI (Lingua, Encoding, Font) ----
\usepackage[utf8]{inputenc}
\usepackage[T1]{fontenc}
\usepackage[english, italian]{babel}
\usepackage{lmodern}

% ---- 2. LAYOUT E GEOMETRIA ----
\usepackage[a4paper,top=2cm,bottom=2cm,left=3cm,right=3cm,marginparwidth=1.75cm]{geometry}

% ---- 3. PACCHETTI MATEMATICI E SIMBOLI ----
\usepackage{amsmath}
\usepackage{amssymb}
\usepackage{amsfonts}

% ---- 4. GRAFICA, FIGURE E COLORI ----
\usepackage{graphicx}
\graphicspath{ {./images/} } % NOTA: Le immagini 'example-image-*' sono fornite da 'mwe'
\usepackage{caption}
\usepackage{subcaption}
\usepackage{wrapfig}
\usepackage[dvipsnames]{xcolor}

% ---- 5. LINK E RIFERIMENTI (MIGLIORAMENTO GRAFICO) ----
\usepackage[pdfusetitle, bookmarksopen=true, bookmarksnumbered=true]{hyperref}
\hypersetup{
    colorlinks=true,
    linkcolor=RoyalBlue,
    citecolor=OliveGreen,
    urlcolor=NavyBlue,
    filecolor=Magenta,
    linktoc=all
}

% ---- 6. HEADER E FOOTER (Stile più leggero) ----
\usepackage{fancyhdr}
\pagestyle{fancy}
\fancyhf{}
\fancyhead[LE,RO]{\small A.A 2024-2025}
\fancyhead[RE,LO]{\small Programmazione ed Algoritmi}
\fancyfoot[RE,LO]{\small \rightmark} % \rightmark = Sezione corrente
\fancyfoot[LE,RO]{\small \thepage}
\renewcommand{\headrulewidth}{0.4pt}
\renewcommand{\footrulewidth}{0.4pt}

% ---- 7. AMBIENTI TEOREMA (MIGLIORAMENTO GRAFICO) ----
\usepackage{amsthm}
\usepackage{mdframed} % Per creare box attorno agli ambienti

% Definizioni di base
\theoremstyle{plain}
\newtheorem{theorem}{Teorema}[section]

\theoremstyle{definition}
\newtheorem{definition}{Definizione}[section]
\newtheorem{example}{Esempio}[section]
\newtheorem{observation}{Osservazione}[subsection]
\newtheorem{dimostrazione}{Dimostrazione}[section]
\newtheorem{note}{Note}[subsection]

% Applichiamo uno stile 'boxed'
\mdfsetup{
    linewidth=1pt,
    linecolor=gray!40,
    backgroundcolor=gray!5,
    roundcorner=5pt,
    skipabove=\topsep,
    skipbelow=\topsep
}
\surroundwithmdframed[style=mddefault]{definition}
\surroundwithmdframed[style=mddefault]{example}
\surroundwithmdframed[style=mddefault]{observation}

% ---- 8. CODICE (MIGLIORAMENTO GRAFICO) ----
\usepackage{listings}

% Definiamo colori
\definecolor{codegray}{rgb}{0.97,0.97,0.97}
\definecolor{commentpurple}{rgb}{0.5,0,0.5}
\definecolor{stringred}{rgb}{0.8,0,0}

\lstdefinelanguage{JavaScript}{
  keywords={typeof, new, true, false, catch, function, return, null, switch, var, let, ref, if, in, while, do, else, case, break},
  keywordstyle=\color{RoyalBlue}\bfseries,
  ndkeywords={class, export, boolean, throw, implements, import, this, print},
  ndkeywordstyle=\color{OliveGreen}\bfseries,
  identifierstyle=\color{black},
  sensitive=false,
  comment=[l]{//},
  morecomment=[s]{/*}{*/},
  commentstyle=\color{commentpurple}\ttfamily,
  stringstyle=\color{stringred}\ttfamily,
  morestring=[b]',
  morestring=[b]"
}

\lstset{
    language=JavaScript,
    backgroundcolor=\color{codegray},
    extendedchars=true,
    basicstyle=\footnotesize\ttfamily,
    showstringspaces=false,
    showspaces=false,
    numbers=left,
    numberstyle=\tiny\color{gray},
    numbersep=5pt,
    tabsize=2,
    breaklines=true,
    showtabs=false,
    captionpos=b,
    frame=tb,
    framesep=3pt,
    framerule=0.4pt,
    aboveskip=1.5em,
    belowskip=1.5em
}








% ---- 7. AMBIENTI TEOREMA (MIGLIORAMENTO GRAFICO) ----
\usepackage{amsthm}
\usepackage{mdframed} % Per creare box attorno agli ambienti

% Definizioni di base (come le avevi)
\theoremstyle{plain}
\newtheorem{theorem}{Teorema}[section]

\theoremstyle{definition}
\newtheorem{definition}{Definizione}[section]
\newtheorem{example}{Esempio}[section]
\newtheorem{observation}{Osservazione}[subsection]
\newtheorem{dimostrazione}{Dimostrazione}[section]
\newtheorem{note}{Note}[subsection]

% --- STILE BOX ISOLATO ---
% 1. Definiamo uno STILE SPECIFICO chiamato 'boxTeorema'.
%    Queste impostazioni (sfondo, bordi, etc.) sono ora
%    isolate e valgono SOLO per questo stile.
\mdfdefinestyle{boxTeorema}{
    linewidth=1pt,
    linecolor=gray!40,
    backgroundcolor=gray!5,
    roundcorner=5pt,
    skipabove=\topsep,
    skipbelow=\topsep
}

% 2. Applichiamo QUELLO stile specifico (style=boxTeorema)
%    invece dello stile di default.
\surroundwithmdframed[style=boxTeorema]{definition}
\surroundwithmdframed[style=boxTeorema]{example}
\surroundwithmdframed[style=boxTeorema]{observation}




% ---- 9. ALTRI PACCHETTI UTILI ----
\usepackage{multicol}
\usepackage{soul}
\usepackage{enumitem}
\usepackage{lipsum} % <-- AGGIUNTO per testo fittizio
\usepackage{mwe}    % <-- AGGIUNTO per immagini di esempio (es. example-image-a)

% ---- TITOLO E AUTORE ----
\title{\textbf{Foglio di Stile (Sarto) \\ Programmazione ed Algoritmi}}
\author{Realizzato da: Joseph Zucchelli}
\date{A.A 2024-2025}

% ===================================================================
% INIZIO DOCUMENTO DI ESEMPIO
% ===================================================================
\begin{document}

\maketitle
\newpage
\tableofcontents

\newpage

% ===================================================================
% ESEMPI DI STILE
% ===================================================================

\section{Stile dei Titoli e Testo}
\label{sec:testo}
Questa sezione mostra lo stile del testo base (font, dimensione 12pt) e la gerarchia dei titoli. L'header e il footer (definiti con `fancyhdr`) sono visibili su questa pagina.

\lipsum[1]

\subsection{Sottosezione di Esempio}
Questo è un titolo di secondo livello. \lipsum[2]

\subsubsection{Sottoparagrafo}
Questo è un titolo di terzo livello. \lipsum[3]

\paragraph{Paragrafo}
Questo è un titolo di quarto livello, solitamente inline. \lipsum[4]

\section{Stili del Testo, Link e Liste}
\lipsum[5]

\subsection{Evidenziazione e Link}
Questo testo mostra lo stile \textbf{grassetto}, \textit{corsivo} e \hl{evidenziato (con soul)}.

Qui ci sono i link (come definiti da \texttt{hypersetup}):
\begin{itemize}
    \item \textbf{Link interno (RoyalBlue):} Vedi la Sezione \ref{sec:testo}.
    \item \textbf{Link web (NavyBlue):} Visita \url{https://www.google.com}.
    \item \textbf{Citazione (OliveGreen):} Secondo \cite{esempio}.
\end{itemize}

\subsection{Liste e Colonne}
Questo è un esempio di lista non ordinata (gestita da \texttt{enumitem}):
\begin{itemize}
    \item Primo elemento.
    \item Secondo elemento con sotto-lista:
    \begin{itemize}
        \item Sotto-elemento A.
        \item Sotto-elemento B.
    \end{itemize}
\end{itemize}

Questo testo è formattato su due colonne utilizzando \texttt{multicol}:
\begin{multicols}{2}
\lipsum[6]
\end{multicols}


\section{Ambienti Teorema e Box}
Questa sezione mostra gli ambienti \texttt{amsthm} personalizzati.

\subsection{Ambienti con Box (mdframed)}
Gli ambienti seguenti sono stati incorniciati utilizzando \texttt{mdframed}.

\begin{definition}[Concetto Chiave]
Questa è una definizione. Ha uno sfondo grigio chiaro, bordi arrotondati e una linea grigia sottile.
\end{definition}

\lipsum[7]

\begin{example}[Calcolo]
Questo è un esempio. Usa lo stesso stile della definizione per coerenza visiva.
\end{example}

\begin{observation}
Questa è un'osservazione. Anch'essa è incorniciata per metterla in evidenza.
\end{observation}

\subsection{Ambienti Standard}
Gli ambienti seguenti usano lo stile \texttt{amsthm} di base.

\begin{theorem}[di Pitagora]
In un triangolo rettangolo, $a^2 + b^2 = c^2$.
\end{theorem}

\begin{dimostrazione}
Questa è una dimostrazione (definita con \texttt{theoremstyle\{definition\}}), quindi appare con testo dritto e titolo in grassetto, ma senza box.
\lipsum[8]
\end{em} % \end{dimostrazione} non è un ambiente standard, ma \em per \it
% Nota: Se 'dimostrazione' dovesse avere un \qed (quadratino),
% andrebbe usato \begin{proof}...\end{proof} da amsthm e
% rinominato con \renewcommand{\proofname}{Dimostrazione}
\begin{proof}[Dimostrazione (con amsthm)]
    Questa è una dimostrazione che usa l'ambiente \texttt{proof} standard di \texttt{amsthm}, che include automaticamente il simbolo Q.E.D.
\end{proof}


\section{Codice (Listings)}
Questo è un blocco di codice formattato con \texttt{listings} secondo le tue specifiche (linguaggio JavaScript, colori personalizzati, sfondo grigio, numeri di riga e frame superiore/inferiore).

\begin{lstlisting}[language=JavaScript, caption={Esempio di codice JavaScript formattato.}, label={lst:js}]
// Questo è un commento
function saluta(nome) {
  var saluto = "Ciao, " + nome;
  let test = true;
  
  if (test) {
    console.log(saluto); // Stampa su console
  }
  
  /* Commento
     multilinea */
  return saluto;
}

var utente = 'Mondo';
saluta(utente);
\end{lstlisting}

\lipsum[9]

\section{Grafica e Figure}
\label{sec:figure}
Qui mostriamo come appaiono le figure, le didascalie e il testo a capo. Le immagini usate sono degli esempi forniti dal pacchetto \texttt{mwe}.

\begin{wrapfigure}{r}{0.4\textwidth}
    \vspace{-15pt} % Aggiustamento per allineare meglio
    \centering
    \includegraphics[width=\linewidth]{example-image-b}
    \caption{Una figura con testo a capo (\texttt{wrapfig}).}
    \label{fig:wrap}
\end{wrapfigure}

Questo è un testo che scorre attorno a un'immagine posizionata sulla destra. Stiamo usando il pacchetto \texttt{wrapfig} per ottenere questo effetto. \lipsum[10]

\begin{figure}[h!]
    \centering
    \includegraphics[width=0.6\textwidth]{example-image-a}
    \caption{Una figura standard centrata.}
    \label{fig:standard}
\end{figure}

Infine, un esempio di sottofigure affiancate utilizzando \texttt{subcaption}.

\begin{figure}[h!]
    \centering
    \begin{subfigure}{0.45\textwidth}
        \includegraphics[width=\linewidth]{example-image-c}
        \caption{Prima sottofigura.}
        \label{fig:sub1}
    \end{subfigure}
    \hfill % Spazio orizzontale flessibile
    \begin{subfigure}{0.45\textwidth}
        \includegraphics[width=\linewidth]{example-image-a}
        \caption{Seconda sottofigura.}
        \label{fig:sub2}
    \end{subfigure}
    \caption{Una figura composta da due sottofigure.}
    \label{fig:subs}
\end{figure}

\section{Elementi Matematici}
Testo con elementi matematici inline, come $E = mc^2$, utilizzando l'ambiente standard. Simboli da \texttt{amssymb} e \texttt{amsfonts}: $\mathbb{R}$, $\mathcal{P}(A)$, $\forall x \in \mathbb{N}$.

Equazioni complesse vengono gestite con \texttt{amsmath}:
\begin{align}
    \int_0^\infty e^{-x^2} dx &= \frac{\sqrt{\pi}}{2} \label{eq:1} \\
    \sum_{n=1}^\infty \frac{1}{n^2} &= \frac{\pi^2}{6} \label{eq:2}
\end{align}


% ---- Bibliografia fittizia per l'esempio \cite ----
\begin{thebibliography}{9}
    \bibitem{esempio}
    A. Autore, \textit{Un libro di esempio per la citazione}, Casa Editrice, 2024.
\end{thebibliography}

\end{document}