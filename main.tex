% ===================================================================
% PREAMBOLO MIGLIORATO (A.A. 2024-2025)
% (Versione corretta e unificata)
% ===================================================================
\documentclass[12pt, a4paper]{article}

% ---- 1. IMPOSTAZIONI FONDAMENTALI (Lingua, Encoding, Font) ----
\usepackage[utf8]{inputenc}
\usepackage[T1]{fontenc}
\usepackage[english, italian]{babel}
\usepackage{lmodern}

% ---- 2. LAYOUT E GEOMETRIA ----
\usepackage[a4paper,top=2cm,bottom=2cm,left=3cm,right=3cm,marginparwidth=1.75cm]{geometry}

% ---- 3. PACCHETTI MATEMATICI E SIMBOLI ----
\usepackage{amsmath}
\usepackage{amssymb}
\usepackage{amsfonts}

% ---- 4. GRAFICA, FIGURE E COLORI ----
\usepackage{graphicx}
\graphicspath{ {./images/} }
\usepackage{caption}
\usepackage{subcaption}
\usepackage{wrapfig}
\usepackage[dvipsnames,svgnames]{xcolor} 
\usepackage{tikz} %%% AGGIUNTO: Pacchetto fondamentale per i grafici
\usetikzlibrary{arrows.meta, positioning, shapes.geometric, calc, decorations.pathmorphing, snakes} %%% AGGIUNTO: Librerie per i grafici
\usepackage{tikz}
\usetikzlibrary{arrows.meta, positioning, shapes.geometric, calc, decorations.pathmorphing, snakes}

% ---- 5. LINK E RIFERIMENTI (MIGLIORAMENTO GRAFICO) ----
\usepackage[pdfusetitle, bookmarksopen=true, bookmarksnumbered=true]{hyperref}
\hypersetup{
    colorlinks=true,
    linkcolor=RoyalBlue,
    citecolor=OliveGreen,
    urlcolor=NavyBlue,
    filecolor=Magenta,
    linktoc=all
}

% ---- 6. HEADER E FOOTER (Stile più leggero) ----
\usepackage{fancyhdr}
\pagestyle{fancy}
\fancyhf{}
\fancyhead[LE,RO]{\small A.A 2024-2025}
\fancyhead[RE,LO]{\small Programmazione ed Algoritmi}
\fancyfoot[RE,LO]{\small \rightmark} 
\fancyfoot[LE,RO]{\small \thepage}
\renewcommand{\headrulewidth}{0.4pt}
\renewcommand{\footrulewidth}{0.4pt}
\setlength{\headheight}{14pt} %%% AGGIUNTO: Corregge l'avviso "headheight is too small"

% ---- 7. AMBIENTI TEOREMA (STILE ISOLATO) ----
\usepackage{amsthm}
\usepackage{mdframed} 

% Definizioni di base
\theoremstyle{plain}
\newtheorem{theorem}{Teorema}[section]
\theoremstyle{definition}
\newtheorem{definition}{Definizione}[section]
\newtheorem{example}{Esempio}[section]
\newtheorem{observation}{Osservazione}[subsection]
\newtheorem{dimostrazione}{Dimostrazione}[section]
\newtheorem{note}{Note}[subsection]

% --- STILE BOX ISOLATO ---
\mdfdefinestyle{boxTeorema}{
    linewidth=1pt,
    linecolor=gray!40,
    backgroundcolor=gray!5,
    roundcorner=5pt,
    skipabove=\topsep,
    skipbelow=\topsep
}
\surroundwithmdframed[style=boxTeorema]{definition}
\surroundwithmdframed[style=boxTeorema]{example}
\surroundwithmdframed[style=boxTeorema]{observation}

% ---- 8. CODICE (LISTINGS) ----
\usepackage{listings}

\definecolor{codegray}{rgb}{0.97,0.97,0.97}
\definecolor{commentpurple}{rgb}{0.5,0,0.5}
\definecolor{stringred}{rgb}{0.8,0,0}
\lstset{ 
    language=javascript, 
    backgroundcolor=\color{codegray}, 
    basicstyle=\footnotesize\ttfamily,
    numbers=left, numberstyle=\tiny\color{gray},
    frame=tb, framerule=0.4pt,
    
}

% ---- 9. ALTRI PACCHETTI UTILI ----
\usepackage{multicol} 
\usepackage{soul}     
\usepackage{enumitem} 
\usepackage{lipsum}   
\usepackage{mwe}      

% ---- 10. PACCHETTI E STILE PSEUDOCODICE (Stile Box Automatico) ----
\usepackage{algorithm}     
\usepackage{algpseudocode} 
\usepackage{etoolbox}      

% --- Stile Box (Grayscale) per Algoritmi ---
% 1. Definiamo uno STILE SPECIFICO 'algoBoxStyle' (scala di grigi).
\mdfdefinestyle{algoBoxStyle}{
    linewidth=1pt,
    linecolor=gray!60,      % Bordo grigio medio
    backgroundcolor=gray!8, % Sfondo grigio chiarissimo
    roundcorner=5pt,
    skipabove=\topsep,
    skipbelow=\topsep,
    nobreak=true            % Tenta di non spezzare il blocco tra le pagine
}

% 2. Applichiamo automaticamente il box a TUTTI gli ambienti 'algorithmic'

\BeforeBeginEnvironment{algorithmic}{%
  \begin{mdframed}[style=algoBoxStyle]%
}
%    Rileva \end{algorithmic} e chiude il box.
\AfterEndEnvironment{algorithmic}{%
  \end{mdframed}%
}

% --- Miglioramento Stile Interno (algpseudocode) ---

% 1. Commenti: grigi, font teletype (tt), corsivo, con simbolo
\algrenewcommand\algorithmiccomment[1]{\color{gray!80}\ttfamily\itshape\(\triangleright\) #1}

% 2. Numeri di riga: piccoli e grigi, per essere meno invadenti
\algrenewcommand\alglinenumber[1]{\tiny\color{gray!80}#1:}

% ===================================================================
% INIZIO DOCUMENTO PRINCIPALE
% ===================================================================
\begin{document}

% --- TITOLO E INDICE (Generati dal Main) ---
\title{\textbf{Appunti di Programmazione ed Algoritmi}}
\author{Realizzato da: Joseph Zucchelli}
\date{A.A 2024-2025}
\maketitle
\tableofcontents

\newpage

% ---- Inclusione dei capitoli/lezioni ----
% Se hai spostato i file in una cartella 'lezioni', usa 'lezioni/Lezione1'
% Altrimenti lascia 'Lezione1' se sono nella stessa cartella.

% ===================================================================
% FILE: Lezione1.tex
% ===================================================================

\part{Lezione 1(13/10/2025) }

\section{Definizione di Algoritmo}
Un algoritmo è una sequenza finita di operazioni elementari (passi) , univocamente determinata (non ambiguo) , che, se eseguita su un calcolatore, porta alla risoluzione di un problema . [cite: 160]
\subsection{Modello RAM (Random Access Machine) }
Nel modello RAM, si assume che le seguenti operazioni elementari abbiano costo "unitario" (costante) : [cite: 161]
\begin{itemize}
    \item \textbf{Operazioni aritmetiche}: +, -, *, /, \% 
    \item \textbf{Operazioni di confronto}: $<, >, ==, !=$ 
    \item \textbf{Operazioni logiche}: AND, OR, NOT 
    \item \textbf{Operazioni di trasferimento}: load/store/assegnamento 
    \item \textbf{Operazioni di controllo}: chiamata di funzione, RETURN 
\end{itemize}

\section{Analisi di Complessità}
Si analizza il costo computazionale (Tempo o Spazio ) in funzione della dimensione dell'input, $n$ . [cite: 162]
\begin{itemize}
    \item \textbf{Complessità in Tempo $T(n)$}: Numero di operazioni elementari eseguite . [cite: 163]
    \item \textbf{Complessità in Spazio $S(n)$}: Numero di celle di memoria utilizzate (oltre a quelle dell'input) . [cite: 164]
\end{itemize}

Ci si concentra sull' \textbf{ordine di grandezza} della funzione $T(n)$, ignorando costanti moltiplicative e termini di ordine inferiore . [cite: 165]
Ad esempio, $T(n) = 3n + 2$ e $T(n) = 5n + \log n + 4$ sono entrambe considerate di complessità \textbf{Lineare} . [cite: 166] $T(n) = 8n^2$ è \textbf{Quadratica} . [cite: 166]

\subsection{Caso Ottimo, Pessimo, Medio}
\begin{itemize}
    \item \textbf{Caso Ottimo} : L'istanza di input che richiede il minor tempo. [cite: 167]
    \item \textbf{Caso Pessimo} : L'istanza di input che richiede il maggior tempo. [cite: 168]
    \item \textbf{Caso Medio}: Complessità media su tutte le possibili istanze. [cite: 169]
\end{itemize}
Ci si concentra sul \textbf{caso pessimo} perché fornisce un limite superiore al costo: l'algoritmo non impiegherà mai più di $T(n)$ . [cite: 170]
\section{Esempio 1: Minimo in Vettore}
\begin{itemize}
    \item \textbf{Input}: Array $A[1..n]$ di interi . [cite: 171]
    \item \textbf{Output}: Il valore minimo contenuto in $A$ . [cite: 171]
\end{itemize}

% Questo ambiente ora funziona grazie ai pacchetti in main.tex
\begin{algorithmic}[1]
\Procedure{Minimo}{A, n} 
    \State $min = A[1]$ \Comment{Costo costante $c_1$ }
    \For{$i = 2 \to n$} \Comment{Eseguito $n-1$ volte }
        \If{$A[i] < min$} \Comment{Costo $c_2$ }
            \State $min = A[i]$ \Comment{Costo $c_3$ }
        \EndIf
    \EndFor
    \State \Return $min$ \Comment{Costo costante $c_4$ }
\EndProcedure
\end{algorithmic}

\textbf{Analisi:}
Il costo totale è $T(n) = c_1 + (n-1)(c_2 \text{ (confronto)} + c_3 \text{ (assegn. caso pessimo)}) + c_4$ . [cite: 172]
$T(n) = c'n + b$. [cite: 173]
La complessità è \textbf{Lineare} , $T(n) \in \Theta(n)$, sia nel caso ottimo che in quello pessimo . [cite: 174]
\section{Esempio 2: Cerca K }
\begin{itemize}
    \item \textbf{Input}: Array $A[1..n]$ di interi, $k$ intero . [cite: 175]
    \item \textbf{Output}: $i$ tale che $A[i]=k$ , o $-1$ se $k \notin A$ . [cite: 176]
\end{itemize}

\begin{algorithmic}[1]
\Procedure{Cerca-K}{A, n, k} 
    \State $i = 1$ 
    \State $trovato = \text{false}$ 
    \While{(\textbf{not} $trovato$) \textbf{and} ($i \le n$)} 
        \If{$A[i] == k$}
            \State $trovato = \text{true}$ 
        \Else
            \State $i = i + 1$ 
        \EndIf
    \EndWhile
    \If{$trovato$}
        \State \Return $i$ [cite: 177]
    \Else
        \State \Return $-1$ 
    \EndIf
\EndProcedure
\end{algorithmic}

\textbf{Analisi:}
\begin{itemize}
    \item \textbf{Caso Ottimo}: $k = A[1]$ . Il ciclo `while` esegue 1 iterazione. $T(n) \in \Theta(1)$ (Costante) . [cite: 178]
    \item \textbf{Caso Pessimo}: $k \notin A$ (o $k=A[n]$) . Il ciclo `while` esegue $n$ iterazioni. $T(n) \in \Theta(n)$ (Lineare) . [cite: 179, 180]
\end{itemize}

\section{Esempio 3: Minimo in Vettore Ordinato }
\begin{itemize}
    \item \textbf{Input}: Array $A[1..n]$ di interi, \textbf{ordinato} .
    \item \textbf{Output}: Il valore minimo contenuto in $A$ .
\end{itemize}

\begin{algorithmic}[1]
\Procedure{Minimo-Ordinato}{A, n} 
    \State \Return $A[1]$ 
\EndProcedure
\end{algorithmic}
\textbf{Analisi}: $T(n) \in \Theta(1)$ (Costante) . [cite: 182]
\section{Esempio 4: Cerca K in Vettore Ordinato (Ricerca Binaria) }
\begin{itemize}
    \item \textbf{Input}: Array $A[1..n]$ di interi \textbf{ordinato}, $k$ intero . [cite: 183]
    \item \textbf{Output}: $i$ tale che $A[i]=k$ , o $-1$ se $k \notin A$ . [cite: 184]
\end{itemize}
L'idea è di confrontare $k$ con l'elemento centrale $A[q]$ e dimezzare lo spazio di ricerca . [cite: 185]
\begin{algorithmic}[1]
\Procedure{BS-IT}{A, p, r, k} 
    \If{($k < A[p]$) \textbf{or} ($k > A[r]$)} \Comment{Controllo opzionale}
        \State \Return $-1$ 
    \EndIf
    \While{$p \le r$} 
        \State $q = \lfloor (p+r)/2 \rfloor$ 
        \If{$A[q] == k$}
            \State \Return $q$ 
        \ElsIf{$A[q] > k$}
            \State $r = q - 1$ [cite: 186]
        \Else
            \State $p = q + 1$ 
        \EndIf
    \EndWhile
    \State \Return $-1$ 
\EndProcedure
\end{algorithmic}

\textbf{Analisi:}
\begin{itemize}
    \item \textbf{Caso Ottimo}: $k = A[q]$ al primo ciclo. $T(n) \in \Theta(1)$ (Costante) . [cite: 187]
    \item \textbf{Caso Pessimo}: $k \notin A$. Il numero di iterazioni è $\log_2 n$ . $T(n) \in \Theta(\log n)$ (Logaritmica) . [cite: 188]
\end{itemize}

\newpage
% ===================================================================
% FILE: Precizazione0.tex
% ===================================================================
\part*{Precizazione}
\paragraph{Guida alla Complessità Computazionale (Notazione O-Grande)}
La complessità computazionale è un modo per descrivere l'efficienza di un algoritmo.
Non misura il tempo esatto in secondi, ma stima come il numero di operazioni (tempo) o l'uso della memoria (spazio) cresce all'aumentare della dimensione dell'input (indicato con $n$).
La notazione O-grande si concentra sull'ordine di grandezza asintotico, ignorando le costanti moltiplicative e i termini di ordine inferiore.
\begin{definition}[Ordine Asintotico]
L'ordine asintotico descrive come si comporta il tempo (o lo spazio) richiesto da un algoritmo quando la dimensione dell'input ($n$) diventa estremamente grande.
È come guardare la "forma" generale della curva di crescita da molto lontano.

Si ignorano i dettagli iniziali e la ripidità iniziale della curva, per concentrarci unicamente sul termine che cresce più velocemente, visto che sarà il più impattante quando $n$ sarà enorme.
Ad esempio, se un algoritmo impiega $3n^2 + 10n + 5$ operazioni, il suo ordine asintotico è $O(n^2)$.

Perché?
Perché quando $n$ diventa grandissimo (es. un milione), il termine $n^2$ è talmente più grande di $n$ e di $5$ che gli altri diventano irrilevanti per capire l'andamento generale.
\end{definition}

\section{Differenza Chiave: $O(n)$ (Lineare) vs. $O(\log n)$ (Logaritmico)}
Spesso si crea confusione tra un costo come $n/2$ e uno come $\log n$, ma appartengono a due universi di efficienza completamente diversi.
\begin{itemize}
    \item \textbf{Lineare $O(n)$:} Un algoritmo con un costo proporzionale a $n$ (come $n$, $n/2$ o $2n$) ha una complessità Lineare, $O(n)$.
    Questo significa che il numero di operazioni è direttamente proporzionale alla dimensione dell'input.
    Se l'input raddoppia, anche il tempo di esecuzione (circa) raddoppia. Nella notazione O-grande, le costanti (come $1/2$) vengono ignorate.
    Ad esempio, la ricerca lineare in un array non ordinato richiede, nel caso medio, $n/2$ controlli.
    La sua complessità è comunque $O(n)$.
    
    \item \textbf{Logaritmico $O(\log n)$:} Un algoritmo con costo $\log n$ (logaritmo in base 2, $\log_2 n$) ha una complessità Logaritmica, $O(\log n)$.
    Questo tipo di algoritmo è estremamente efficiente perché, ad ogni passo, è in grado di scartare una frazione significativa del problema (di solito la metà).
    L'esempio classico è la ricerca binaria.
\end{itemize}

\begin{example}[Confronto Pratico: $O(n)$ vs $O(\log n)$]
Su un input di $n = 1.000.000$ di elementi:
\begin{itemize}
    \item Un algoritmo lineare ($n/2$) richiederebbe circa \textbf{500.000} operazioni.
    \item Un algoritmo logaritmico ($\log_2 n$) ne richiederebbe circa \textbf{20}.
\end{itemize}
$O(\log n)$ è drasticamente più veloce di $O(n)$.
\end{example}

\section{Caso Pessimo, Medio e Ottimo}
La performance di un algoritmo può cambiare non solo in base alla dimensione dell'input ($n$), ma anche in base a come è fatto l'input.
\begin{observation}[Caso Pessimo, Medio e Ottimo] \\
\begin{itemize}
    \item \textbf{Caso Pessimo (Worst Case):} Rappresenta lo scenario che richiede il massimo numero di operazioni.
    È l'input "peggiore" possibile. È la metrica più importante e quasi sempre quella che si utilizza, perché fornisce una garanzia sulla performance: l'algoritmo non farà mai peggio di così.
    \item \textbf{Caso Medio (Average Case):} Descrive la performance "tipica" calcolata come media su tutti i possibili input.
    \item \textbf{Caso Ottimo (Best Case):} Descrive lo scenario più veloce in assoluto (ma spesso poco utile, perché si verifica solo in condizioni molto specifiche).
\end{itemize}
\end{observation}

\section{Classi di Complessità (dal più veloce al più lento)}
\begin{example}[Gerarchia delle Complessità]
\begin{description}
    \item[$O(1)$ - Costante:] Il tempo non dipende da $n$. (Es. Accesso a un array `array[i]`).
    \item[$O(\log n)$ - Logaritmico:] Il tempo cresce molto lentamente. (Es. Ricerca binaria).
    \item[$O(n)$ - Lineare:] Il tempo cresce linearmente con $n$. (Es. Un singolo ciclo for, trovare il massimo).
    \item[$O(n \log n)$ - Linearitmico:] Ottima complessità per gli algoritmi di ordinamento. (Es. Merge Sort, Heapsort).
    \item[$O(n^2)$ - Quadratico:] Il tempo cresce con il quadrato di $n$. (Es. Due cicli for annidati, Bubble Sort, Insertion Sort).
    \item[$O(n^k)$ - Polinomiale:] Il tempo cresce con $n$ elevato a una costante $k$. (Es. Tre cicli annidati $O(n^3)$).
    \item[$O(2^n)$ - Esponenziale:] Diventa intrattabile molto rapidamente. (Es. Soluzioni "brute force" che provano tutte le combinazioni).
    \item[$O(n!)$ - Fattoriale:] Il peggior caso possibile. (Es. "Brute force" al problema del commesso viaggiatore).
\end{description}
\end{example}

\section{Regole di Calcolo (Come Combinare)}
Per calcolare la complessità di un programma intero, si combinano i costi delle sue parti usando due regole fondamentali.
\begin{observation}[Regole di Calcolo Asintotico]
\begin{itemize}
    \item \textbf{Regola della Somma (Operazioni in sequenza):} Se hai un blocco A seguito da un blocco B, la complessità totale è $O(A) + O(B)$.
    Si tiene solo il termine dominante.
    Ad esempio, un ciclo $O(n)$ seguito da due cicli annidati $O(n^2)$ ha una complessità totale $O(n) + O(n^2)$, che si semplifica in $O(n^2)$.
    \item \textbf{Regola del Prodotto (Operazioni annidate):} Se un blocco B è all'interno di un blocco A, le complessità si moltiplicano: $O(A) \times O(B)$.
    L'esempio classico è un `for` $O(n)$ che contiene un altro `for` $O(n)$: la complessità totale è $O(n \times n) = O(n^2)$.
\end{itemize}
\end{observation}

\section{Impatto dell'Ordinamento: Array Ordinato vs. Non Ordinato}
Avere un array di input già ordinato (o decidere di ordinarlo) può cambiare drasticamente la complessità.
L'operazione di ordinamento in sé ha un costo, tipicamente $O(n \log n)$.
\begin{example}[Ricerca di un elemento: $O(n)$ vs $O(\log n)$]
\begin{itemize}
    \item \textbf{Non Ordinato:} Ricerca lineare (controllarli uno per uno).
    Caso pessimo: $O(n)$.
    \item \textbf{Ordinato:} Ricerca binaria (dimezzando l'intervallo). Caso pessimo: $O(\log n)$.
\end{itemize}
\end{example}

\begin{example}[Ricerca di duplicati: $O(n^2)$ vs $O(n)$]
\begin{itemize}
    \item \textbf{Non Ordinato:} Confrontare ogni elemento con ogni altro.
    Caso pessimo: $O(n^2)$.
    \item \textbf{Ordinato:} Basta una singola scansione lineare. Se $A[i] == A[i+1]$, esiste un duplicato. Caso pessimo: $O(n)$.
\end{itemize}
\end{example}

\begin{example}[Trovare il minimo o il massimo: $O(n)$ vs $O(1)$]
\begin{itemize}
    \item \textbf{Non Ordinato:} Bisogna scorrere tutto l'array.
    Caso pessimo: $O(n)$.
    \item \textbf{Ordinato:} Il minimo è il primo elemento ($A[0]$) e il massimo è l'ultimo ($A[n-1]$).
    Caso pessimo: $O(1)$.
\end{itemize}
\end{example}

\subsection{Quando ha senso ordinare l'array?}
Ordinare (costo $O(n \log n)$) ha senso quando il costo totale è inferiore a quello dell'operazione sull'array non ordinato.
\begin{example}[Scenario 1: Operazione singola (Trova max)]
\begin{itemize}
    \item \textbf{Costo (Non ordinato):} $O(n)$.
    \item \textbf{Costo (Ordinando prima):} $O(n \log n) \text{ (sort)} + O(1) \text{ (accesso)} = O(n \log n)$.
    \item \textbf{Verdetto:} $O(n)$ è molto meglio. Non ordinare.
\end{itemize}
\end{example}

\begin{example}[Scenario 2: Operazioni multiple (k ricerche)]
Se devi effettuare $k$ ricerche diverse su $n$ elementi.
\begin{itemize}
    \item \textbf{Costo (Non ordinato):} $k$ ricerche lineari $\implies k \times O(n) = O(k \cdot n)$.
    \item \textbf{Costo (Ordinando prima):} $O(n \log n) \text{ (una tantum)} + k \times O(\log n) \implies O(n \log n + k \log n)$.
    \item \textbf{Verdetto:} Se $k$ è grande (es. $k \approx O(n)$), il costo $O(n \log n)$ è drasticamente migliore di $O(n^2)$.
    Ha senso ordinare.
\end{itemize}
\end{example}

\section{Complessità Spaziale (Cenno)}
Oltre al tempo, la complessità spaziale misura quanta memoria ausiliaria (spazio extra oltre all'input) usa l'algoritmo.
Può essere $O(1)$ (costante), se usa solo un numero fisso di variabili, o $O(n)$ (lineare), se ha bisogno di creare una struttura dati (come un array di supporto) grande quanto l'input.
\section{Esempi di Analisi (Linguaggio MAO)}
Analizziamo il costo (caso pessimo) di alcuni frammenti di codice MAO.
% --- NOTA: Rimuovo il \newcommand{\mao} che era rotto e non usato.
% Uso direttamente l'ambiente 'listings' come hai fatto.
\begin{example}[Esempio 1: Ciclo Singolo (Lineare)]
\begin{lstlisting}[language={}, basicstyle=\ttfamily\bfseries, backgroundcolor=\color{codegray}, frame=tb, numbers=none, breaklines=true, tabsize=2]
int i=0;
int s=0;
while (i < n) {
    s := s + i;
    i := i + 1;
}
\end{lstlisting}
\textbf{Analisi ($O(n)$):} Il codice esegue due comandi iniziali $O(1)$. Segue un ciclo `while`.
Il corpo del ciclo ha costo costante $O(1)$. La guardia viene valutata $n+1$ volte e il corpo viene eseguito $n$ volte.
La complessità totale è (Regola della Somma): $O(1) + O(n \times 1)$. Il termine dominante è $O(n)$.
\end{example}

\begin{example}[Esempio 2: Cicli Annidati (Quadratico)]
\begin{lstlisting}[language={}, basicstyle=\ttfamily\bfseries, backgroundcolor=\color{codegray}, frame=tb, numbers=none, breaklines=true, tabsize=2]
int i=0;
int r=0;
while (i < n) {
    int j=0;
while (j < n) {
        r := r + 1;
j := j + 1;
    }
    i := i + 1;
}
\end{lstlisting}
\textbf{Analisi ($O(n^2)$):} Abbiamo due cicli annidati.
Il ciclo esterno (while $i<n$) esegue il suo corpo $n$ volte.
Il corpo contiene un ciclo interno (while $j<n$) che viene eseguito $n$ volte per ogni iterazione esterna.
Per la Regola del Prodotto, la complessità è $O(n \times n) = O(n^2)$.
\end{example}

\begin{example}[Esempio 3: Sequenza di Cicli (Regola della Somma)]
\begin{lstlisting}[language={}, basicstyle=\ttfamily\bfseries, backgroundcolor=\color{codegray}, frame=tb, numbers=none, breaklines=true, tabsize=2]
int s=0;
int i=0;
while (i < n) {
    s := s + i;
    i := i + 1;
}
int j=0;
while (j < n) {
    int k=0;
while (k < n) {
        s := s + 1;
k := k + 1;
    }
    j := j + 1;
}
\end{lstlisting}
\textbf{Analisi ($O(n^2)$):} Questo codice è una sequenza di due blocchi.
\begin{itemize}
    \item Il primo blocco è un ciclo singolo: costo $O(n)$.
    \item Il secondo blocco è composto da due cicli annidati: costo $O(n^2)$.
\end{itemize}
Per la Regola della Somma, la complessità totale è $O(n) + O(n^2)$.
Si considera solo il termine dominante, quindi la complessità è $O(n^2)$.
\end{example}

\begin{example}[Esempio 4: Ciclo Logaritmico]
\begin{lstlisting}[language={}, basicstyle=\ttfamily\bfseries, backgroundcolor=\color{codegray}, frame=tb, numbers=none, breaklines=true, tabsize=2]
int i=1;
while (i < n) {
    skip;
i := i * 2;
}
\end{lstlisting}
\textbf{Analisi ($O(\log n)$):} La variabile di controllo $i$ non viene incrementata linearmente ($i+1$), ma viene moltiplicata per 2. I valori di $i$ saranno $1, 2, 4, 8, 16, ..., 2^k$ fino a superare $n$.
Il numero di iterazioni ($k$) è il più piccolo intero tale che $2^k \ge n$.
Questo $k$ è esattamente $\log_2 n$. Poiché il corpo del ciclo ha costo $O(1)$, la complessità totale è $O(\log n)$.
\end{example}

\begin{example}[Esempio 5: Condizionale nel Caso Pessimo]
\begin{lstlisting}[language={}, basicstyle=\ttfamily\bfseries, backgroundcolor=\color{codegray}, frame=tb, numbers=none, breaklines=true, tabsize=2]
int i=0;
int r=0;
while (i < n) {
    if (i < 10) {
        r := r + 1;
// Costo O(1)
    } else {
        int j=0;
while (j < n) {
            r := r + j;
// Costo O(n)
            j := j + 1;
}
    }
    i := i + 1;
}
\end{lstlisting}
\textbf{Analisi ($O(n^2)$):} Stiamo analizzando il caso pessimo.
Il ciclo esterno (while $i<n$) viene eseguito $n$ volte. All'interno c'è un `if`.
Dobbiamo considerare il costo del ramo più pesante.
\begin{itemize}
    \item Il ramo `if` ($i<10$) ha costo $O(1)$.
    \item Il ramo `else` contiene un ciclo lineare, costo $O(n)$.
\end{itemize}
Nell'analisi del caso pessimo, assumiamo che venga sempre eseguito il ramo più costoso.
Quasi tutte le iterazioni (per $i \ge 10$) eseguiranno il ramo `else`, che costa $O(n)$.
Applicando la Regola del Prodotto, abbiamo il ciclo esterno $O(n)$ che contiene un blocco che (nel caso peggiore) costa $O(n)$.
La complessità totale è $O(n \times n) = O(n^2)$.
\end{example}

\newpage

% ===================================================================
% FILE: Lezione2.tex
% ===================================================================

\part{Lezione 2 (16/10/2025) }

\section{Selection Sort (Analisi) }
Pseudocodice (identico a Lez12).
\begin{algorithmic}[1]
    \Procedure{SelectionSort}{A}
        \For{$i = 1 \to n-1$}
            \State $min = i$
            \For{$j = i+1 \to n$} \Comment{Il loop interno fa $(n-i)$ iterazioni }
                \If{$A[j] < A[min]$}
                    \State $min = j$
                \EndIf
            \EndFor
            \State \Call{Swap}{A[i], A[min]}
        \EndFor
    \EndProcedure
\end{algorithmic}

\begin{explanation}{Selection Sort}
L'algoritmo seleziona iterativamente il minimo dalla parte non ordinata e lo sposta alla fine della parte ordinata.
\begin{itemize}
    \item \textbf{Ciclo Esterno}: Avanza il confine tra ordinato e non ordinato.
    \item \textbf{Ciclo Interno}: Cerca il minimo nel sottoarray destro.
    \item \textbf{Swap}: Scambia il minimo trovato con l'elemento corrente.
\end{itemize}
\end{explanation}

\subsection{Analisi Complessità (Numero Confronti) }
Il costo è dominato dal numero di confronti ($A[j] < A[min]$).
Il ciclo esterno \texttt{for i} esegue $n-1$ iterazioni.
Il ciclo interno \texttt{for j} esegue $n-i$ iterazioni per ogni $i$.
Il numero totale di confronti $C(n)$ è:
\[ C(n) = \sum_{i=1}^{n-1} (n-i) \]
\[ C(n) = (n-1) + (n-2) + \dots + 2 + 1 \]
Questa è la somma dei primi $n-1$ numeri naturali.
\[ C(n) = \frac{(n-1)n}{2} = \frac{n^2}{2} - \frac{n}{2} \]

La complessità è \textbf{Quadratica}, $T(n) \in \Theta(n^2)$.
\begin{observation}
    A differenza di Insertion Sort, la complessità di Selection Sort è $\Theta(n^2)$ \emph{sempre}, sia nel caso ottimo, medio e pessimo, perché i cicli \texttt{for} vengono eseguiti sempre lo stesso numero di volte.
\end{observation}

\subsection{Invariante di Ciclo }

\begin{definition}[Invariante: Selection Sort]
    \textbf{Invariante:} All'inizio dell'iterazione $i$-esima del ciclo FOR esterno (per $i=1..n-1$):
    \begin{enumerate}
        \item Il sottoarray $A[1..i-1]$ contiene gli $i-1$ elementi più piccoli di A.
        \item Il sottoarray $A[1..i-1]$ è ordinato.
    \end{enumerate}
    (Si dimostra per induzione ).
\end{definition}

\section{Esercizi}


\begin{example}[Esercizio 1: Cerca A[i] = i (Array non ordinato)]
    \begin{itemize}
        \item \textbf{Input}: Array $A[1..n]$ di interi.
        \item \textbf{Output}: TRUE se $\exists i$ t.c. $A[i] = i$, FALSE altrimenti.
    \end{itemize}

    \begin{algorithmic}[1]
        \Procedure{Cerca-Indice}{A, n}
            \State $i = 1$
            \State $trovato = \text{false}$
            \While{(\textbf{not} $trovato$) \textbf{and} ($i \le n$)}
                \If{$A[i] == i$}
                    \State $trovato = \text{true}$
                \Else
                    \State $i = i + 1$
                \EndIf
            \EndWhile
            \State \Return $trovato$
        \EndProcedure
    \end{algorithmic}

\begin{explanation}{Ricerca Lineare}
Scansiona l'array elemento per elemento.
Se trova $A[i] == i$, si ferma e ritorna TRUE.
Nel caso pessimo (non trovato), scorre tutto l'array ($n$ passi).
\end{explanation}
    \textbf{Analisi:}
    \begin{itemize}
        \item Caso Ottimo: $A[1]=1$. $T(n) \in \Theta(1)$.
        \item Caso Pessimo: Nessun $i$ t.c. $A[i]=i$. $T(n) \in \Theta(n)$ (Lineare).
    \end{itemize}
\end{example}


\begin{example}[Esercizio 2: Cerca A[i] = i (Array ordinato)]
    \begin{itemize}
        \item \textbf{Input}: Array $A[1..n]$ di interi, \textbf{ordinato}.
        \item \textbf{Output}: TRUE se $\exists i$ t.c. $A[i] = i$.
    \end{itemize}
    Si può usare una modifica della Ricerca Binaria.
    Si calcola $q = \lfloor (p+r)/2 \rfloor$.
    \begin{itemize}
        \item Se $A[q] == q$: Trovato.
        \item Se $A[q] > q$: L'elemento $i$ (se esiste) non può essere a destra di $q$. Si cerca a sinistra ($r=q-1$).
        \item Se $A[q] < q$: L'elemento $i$ (se esiste) non può essere a sinistra di $q$. Si cerca a destra ($p=q+1$).
    \end{itemize}
    \textbf{Analisi}: $T(n) \in O(\log n)$.
\end{example}


\begin{example}[Esercizio 3: Cerca A[i] = i (Ordinato, positivi, distinti)]
    \begin{itemize}
        \item \textbf{Input}: Array $A[1..n]$ ordinato, di interi \textbf{positivi} e \textbf{distinti}.
        \item \textbf{Output}: TRUE se $\exists i$ t.c. $A[i] = i$.
    \end{itemize}
    Se $A[1] = 1$: Ritorna TRUE.
    Se $A[1] > 1$: (cioè $A[1] \ge 2$). Allora $A[i] \ge A[1] + (i-1) \ge 2 + i - 1 = i+1$.
    Quindi $A[i] > i$ per ogni $i$. Ritorna FALSE.
    L'algoritmo corretto è:
    \begin{algorithmic}[1]
        \Procedure{Cerca-i-Positivi}{A}
            \State \Return $(A[1] == 1)$
        \EndProcedure
    \end{algorithmic}
    \textbf{Analisi}: $T(n) \in \Theta(1)$ (Costante).
\end{example}

\begin{example}[Esercizio 4: Somma K]
    \begin{itemize}
        \item \textbf{Input}: Array $A[1..n]$ di interi, $k$ intero.
        \item \textbf{Output}: TRUE se $\exists i, j$ t.c. $A[i] + A[j] = k$.
    \end{itemize}
    \textbf{Soluzione 1 (Brute force):}
    \begin{algorithmic}[1]
        \For{$i = 1 \to n-1$}
            \For{$j = i+1 \to n$}
                \If{$A[i] + A[j] == k$}
                    \State \Return $\text{true}$
                \EndIf
            \EndFor
        \EndFor
        \State \Return $\text{false}$
    \end{algorithmic}
    \textbf{Analisi 1}: Caso pessimo $\Theta(n^2)$ (Quadratico).
    \textbf{Soluzione 2 (se $A$ è ordinato ):}
    Si usano due indici, $L=1$ e $R=n$.
    \begin{algorithmic}[1]
        \State $L=1$, $R=n$
        \While{$L < R$}
            \State $somma = A[L] + A[R]$
            \If{$somma == k$}
                \State \Return $\text{true}$
            \ElsIf{$somma < k$}
                \State $L = L + 1$ \Comment{Serve una somma più grande }
            \Else
                \State $R = R - 1$ \Comment{Serve una somma più piccola}
            \EndIf
        \EndWhile
        \State \Return $\text{false}$
    \end{algorithmic}

\begin{explanation}{Tecnica dei Due Indici}
Poiché l'array è ordinato, possiamo restringere la ricerca:
\begin{itemize}
    \item $Somma < K \to$ Incremento $L$ (serve valore più grande).
    \item $Somma > K \to$ Decremento $R$ (serve valore più piccolo).
\end{itemize}
Questo riduce la complessità da quadratica a lineare.
\end{explanation}
    \textbf{Analisi 2}: $T(n) \in \Theta(n)$ (Lineare).
\end{example}


\begin{example}[Esercizio 5: Array Palindromo]
    \begin{itemize}
        \item \textbf{Input}: Array $A[1..n]$.
        \item \textbf{Output}: TRUE se $A$ è palindromo, FALSE altrimenti. (E.g., \texttt{[3, 7, 21, 40, 21, 7, 3]} ).
    \end{itemize}
    \textbf{Soluzione (con due indici):}
    \begin{algorithmic}[1]
        \State $i=1$, $j=n$
        \While{$i < j$}
            \If{$A[i] \neq A[j]$}
                \State \Return $\text{false}$
            \EndIf
            \State $i = i + 1$
            \State $j = j - 1$
        \EndWhile
        \State \Return $\text{true}$
    \end{algorithmic}

\begin{explanation}{Verifica Palindromo}
Confronta gli estremi convergendo verso il centro.
Se una coppia non corrisponde, non è palindromo.
\end{explanation}
    \textbf{Analisi}: $T(n) \in \Theta(n)$.
\end{example}

\newpage

\input{lezioni/Lezione3}
% ===================================================================
% FILE: Lezione4.tex
% ===================================================================

\part{Lezione 14 (20/10/2025) }

\section{Notazione Asintotica }
La complessità $T(n)$ si esprime in \textbf{ordine di grandezza} , ignorando costanti moltiplicative e termini di ordine inferiore . [cite: 41]
\begin{itemize}
    \item $T(n) = 3n^2 + 2n + 5 \to$ Quadratica ($\Theta(n^2)$) 
    \item $T(n) = 7n + 24 \to$ Lineare ($\Theta(n)$) 
    \item $T(n) = 5 \to$ Costante ($\Theta(1)$) 
    \item $T(n) = \log_3 n + 2 \to$ Logaritmica ($\Theta(\log n)$) 
\end{itemize}
Si usano funzioni di riferimento semplici $g(n)$ (es. $n^2$, $n$, $\log n$) per classificare $f(n) = T(n)$ . [cite: 42]
\section{Notazione $\Theta$ (Theta) - Limite Stretto }
% --- APPLICAZIONE STILE ---
\begin{definition}[Notazione $\Theta$ (Theta)]
$$ \Theta(g(n)) = \{ f(n) \mid \exists c_1, c_2, n_0 > 0 : \forall n \ge n_0, 0 \le c_1 g(n) \le f(n) \le c_2 g(n) \} $$
Si dice "$f(n)$ è in Theta di $g(n)$" .
$g(n)$ è un \textbf{limite asintotico stretto} per $f(n)$ . [cite: 44]
Graficamente, da $n_0$ in poi, $f(n)$ è "intrappolata" tra $c_1 g(n)$ e $c_2 g(n)$ . [cite: 45]
\begin{itemize}
    \item Esempio: Selection Sort è $\Theta(n^2)$ . [cite: 46]
    \item Esempio: $\frac{1}{2}n^2 - 2n \in \Theta(n^2)$ . [cite: 46]
\end{itemize}
\end{definition}

\section{Notazione $O$ (O-grande) - Limite Superiore }
% --- APPLICAZIONE STILE ---
\begin{definition}[Notazione $O$ (O-grande)]
$$ O(g(n)) = \{ f(n) \mid \exists c, n_0 > 0 : \forall n \ge n_0, 0 \le f(n) \le c g(n) \} $$
Si dice "$f(n)$ è in O-grande di $g(n)$" .
$g(n)$ è un \textbf{limite asintotico superiore} per $f(n)$ . [cite: 47]
Graficamente, da $n_0$ in poi, $f(n)$ non cresce più velocemente di $c g(n)$ . [cite: 48]
\begin{itemize}
    \item Esempio: $f(n) = an^2 + bn + c \in O(n^2)$ . [cite: 49]
    \item Esempio: $f(n) = an^2 + bn + c \in O(n^3)$ . [cite: 50]
    \item Esempio: $f(n) = an^2 + bn + c \notin O(n)$ . [cite: 51]
\end{itemize}
Proprietà: $f(n) \in \Theta(g(n)) \implies f(n) \in O(g(n))$ .
\end{definition}

\section{Notazione $\Omega$ (Omega) - Limite Inferiore }
% --- APPLICAZIONE STILE ---
\begin{definition}[Notazione $\Omega$ (Omega)]
$$ \Omega(g(n)) = \{ f(n) \mid \exists c, n_0 > 0 : \forall n \ge n_0, 0 \le c g(n) \le f(n) \} $$
Si dice "$f(n)$ è in Omega di $g(n)$" .
$g(n)$ è un \textbf{limite asintotico inferiore} per $f(n)$ . [cite: 52]
Graficamente, da $n_0$ in poi, $f(n)$ non cresce più lentamente di $c g(n)$ . [cite: 53]
\begin{itemize}
    \item Esempio: $an^2 + bn + c \in \Omega(n^2)$ . [cite: 54]
    \item Esempio: $an^2 + bn + c \in \Omega(n)$ . [cite: 54]
    \item Esempio: $an^2 + bn + c \notin \Omega(n^3)$ . [cite: 55]
\end{itemize}
Proprietà: $f(n) \in \Theta(g(n)) \implies f(n) \in \Omega(g(n))$ .
\end{definition}

\subsection{Teorema}
% --- APPLICAZIONE STILE ---
\begin{theorem}
$$ f(n) \in \Theta(g(n)) \iff f(n) \in O(g(n)) \text{ e } f(n) \in \Omega(g(n)) $$
\end{theorem}

\section{Proprietà e Gerarchia}
% --- APPLICAZIONE STILE ---
\begin{observation}[Proprietà della Notazione Asintotica]
\begin{itemize}
    \item \textbf{Riflessività}: $f(n) \in \Theta(f(n))$, $f(n) \in O(f(n))$, $f(n) \in \Omega(f(n))$ . [cite: 56]
    \item \textbf{Simmetria (Theta)}: $f(n) \in \Theta(g(n)) \iff g(n) \in \Theta(f(n))$ . [cite: 57]
    \item \textbf{Trasposta (O/Omega)}: $f(n) \in O(g(n)) \iff g(n) \in \Omega(f(n))$ . [cite: 57]
    \item \textbf{Transitività}: Vale per $O, \Omega, \Theta$. [cite: 58]
    Es: $f_1 \in O(f_2)$ e $f_2 \in O(f_3) \implies f_1 \in O(f_3)$ . [cite: 59]
    \item \textbf{Somma}: $f_1 \in O(g_1)$ e $f_2 \in O(g_2) \implies f_1+f_2 \in O(\max(g_1, g_2))$ . [cite: 60]
    \item \textbf{Prodotto}: $f_1 \in O(g_1)$ e $f_2 \in O(g_2) \implies f_1 \cdot f_2 \in O(g_1 \cdot g_2)$ . [cite: 61]
\end{itemize}
\end{observation}


\begin{observation}[Equivalenza dei Logaritmi]
Tutte le basi dei logaritmi sono asintoticamente equivalenti . [cite: 62]
Dalla formula del cambio di base: $\log_a n = \frac{\log_b n}{\log_b a}$ . [cite: 63]
Poiché $\frac{1}{\log_b a}$ è una costante , si ha $\Theta(\log_a n) = \Theta(\log_b n)$ . [cite: 64]
Per questo motivo, si scrive genericamente $O(\log n)$ . [cite: 64]
\end{observation}

\subsection{Gerarchia degli ordini di grandezza }

\begin{example}[Gerarchia di Crescita]
Per $0 < h \le k$ e $1 < a < b$ :
$$ \Theta(1) \subset \dots \subset \Theta(\log n) \subset \dots \subset \Theta(n^h) \subset \Theta(n^k) \subset \Theta(n^k \log n) \subset \dots \subset \Theta(a^n) \subset \Theta(b^n) \subset \dots $$
Ordinando le funzioni in per ordine crescente:
$1$ (costante), $4^5$ (costante) , $\log n$, $\log^2 n$, $n^{1/2}$ (o $\sqrt{n}$), $n$, $n \log n$, $n^4 - 7n^3 (\sim n^4)$, $n^5 - 5n^2 (\sim n^5)$, $2^n$, $3^n$. [cite: 65]
\end{example}
\newpage
% ===================================================================
% FILE: Lezione5.tex
% ===================================================================

\part{Lezione 21 (06/11/2025) }

\section{Paradigma Divide et Impera }
Il paradigma "Divide et Impera" (Dividi e Conquista) è una tecnica per progettare algoritmi, tipicamente ricorsivi, che si articola in tre fasi:

\begin{definition}[Paradigma Divide et Impera]
    \begin{enumerate}
        \item \textbf{DIVIDE}: Il problema di dimensione $n$ viene suddiviso in $a$ sottoproblemi dello stesso tipo, ma di dimensione minore ($n/b$).
        \item \textbf{IMPERA}: I sottoproblemi vengono risolti. Se sono abbastanza piccoli (casi base), vengono risolti direttamente. Altrimenti, vengono risolti ricorsivamente con la stessa tecnica.
        \item \textbf{COMBINE}: Le soluzioni degli $a$ sottoproblemi vengono combinate per ottenere la soluzione del problema originale.
    \end{enumerate}
\end{definition}

\subsection{Diagramma Concettuale }
\begin{center}
    \begin{tikzpicture}[
        node distance=2cm and 1.5cm,
        block/.style={draw, rectangle, minimum height=1cm, minimum width=1.5cm, text centered, font=\small},
        dot/.style={node distance=1.5cm and 1cm}
        ]

        \node[block] (P) {$P, n$ (Problema) };
        \node[block] (P1) [below left of=P, yshift=-1cm] {$P_1, n_1$ };
        \node (P_dots) [right=of P1, style=dot] {$\dots$};
        \node[block] (Pa) [right=of P_dots, style=dot] {$P_a, n_a$ };

        \node[block] (S1) [below=of P1] {$S_1, n_1$ };
        \node (S_dots) [right=of S1, style=dot] {$\dots$};
        \node[block] (Sa) [right=of S_dots, style=dot] {$S_a, n_a$ };
        \node[block] (S) [below right of=S_dots, yshift=-1cm] {$S, n$ (Soluzione) };

        \draw[->, thick] (P.south) -- (P1.north) node [midway, left, xshift=-5mm] {\textbf{DIVIDE} };
        \draw[->, thick] (P.south) -- (P_dots.north);
        \draw[->, thick] (P.south) -- (Pa.north);

        \draw[->, thick, decorate, decoration={snake, segment length=8mm, amplitude=1mm}] (P1.south) -- (S1.north) node [midway, left, xshift=-5mm] {\textbf{"IMPERA"} };
        \draw[->, thick, decorate, decoration={snake, segment length=8mm, amplitude=1mm}] (P_dots.south) -- (S_dots.north);
        \draw[->, thick, decorate, decoration={snake, segment length=8mm, amplitude=1mm}] (Pa.south) -- (Sa.north);

        \draw[->, thick] (S1.south) -- (S.west);
        \draw[->, thick] (S_dots.south) -- (S.north);
        \draw[->, thick] (Sa.south) -- (S.east) node [midway, below, xshift=5mm, yshift=-5mm] {\textbf{COMBINE} };
    \end{tikzpicture}
\end{center}

\section{Analisi Complessità D\&I}

\begin{definition}[Analisi Complessità D\&I]
    Sia $T(n)$ il costo per risolvere un problema di dimensione $n$.
    Sia $D(n)$ il costo della fase DIVIDE.
    Sia $C(n)$ il costo della fase COMBINE.
    L'equazione di ricorrenza generale è:
    $$ T(n) = \sum_{i=1}^{a} T(n_i) + D(n) + C(n) $$

    Caso particolare: \textbf{Divisione Bilanciata}.
    Il problema è diviso in $a$ sottoproblemi, ognuno di dimensione $n/b$.
    Sia $f(n) = D(n) + C(n)$ il costo di divide e combine.
    $$ T(n) = aT(n/b) + f(n) $$
\end{definition}


\section{Esempio: Ricerca Binaria (D\&I) }

\begin{example}[Ricerca Binaria: Setup]
    \begin{itemize}
        \item \textbf{Input}: $A[p..r]$ ordinato, chiave $k$.
        \item \textbf{Output}: Indice $i$ t.c. $A[i]=k$, o $-1$.
    \end{itemize}
\end{example}

\begin{algorithmic}[1]
    \Procedure{BinarySearch}{A, p, r, k}
        \If{$p > r$} \Comment{Caso Base 1: array vuoto }
            \State \Return $-1$
        \EndIf

        \State $q = \lfloor (p+r)/2 \rfloor$ \Comment{DIVIDE }

        \If{$A[q] == k$} \Comment{IMPERA (Caso Base 2) }
            \State \Return $q$
        \ElsIf{$A[q] > k$} \Comment{IMPERA (Ricorsione) }
            \State \Return \Call{BinarySearch}{A, p, q-1, k}
        \Else
            \State \Return \Call{BinarySearch}{A, q+1, r, k}
        \EndIf

        \Comment{COMBINE: non necessario, costo $\Theta(1)$ }
    \EndProcedure
\end{algorithmic}

\begin{explanation}{Ricerca Binaria}
Ad ogni passo dimezza lo spazio di ricerca ($n \to n/2 \to n/4 \dots$).
\begin{itemize}
    \item Se $A[q] > k$, cerco a sinistra.
    \item Se $A[q] < k$, cerco a destra.
    \item Costo logaritmico $\Theta(\log n)$.
\end{itemize}
\end{explanation}


\begin{observation}[Analisi: Ricerca Binaria]
    \textbf{Analisi Ricorrenza BS: }
    C'è $a=1$ sottoproblema di dimensione $n/b = n/2$.
    $f(n) = D(n) + C(n) = \Theta(1) + \Theta(1) = \Theta(1)$.
    $$ T(n) = \begin{cases} \Theta(1) & \text{se } n \le 1 \\ T(n/2) + \Theta(1) & \text{se } n > 1 \end{cases} $$

    \textbf{Soluzione (Metodo Iterativo)}:
    $T(n) = T(n/2) + c$
    $T(n) = (T(n/4) + c) + c = T(n/4) + 2c$
    $T(n) = (T(n/8) + c) + 2c = T(n/8) + 3c$
    ... dopo $i$ passi...
    $T(n) = T(n/2^i) + i \cdot c$
    Ci si ferma al caso base quando $n/2^i = 1 \implies i = \log_2 n$.
    $T(n) = T(1) + c \cdot \log_2 n = \Theta(1) + \Theta(\log n) = \Theta(\log n)$.
\end{observation}

\section{Esempio: Minimo/Massimo (D\&I) }

\begin{example}[Minimo/Massimo: Setup]
    \begin{itemize}
        \item \textbf{Input}: $A[1..n]$.
        \item \textbf{Output}: Coppia $\langle min, max \rangle$ di A.
    \end{itemize}
\end{example}

\begin{algorithmic}[1]
    \Procedure{MinMax}{A, p, r}
        \If{$r - p \le 1$} \Comment{Caso Base: 1 o 2 elementi }
            \If{$A[p] \le A[r]$}
                \State \Return $\langle A[p], A[r] \rangle$
            \Else
                \State \Return $\langle A[r], A[p] \rangle$
            \EndIf
        \Else
            \State $q = \lfloor (p+r)/2 \rfloor$ \Comment{DIVIDE}
            \State $\langle min_1, max_1 \rangle = \Call{MinMax}{A, p, q}$ \Comment{IMPERA }
            \State $\langle min_2, max_2 \rangle = \Call{MinMax}{A, q+1, r}$ \Comment{IMPERA }
            \State $min = \min(min_1, min_2)$ \Comment{COMBINE }
            \State $max = \max(max_1, max_2)$
            \State \Return $\langle min, max \rangle$
        \EndIf
    \EndProcedure
\end{algorithmic}

\begin{explanation}{Minimo e Massimo Simultanei}
Per trovare min e max con meno confronti ($3 \lfloor n/2 \rfloor$ invece di $2n$):
\begin{itemize}
    \item Divide l'array in due metà.
    \item Risolve ricorsivamente.
    \item Combina confrontando i minimi tra loro e i massimi tra loro.
\end{itemize}
\end{explanation}


\begin{observation}[Analisi: Minimo/Massimo]
    \textbf{Analisi Ricorrenza MinMax:}
    $a=2$ sottoproblemi, $n/b = n/2$.
    $f(n) = D(n) \text{ (cost.)} + C(n) \text{ (2 confr.)} = \Theta(1)$.
    $$ T(n) = \begin{cases} \Theta(1) & \text{se } n \le 2 \\ 2T(n/2) + \Theta(1) & \text{se } n \ge 3 \end{cases} $$
    (Questa ricorrenza si risolve in $T(n) = \Theta(n)$).
\end{observation}

\newpage

% ===================================================================
% FILE: Lezione6.tex
% ===================================================================

\part{Lezione 22 (10/11/2025) }

\section{Mergesort }
Mergesort è un algoritmo di ordinamento basato su Divide et Impera .
\begin{definition}[Mergesort: Paradigma D\&I]
\begin{itemize}
    \item \textbf{DIVIDE} : Divide l'array $A[p..r]$ in due metà, $A[p..q]$ e $A[q+1..r]$, dove $q = \lfloor (p+r)/2 \rfloor$ .
    \item \textbf{IMPERA} : Ordina ricorsivamente le due metà chiamando `Mergesort(A, p, q)` e `Mergesort(A, q+1, r)` .
    \item \textbf{COMBINE} : Combina (fonde) i due sottoarray ordinati $A[p..q]$ e $A[q+1..r]$ in un unico array ordinato $A[p..r]$ tramite la procedura `Merge(A, p, q, r)` .
\end{itemize}
\end{definition}

\subsection{Pseudocodice Mergesort}
\begin{algorithmic}[1]
\Procedure{Mergesort}{A, p, r} 
    \If{$p < r$} 
        \State $q = \lfloor (p+r)/2 \rfloor$ \Comment{DIVIDE }
        \State \Call{Mergesort}{A, p, q} \Comment{IMPERA }
        \State \Call{Mergesort}{A, q+1, r} \Comment{IMPERA }
        \State \Call{Merge}{A, p, q, r} \Comment{COMBINE }
    \EndIf
\EndProcedure
\end{algorithmic}

\subsection{Procedura Merge }

\begin{observation}[Procedura Merge]
La procedura `Merge` fonde due sottoarray contigui $A[p..q]$ e $A[q+1..r]$, che si assumono \textbf{già ordinati}.
Ha complessità \textbf{Lineare} $T(n) = \Theta(n)$ , dove $n = r-p+1$ .
Utilizza due array di appoggio, $L$ e $R$ , e due "sentinelle" ($\infty$) per evitare controlli sull'indice .
\end{observation}

\begin{algorithmic}[1]
\Procedure{Merge}{A, p, q, r} 
    \State $n_1 = q - p + 1$ \Comment{Dim. primo sottoarray }
    \State $n_2 = r - q$ \Comment{Dim. secondo sottoarray }
    \State Crea array $L[1..n_1+1]$ e $R[1..n_2+1]$ 
    
    \Comment{Copia i dati negli array di appoggio}
    \For{$i = 1 \to n_1$}
        \State $L[i] = A[p + i - 1]$ 
    \EndFor
    \For{$j = 1 \to n_2$}
        \State $R[j] = A[q + j]$ 
    \EndFor
    
    \State $L[n_1 + 1] = +\infty$ \Comment{Sentinella }
    \State $R[n_2 + 1] = +\infty$ \Comment{Sentinella }
    
    \State $i = 1$ \Comment{Indice per $L$ }
    \State $j = 1$ \Comment{Indice per $R$ }
    
    \Comment{Fondi L e R nell'array A}
    \For{$k = p \to r$} 
        \If{$L[i] \le R[j]$} 
            \State $A[k] = L[i]$ 
            \State $i = i + 1$ 
        \Else
            \State $A[k] = R[j]$
            \State $j = j + 1$
        \EndIf
    \EndFor
\EndProcedure
\end{algorithmic}

\subsection{Analisi Complessità Mergesort }

\begin{definition}[Analisi Mergesort: Ricorrenza]
\textbf{Equazione di Ricorrenza:}
$a=2$ sottoproblemi, $n/b = n/2$.
$f(n) = D(n) \text{ (cost.)} + C(n) \text{ (Merge)} = \Theta(1) + \Theta(n) = \Theta(n)$.
$$ T(n) = \begin{cases} \Theta(1) & \text{se } n = 1 \text{ } \\ 2T(n/2) + \Theta(n) & \text{se } n > 1 \text{ } \end{cases} $$
\end{definition}

\textbf{Soluzione 1: Albero di Ricorsione} 
L'albero di ricorsione mostra il costo $f(n_i)$ ad ogni livello.
\begin{center}
\begin{tikzpicture}[
    level distance=1.5cm, 
    level 1/.style={sibling distance=5cm},
    level 2/.style={sibling distance=2.5cm},
    level 3/.style={sibling distance=1.2cm},
    level 4/.style={sibling distance=0.8cm},
    every node/.style={draw, circle, inner sep=1pt, minimum size=6mm, font=\small},
    ]
  % Albero
  \node (z){$n$}
    child {node {$n/2$} 
      child {node {$n/4$} 
        child {node[draw=none] {$\vdots$}
            child{node {$1$} }
        }
      } 
      child {node {$n/4$}
        child {node[draw=none] {$\vdots$}
            child{node {$1$}}
        }
      }
    }
    child {node {$n/2$} 
      child {node {$n/4$} 
         child {node[draw=none] {$\vdots$}
            child{node {$1$}}
         }
      } 
      child {node {$n/4$}
         child {node[draw=none] {$\vdots$}
            child{node {$1$}}
         }
      }
    };
  % Costi
    \node[right=of z, xshift=5cm, draw=none, font=\small] (l0) {Costo: $cn$ };
    \node[right=of z, xshift=5cm, yshift=-1.5cm, draw=none, font=\small] (l1) {Costo: $2 \cdot c(n/2) = cn$ };
    \node[right=of z, xshift=5cm, yshift=-3cm, draw=none, font=\small] (l2) {Costo: $4 \cdot c(n/4) = cn$ };
    \node[right=of z, xshift=5cm, yshift=-4.5cm, draw=none, font=\small] (l3) {$\vdots$};
    \node[right=of z, xshift=5cm, yshift=-6cm, draw=none, font=\small] (l4) {Costo (foglie): $n \cdot c(1) = \Theta(n)$ };
  % Altezza
    \draw[decorate, decoration={brace, amplitude=10pt}] (l0.north west) ++ (-12, 0.5) -- (l4.south west) ++ (-12, -0.5) node [midway, left, xshift=-10pt, font=\small] {Altezza $\log_2 n$};
    \node[below=of l4, yshift=-1cm, draw=none, font=\small] {Totale: $\sum_{i=0}^{\log_2 n - 1} cn + \Theta(n) = cn \log_2 n + \Theta(n) = \Theta(n \log n)$};
\end{tikzpicture}
\end{center}
Il costo per ogni livello è $cn$ . L'albero ha $\log_2 n$ livelli.
Il costo totale è $cn \cdot \log_2 n = \Theta(n \log n)$ .
\begin{observation}[Analisi Mergesort: Metodo Iterativo]
\textbf{Soluzione 2: Metodo Iterativo (Sostituzione)} 
$T(n) = 2T(n/2) + cn$ 
$T(n) = 2(2T(n/4) + c(n/2)) + cn = 4T(n/4) + cn + cn = 4T(n/4) + 2cn$ 
$T(n) = 4(2T(n/8) + c(n/4)) + 2cn = 8T(n/8) + cn + 2cn = 8T(n/8) + 3cn$ 
... dopo $i$ passi ...
$T(n) = 2^i T(n/2^i) + i \cdot cn$ 
Ci si ferma al caso base $n/2^i = 1 \implies i = \log_2 n$.
$T(n) = 2^{\log_2 n} T(1) + (\log_2 n) \cdot cn$ 
$T(n) = n \cdot \Theta(1) + cn \log_2 n = \Theta(n \log n)$ .
\end{observation}


\begin{observation}[Complessità in Spazio: Mergesort]
Mergesort \textbf{non} ordina "in loco" , poiché richiede $\Theta(n)$ spazio ausiliario per gli array $L$ e $R$ ad ogni chiamata di `Merge` .
\end{observation}

\subsection{Esempio: Albero delle Chiamate }
Per $A[1..7]$ , l'ordine delle chiamate ricorsive è :
\begin{center}
\begin{tikzpicture}[
    level distance=1.3cm,
    level 1/.style={sibling distance=4cm},
    level 2/.style={sibling distance=2cm},
    level 3/.style={sibling distance=1.5cm},
    every node/.style={align=center, font=\small}
    ]
  \node {$MS(1,7)$ \\ $n=7$ }
    child {node {$MS(1,4)$ \\ $n=4$ }
      child {node {$MS(1,2)$ \\ $n=2$ }
        child {node {$MS(1,1)$ \\ $n=1$ }}
        child {node {$MS(2,2)$ \\ $n=1$ }}
      } 
      child {node {$MS(3,4)$ \\ $n=2$ }
        child {node {$MS(3,3)$ \\ $n=1$ }}
        child {node {$MS(4,4)$ \\ $n=1$ }}
      }
    }
    child {node {$MS(5,7)$ \\ $n=3$ }
      child {node {$MS(5,6)$ \\ $n=2$ }
        child {node {$MS(5,5)$ \\ $n=1$ }}
        child {node {$MS(6,6)$ \\ $n=1$ }}
      } 
      child {node {$MS(7,7)$ \\ $n=1$ }}
    };
\end{tikzpicture}
\end{center}

\newpage
% ===================================================================
% FILE: Lezione7.tex
% ===================================================================

\part*{Spiegazione della Lezione 23 (12/10/2025)}
\section*{Introduzione: Relazioni di Ricorrenza}

Questi appunti della Lezione 23 affrontano un argomento cruciale nell'analisi degli algoritmi: le \textbf{Relazioni di Ricorrenza}.
In breve, queste sono equazioni matematiche usate per descrivere il tempo di esecuzione, $T(n)$, di un algoritmo che chiama sé stesso (cioè un algoritmo ricorsivo).

\subsection*{Tipi di Relazioni di Ricorrenza}
Gli appunti ne identificano tre tipi principali.


\begin{definition}[Relazioni Bilanciate (Divide et Impera)]
    Sono le più comuni negli algoritmi "Divide et Impera" (come Mergesort).
    Hanno una forma specifica:
    \[
    T(n) = aT\left(\frac{n}{b}\right) + f(n)
    \]
    \begin{itemize}
        \item \textbf{$a$} è il numero di sotto-problemi in cui dividiamo il problema principale.
        \item \textbf{$n/b$} è la dimensione di ciascun sotto-problema.
        \item \textbf{$f(n)$} è chiamata la "forzante" e rappresenta il lavoro "extra" fatto per dividere e ricombinare i risultati.
    \end{itemize}
\end{definition}

\begin{enumerate}
    \setcounter{enumi}{1}
    \item \textbf{Relazioni di Ordine K:} Queste dipendono dai valori immediatamente precedenti, come $T(n-1)$, $T(n-2)$, ecc. (es. Fibonacci).
    \item \textbf{Caso Generale:} Una forma più complessa dove i sotto-problemi potrebbero non avere dimensioni uguali.
\end{enumerate}

\subsection*{Esempi Concreti di Relazioni Bilanciate}

\begin{example}[Mergesort]
    Per ordinare un array, lo divide in 2 metà ($a=2$), le ordina ricorsivamente (ciascuna di dimensione $n/2$, quindi $b=2$) e poi le fonde (un'operazione che costa $\Theta(n)$).
    La sua relazione è: $T(n) = 2T\left(\frac{n}{2}\right) + \Theta(n)$.
\end{example}

\begin{example}[Ricerca Binaria]
    Per cercare in un array ordinato, fa un confronto, poi chiama ricorsivamente sé stessa su \emph{una} sola metà ($a=1$) di dimensione $n/2$ ($b=2$).
    Il costo del confronto è costante, $\Theta(1)$. La sua relazione è: $T(n) \le T\left(\frac{n}{2}\right) + \Theta(1)$.
\end{example}

\subsection*{Come Risolvere Queste Relazioni?}
Una volta che abbiamo l'equazione, come troviamo la complessità finale?
Gli appunti elencano quattro metodi:

\begin{enumerate}
    \item \textbf{Metodo Iterativo}
    \item \textbf{Metodo di Sostituzione} (Induzione)
    \item \textbf{Albero di Ricorsione} (Metodo grafico)
    \item \textbf{Teorema Principale (Master Theorem)}
\end{enumerate}


\section*{Il Cuore della Lezione: Il Master Theorem}
Il Teorema Principale (Master Theorem) è una "ricetta" che funziona solo per le relazioni bilanciate $T(n) = aT\left(\frac{n}{b}\right) + f(n)$.
L'idea centrale è \textbf{confrontare due "forze"}:
\begin{enumerate}
    \item Il costo della \textbf{ricorsione} (quanti sotto-problemi si creano).
    \item Il costo del \textbf{lavoro extra} $f(n)$ (la "forzante").
\end{enumerate}

\begin{definition}[Il Master Theorem]
    Data $T(n) = aT\left(\frac{n}{b}\right) + f(n)$, si calcola la \textbf{"Funzione Spartiacque"}: \textbf{$n^{\log_b a}$}.
    Confrontando $f(n)$ con $n^{\log_b a}$ si ricade in uno dei tre casi:

    \begin{itemize}
        \item \textbf{Caso 1: $f(n)$ polinomialmente minore} ($f(n) = O(n^{\log_b a - \epsilon})$)
        \begin{itemize}
            \item \textbf{Logica:} Il costo è dominato dalla ricorsione (dalle foglie).
            \item \textbf{Soluzione: $T(n) = \Theta(n^{\log_b a})$}.
        \end{itemize}

        \item \textbf{Caso 2: $f(n)$ circa uguale} ($f(n) = \Theta(n^{\log_b a} \cdot \log^k n)$)
        \begin{itemize}
            \item \textbf{Logica:} Le forze sono bilanciate; il costo è lo stesso ad ogni livello.
            \item \textbf{Soluzione: $T(n) = \Theta(n^{\log_b a} \cdot \log^{k+1} n)$}. (Se $k=0$, la soluzione è $\Theta(n^{\log_b a} \cdot \log n)$).
        \end{itemize}

        \item \textbf{Caso 3: $f(n)$ polinomialmente maggiore} ($f(n) = \Omega(n^{\log_b a + \epsilon})$)
        \begin{itemize}
            \item \textbf{Logica:} Il costo è dominato dal lavoro extra $f(n)$ (il collo di bottiglia).
            \item \textbf{Controllo:} Richiede la "Condizione di Regolarità" ($a f(n/b) \le c f(n)$).
            \item \textbf{Soluzione: $T(n) = \Theta(f(n))$}.
        \end{itemize}
    \end{itemize}
\end{definition}


\subsection*{Applicazioni del Master Theorem}

\begin{example}[Mergesort]
    $T(n) = 2T\left(\frac{n}{2}\right) + \Theta(n)$
    \begin{itemize}
        \item $a=2, b=2$. Spartiacque: $n^{\log_2 2} = n$.
        \item Confronto: $f(n) = \Theta(n)$ è \emph{uguale} allo spartiacque (Caso 2 con $k=0$).
        \item \textbf{Soluzione: $\Theta(n \log n)$}.
    \end{itemize}
\end{example}

\begin{example}[Ricerca Binaria]
    $T(n) = T\left(\frac{n}{2}\right) + \Theta(1)$
    \begin{itemize}
        \item $a=1, b=2$. Spartiacque: $n^{\log_2 1} = n^0 = 1$.
        \item Confronto: $f(n) = \Theta(1)$ è \emph{uguale} allo spartiacque (Caso 2 con $k=0$).
        \item \textbf{Soluzione: $\Theta(1 \cdot \log n) = \Theta(\log n)$}.
    \end{itemize}
\end{example}

\begin{example}[Esempio (Min/Max)]
    $T(n) = 2T\left(\frac{n}{2}\right) + \Theta(1)$
    \begin{itemize}
        \item $a=2, b=2$. Spartiacque: $n^{\log_2 2} = n$.
        \item Confronto: $f(n) = \Theta(1)$ è \emph{polinomialmente minore} di $n$ (Caso 1).
        \item \textbf{Soluzione: $\Theta(n)$}.
    \end{itemize}
\end{example}

\begin{example}[Esempio 1 (dal testo)]
    $T(n) = 9T\left(\frac{n}{3}\right) + n$
    \begin{itemize}
        \item $a=9, b=3$. Spartiacque: $n^{\log_3 9} = n^2$.
        \item Confronto: $f(n) = n$ è \emph{polinomialmente minore} di $n^2$ (Caso 1).
        \item \textbf{Soluzione: $\Theta(n^2)$}.
    \end{itemize}
\end{example}

\begin{example}[Esempio 2 (dal testo)]
    $T(n) \le T\left(\frac{2n}{3}\right) + 1$
    \begin{itemize}
        \item $a=1, b=3/2$. Spartiacque: $n^{\log_{3/2} 1} = n^0 = 1$.
        \item Confronto: $f(n) = 1$ è \emph{uguale} allo spartiacque (Caso 2 con $k=0$).
        \item \textbf{Soluzione: $\Theta(\log n)$}.
    \end{itemize}
\end{example}

\begin{example}[Esempio 3 (dal testo)]
    $T(n) = 3T\left(\frac{n}{4}\right) + n \log n$
    \begin{itemize}
        \item $a=3, b=4$. Spartiacque: $n^{\log_4 3} \approx n^{0.792}$.
        \item Confronto: $f(n) = n \log n$ è \emph{polinomialmente maggiore} (Caso 3).
        \item (Si verifica la condizione di regolarità).
        \item \textbf{Soluzione: $\Theta(f(n)) = \Theta(n \log n)$}.
    \end{itemize}
\end{example}



\begin{observation}[Riepilogo della Lezione]
    Gli appunti della Lezione 23 introducono le \textbf{Relazioni di Ricorrenza} per analizzare $T(n)$ degli algoritmi ricorsivi. Si concentrano sulle \textbf{Relazioni Bilanciate} ($T(n) = aT(n/b) + f(n)$). Dopo aver elencato quattro metodi di risoluzione, si focalizzano sul \textbf{Master Theorem}.

    Il teorema confronta $f(n)$ con lo "spartiacque" $n^{\log_b a}$ e definisce tre casi:
    \begin{enumerate}
        \item \textbf{Caso 1 ($f(n)$ minore):} Soluzione: $\Theta(n^{\log_b a})$.
        \item \textbf{Caso 2 ($f(n)$ uguale):} Soluzione: $\Theta(n^{\log_b a} \cdot \log n)$ (o $\log^{k+1} n$).
        \item \textbf{Caso 3 ($f(n)$ maggiore):} Soluzione: $\Theta(f(n))$ (con C.R.).
    \end{enumerate}
\end{observation}


\section*{ASA (Esercizi per casa)}

\begin{example}[Esercizi ASA]
    Risolvere le seguenti relazioni di ricorrenza:
    \begin{enumerate}
        \item $T(n) = 3T(\frac{n}{2}) + n^2$
        \item $T(n) = 3T(\frac{n}{4}) + \frac{n}{5} \log n$
    \end{enumerate}
\end{example}

\newpage

% ===================================================================
% FILE: Lezione24.tex
% ===================================================================

\part{Lezione 24 (13/11/2025)}

\section{Dimostrazione del Teorema Principale}
L'obiettivo è risolvere la relazione di ricorrenza $T(n) = aT(n/b) + f(n)$.
Si può derivare la formula generale usando l'albero di ricorsione o il metodo iterativo.
\subsection{Metodo Iterativo (Derivazione della Formula)}
Si espande la ricorrenza sostituendo $T(n)$ dentro sé stessa.
% --- APPLICAZIONE STILE ---
\begin{definition}[Formula Generale (Metodo Iterativo)]
    Partiamo dalla ricorrenza:
    \[ T(n) = aT(n/b) + f(n) \]
    Sostituiamo $T(n/b)$ nell'equazione:
    \[ T(n) = a \left[ aT(n/b^2) + f(n/b) \right] + f(n) = a^2 T(n/b^2) + af(n/b) + f(n) \]
    Sostituiamo $T(n/b^2)$ nell'equazione:
    \[ T(n) = a^2 \left[ aT(n/b^3) + f(n/b^2) \right] + af(n/b) + f(n) = a^3 T(n/b^3) + a^2 f(n/b^2) + af(n/b) + f(n) \]

    Dopo $i$ passi, la formula generale è:
    \[ T(n) = a^i T(n/b^i) + \sum_{j=0}^{i-1} a^j f(n/b^j) \]
    Ci si ferma al caso base quando la dimensione del problema è 1, cioè $n/b^i = 1$, che avviene quando $i = \log_b n$.
    Sostituendo $i = \log_b n$:
    \[ T(n) = a^{\log_b n} T(1) + \sum_{j=0}^{\log_b n - 1} a^j f(n/b^j) \]
    Usando l'identità $a^{\log_b n} = n^{\log_b a}$ (dimostrata sotto) e sapendo che $T(1) = \Theta(1)$, la formula finale del costo è:
    \[ T(n) = \Theta(n^{\log_b a}) + \sum_{j=0}^{\log_b n - 1} a^j f(n/b^j) \]
    Questo costo totale è la somma di due parti:
    \begin{itemize}
        \item \textbf{$\Theta(n^{\log_b a})$}: Il costo per la soluzione dei casi base (le foglie dell'albero).
        \item \textbf{$\sum a^j f(n/b^j)$}: Il costo totale del lavoro di "Divide" e "Combine" speso a tutti i livelli della ricorsione.
    \end{itemize}
\end{definition}

% --- APPLICAZIONE STILE ---
\begin{observation}[Identità delle Foglie: $a^{\log_b n} = n^{\log_b a}$]
    \textbf{Dimostrazione:}
    Si parte da $a^{\log_b n}$.
    Si applica la proprietà $x = n^{\log_n x}$:
    \[ a^{\log_b n} = (n^{\log_n a})^{\log_b n} \]
    Si applica la formula del cambio di base $\log_n a = \frac{\log_b a}{\log_b n}$:
    \[ a^{\log_b n} = \left( n^{\frac{\log_b a}{\log_b n}} \right)^{\log_b n} \]
    Moltiplicando gli esponenti:
    \[ a^{\log_b n} = n^{\frac{\log_b a}{\log_b n} \cdot \log_b n} = n^{\log_b a} \]
\end{observation}

\subsection{Analisi dei Casi del Teorema}
L'analisi consiste nel determinare quale dei due termini della formula $T(n) = \Theta(n^{\log_b a}) + \sum...$ domina.
% --- APPLICAZIONE STILE ---
\begin{definition}[Caso 1: $f(n)$ polinomialmente minore]
    \begin{itemize}
        \item \textbf{Condizione:} $f(n) \in O(n^{\log_b a - \epsilon})$ per $\epsilon > 0$.
        \item \textbf{Analisi:} La sommatoria $\sum_{j=0}^{\log_b n - 1} a^j f(n/b^j)$ può essere analizzata come una serie geometrica.
        Sostituendo la condizione, si dimostra che la somma cresce più lentamente del primo termine (la ragione della serie è $r = b^\epsilon > 1$).
        \item \textbf{Logica:} Il costo è dominato dal lavoro svolto nei casi base (le foglie).
        \item \textbf{Soluzione:} $T(n) \in \Theta(n^{\log_b a})$.
    \end{itemize}
\end{definition}

% --- APPLICAZIONE STILE ---
\begin{definition}[Caso 2: $f(n)$ bilanciato (caso $k=0$)]
    \begin{itemize}
        \item \textbf{Condizione:} $f(n) = \Theta(n^{\log_b a})$.
        \item \textbf{Analisi:} Partiamo dalla formula $T(n) = \Theta(n^{\log_b a}) + \sum_{j=0}^{\log_b n - 1} a^j f(n/b^j)$.
        Sostituiamo la condizione $f(n/b^j) = \Theta((n/b^j)^{\log_b a})$ nella sommatoria:
        \[ \sum_{j=0}^{\log_b n - 1} a^j \cdot \left(\frac{n}{b^j}\right)^{\log_b a} = \sum_{j=0}^{\log_b n - 1} a^j \cdot \frac{n^{\log_b a}}{(b^{\log_b a})^j} = \sum_{j=0}^{\log_b n - 1} a^j \cdot \frac{n^{\log_b a}}{a^j} \]
        Semplificando $a^j$, otteniamo:
        \[ \sum_{j=0}^{\log_b n - 1} n^{\log_b a} = n^{\log_b a} \sum_{j=0}^{\log_b n - 1} 1 = n^{\log_b a} \cdot (\log_b n) \]
        \item \textbf{Logica:} Il costo del lavoro extra è bilanciato con il costo delle foglie.
        Il costo totale è il costo di un livello ($n^{\log_b a}$) moltiplicato per il numero di livelli ($\log n$).
        \item \textbf{Soluzione:} $T(n) = \Theta(n^{\log_b a}) + \Theta(n^{\log_b a} \cdot \log n) = \Theta(n^{\log_b a} \cdot \log n)$.
    \end{itemize}
\end{definition}

% --- INIZIO GRAFICO RICOSTRUITO ---
\begin{figure}[h!]
    \centering
    \begin{tikzpicture}[
        level distance=2cm,
        sibling distance=4cm,
        level 1/.style={sibling distance=5cm},
        level 2/.style={sibling distance=2.5cm},
        level 3/.style={sibling distance=1.5cm, font=\small},
        level 4/.style={font=\tiny},
        every node/.style={align=center}
        ]

        % --- L'ALBERO ---
        % Livello 0 (Radice)
        \node (root) {$cn^2$}
        % Livello 1
        child { node (l1_1) {$c(\frac{n}{4})^2$}
        % Livello 2
        child { node (l2_1) {$c(\frac{n}{16})^2$}
        % Livello 3 (Puntini)
        child { node[draw=none] (l3_1) {$\vdots$}
        % Livello 4 (Foglie)
        child { node (l4_1) {$T(1)$} }
        }
        }
        child { node (l2_2) {$c(\frac{n}{16})^2$}
        child { node[draw=none] (l3_2) {$\vdots$}
        child { node (l4_2) {$T(1)$} }
        }
        }
        child { node (l2_3) {$c(\frac{n}{16})^2$}
        child { node[draw=none] (l3_3) {$\vdots$}
        child { node (l4_3) {$T(1)$} }
        }
        }
        }
        child { node (l1_2) {$c(\frac{n}{4})^2$}
        child { node (l2_4) {$c(\frac{n}{16})^2$}
        child { node[draw=none] (l3_4) {$\vdots$}
        child { node (l4_4) {$T(1)$} }
        }
        }
        child { node (l2_5) {$c(\frac{n}{16})^2$}
        child { node[draw=none] (l3_5) {$\vdots$}
        child { node (l4_5) {$T(1)$} }
        }
        }
        child { node (l2_6) {$c(\frac{n}{16})^2$}
        child { node[draw=none] (l3_6) {$\vdots$}
        child { node (l4_6) {$T(1)$} }
        }
        }
        }
        child { node (l1_3) {$c(\frac{n}{4})^2$}
        child { node (l2_7) {$c(\frac{n}{16})^2$}
        child { node[draw=none] (l3_7) {$\vdots$}
        child { node (l4_7) {$T(1)$} }
        }
        }
        child { node (l2_8) {$c(\frac{n}{16})^2$}
        child { node[draw=none] (l3_8) {$\vdots$}
        child { node (l4_8) {$T(1)$} }
        }
        }
        child { node (l2_9) {$c(\frac{n}{16})^2$}
        child { node[draw=none] (l3_9) {$\vdots$}
        child { node (l4_9) {$T(1)$} }
        }
        }
        };
        % --- PUNTINI CENTRALI TRA LE FOGLIE ---
        \node[right=of l4_5, node distance=1.5cm, draw=none] (dots_mid) {$\dots$};
        \node[right=of l4_6, node distance=1.5cm, draw=none] (dots_mid2) {$\dots$};

        % --- ANNOTAZIONI A DESTRA (COSTI PER LIVELLO) ---
        \node[right=of root, node distance=6cm, draw=none] (cost0) {$cn^2$};
        \node[right=of l1_3, node distance=3.5cm, draw=none] (cost1) {$\frac{3}{16} cn^2$};
        \node[right=of l2_9, node distance=2.5cm, draw=none] (cost2) {$(\frac{3}{16})^2 cn^2$};
        \node[below=of cost2, draw=none, node distance=1cm] (cost_dots) {$\vdots$};

        \draw[dotted, thick] (root.east) -- (cost0.west);
        \draw[dotted, thick] (l1_3.east) -- (cost1.west);
        \draw[dotted, thick] (l2_9.east) -- (cost2.west);

        % --- ANNOTAZIONI IN BASSO (COSTO FOGLIE E CONTEGGIO) ---
        \node[below=of l4_5, node distance=1.5cm, draw=none] (leaves_cost) {$\Theta(n^{\log_4 3})$};
        \draw [decorate, decoration={brace, amplitude=10pt, mirror}] (l4_1.south west) -- (l4_9.south east)
        node [midway, below, yshift=-10pt] {$n^{\log_4 3}$ foglie};
        % --- ANNOTAZIONE A SINISTRA (ALTEZZA) ---
        \draw[<->, thick] (root.north west) ++ (-6.5cm, 0.5cm) -- (l4_1.south west) ++ (-1.5cm, -0.5cm)
        node [midway, left, xshift=-5pt] {$\log_4 n$};
        % --- TOTALE ---
        \node[below=of leaves_cost, node distance=1.5cm, right=of leaves_cost, xshift=6cm, draw=none, font=\Large] (total) {Totale: $O(n^2)$};
    \end{tikzpicture}
    \caption{Visualizzazione dell'albero di ricorsione per $T(n) = 3T(n/4) + cn^2$.
    Questo è un esempio del \textbf{Caso 3} del Master Theorem, dove il costo è dominato dalla radice (root).}
    \label{fig:master-tree-case3}
\end{figure}
% --- FINE GRAFICO ---


\section{Esercizio (Compitino 24-25)}
Analizzare la complessità di un algoritmo la cui struttura (semplificata) è la seguente, ipotizzando diversi costi per il lavoro $f(n)$.
% --- APPLICAZIONE STILE ---
\begin{example}[Analisi Algoritmo Ricorsivo]
    Dato il seguente algoritmo:
    \begin{algorithmic}[1]
        \Procedure{ALGO}{A, p, r}
            \If{$p < r$}
                \State $q = \lfloor (p+r)/2 \rfloor$
                \State \Call{ALGO}{A, p, q} \Comment{Costo $T(n/2)$}
                \State \Call{ALGO}{A, q+1, r} \Comment{Costo $T(n/2)$}
                \State \Call{ALGO}{A, p, q} \Comment{Costo $T(n/2)$}
                \State \Call{ALGO}{A, q+1, r} \Comment{Costo $T(n/2)$}
                \State... (Lavoro extra con costo $f(n)$)...
            \EndIf
        \EndProcedure
    \end{algorithmic}

    \begin{explanation}{Analisi Ricorsiva}
    L'algoritmo divide il problema in 2 metà ($n/2$) ma effettua \textbf{4 chiamate ricorsive} ($a=4$).
    Questo porta a un costo elevato che domina il lavoro locale $f(n)$, a meno che $f(n)$ non sia molto pesante.
    \end{explanation}

    L'algoritmo fa $a=4$ chiamate ricorsive su sottoproblemi di dimensione $n/2$ (quindi $b=2$).
    % --- APPLICAZIONE STILE ---
    \begin{observation}[Caso Pessimo: $f(n) = n^2$]
        La relazione di ricorrenza è: $T(n) = 4T(n/2) + n^2$.
        \begin{itemize}
            \item $a = 4$, $b = 2$.
            \item \textbf{Spartiacque:} $n^{\log_b a} = n^{\log_2 4} = n^2$.
            \item \textbf{Confronto:} $f(n) = n^2$ è uguale allo spartiacque.
            \item Siamo nel \textbf{Caso 2} (con $k=0$).
            \item \textbf{Soluzione:} $T(n) = \Theta(n^{\log_b a} \cdot \log n) = \Theta(n^2 \cdot \log n)$.
        \end{itemize}
    \end{observation}

    % --- APPLICAZIONE STILE ---
    \begin{observation}[Caso Ottimo (ipotetico): $f(n) = n$]
        La relazione di ricorrenza è: $T(n) = 4T(n/2) + n$.
        \begin{itemize}
            \item $a = 4$, $b = 2$. \textbf{Spartiacque:} $n^2$.
            \item \textbf{Confronto:} $f(n) = n$ è polinomialmente minore di $n^2$ (poiché $n = O(n^{2-\epsilon})$ per $\epsilon=1$).
            \item Siamo nel \textbf{Caso 1}.
            \item \textbf{Soluzione:} $T(n) = \Theta(n^{\log_b a}) = \Theta(n^2)$.
        \end{itemize}
    \end{observation}

\end{example}
\newpage

% ===================================================================
% FILE: Precizazione1.tex
% ===================================================================

\part {Master's Theorem}
Quando si tratta di risolvere equazioni di ricorrenza \textbf{bilanciate}, è possibile utilizzare il Master's Theorem.
\begin{equation}
    T(n)=\begin{cases}
        \Theta(1) & n \leq k \\
        a \cdot T(\frac{n}{b}) + f(n) & n >k
    \end{cases}
\end{equation}
L'intuizione consiste nel fare un confronto tra $f(n)$ e $n^{\log_b{a}}$.
% --- APPLICAZIONE STILE ---
\begin{definition}[Master Theorem: I Tre Casi]
Ci sono tre casi possibili:
\begin{itemize}
    \item \textbf{Minore}: $f(n) = O(n^{\log_b{a}-\epsilon})$ per qualche costante $\epsilon > 0$.
    $f(n)$ cresce \textbf{polinomialmente} più lentamente di $n^{\log_b{a}}$.
    \emph{Soluzione}: $T(n) = \Theta(n^{\log_b{a}})$.
    
    \begin{example}
        Data la seguente equazione di ricorrenza:
        \begin{equation}
            T(n) = 9 \cdot T(\frac{n}{3}) + n
        \end{equation}
        Abbiamo che $a=9$, $b=3$, $f(n) = n$, $n^{\log_3 9} = n^2$.Possiamo dedurre quindi che, per un $\epsilon = 1$:
        \begin{equation}
            f(n)  = n = O(n^{\log_3 9 - \epsilon}) = O(n)
        \end{equation}
    \end{example}
    
    \item \textbf{Uguale}: $f(n) = \Theta(n^{\log_b{a}}\cdot \ln^k{n})$ per qualche costante $k \geq 0$.
    $f(n)$ e $n^{\log_b{a}}$ crescono allo stesso modo.
    \emph{Soluzione}: $T(n) = \Theta(n^{\log_b{a}} \cdot \ln^{k+1}{n})$.
    \item \textbf{Maggiore}: $f(n) = \Omega(n^{\log_b{a} + \epsilon})$ per qualche costante $\epsilon > 0$.
    $f(n)$ cresce \textbf{polinomialmente} più in fretta e rispetta la \textbf{condizione di regolarità}: $a \cdot f(\frac{n}{b}) \leq c \cdot f(n)$ per $c<1$.
    \emph{Soluzione}: $T(n) = \Theta(f(n))$.
\end{itemize}
\end{definition}


\begin{observation}
    Il Master's Theorem si può usare solamente quando $f(x)$ cresce \textbf{polinomialmente} più in fretta o lentamente di $n^{\log_b{a}}$.
\end{observation}


\newpage
\section{Guida Pratica all'Applicazione del Master's Theorem}

Il Teorema Master è uno strumento potente per risolvere equazioni di ricorrenza \textbf{bilanciate}.
La forma standard è:
\[
    T(n) = a \cdot T(\frac{n}{b}) + f(n)
\]

% --- APPLICAZIONE STILE ---
\begin{observation}[Logica: "Tradurre" l'Equazione]
Per usare il teorema, devi prima "tradurre" la tua equazione:
\begin{itemize}
    \item \textbf{$a$}: Il numero di \textbf{sotto-problemi} ($a \geq 1$).
    \item \textbf{$n/b$}: La dimensione di \textbf{ciascun sotto-problema} ($b > 1$).
    \item \textbf{$f(n)$}: Il costo "extra" per \textbf{dividere} e \textbf{combinare}.
\end{itemize}
L'idea centrale è confrontare $f(n)$ con $n^{\log_b{a}}$.
\end{observation}

\paragraph{Come Applicarlo in un Esercizio: Passo Passo}

\begin{example}[Guida Passo-Passo: $T(n) = 9 \cdot T(\frac{n}{3}) + n$]
    
    \subparagraph{Passo 1: Identificare $a$, $b$, e $f(n)$}
    Dall'equazione $T(n) = 9 \cdot T(\frac{n}{3}) + n$:
    \begin{itemize}
        \item $a = 9$ (sotto-problemi)
        \item $b = 3$ (dimensione 1/3)
        \item $f(n) = n$ (costo extra)
    \end{itemize}
    
    
    
    \subparagraph{Passo 2: Calcolare il "Confine" Ricorsivo}
    Calcola il valore $n^{\log_b{a}}$.
    \begin{itemize}
        \item $n^{\log_3 9} = n^2$
    \end{itemize}
    
    \subparagraph{Passo 3: Confrontare $f(n)$ con $n^{\log_b{a}}$}
    Confrontiamo $f(n) = n$ con $n^2$.
    \paragraph{I Tre Casi del Teorema}
    \subparagraph{Caso 1: $f(n)$ cresce più LENTAMENTE}
    \begin{itemize}
        \item \textbf{Logica:} Il costo è dominato dalla ricorsione.
        \item \textbf{Condizione:} $f(n) = O(n^{\log_b{a}-\epsilon})$ per $\epsilon > 0$.
        \item \textbf{Soluzione:} $T(n) = \Theta(n^{\log_b{a}})$.
    \end{itemize}
    
    \subparagraph{Verifica del nostro Esempio:}
    $f(n) = n$ e $n^{\log_b{a}} = n^2$.
    $f(n) = n$ è $O(n^{2-\epsilon})$ scegliendo $\epsilon=1$.
    Rientriamo nel \textbf{Caso 1}.
    \emph{Soluzione:} $T(n) = \Theta(n^2)$.
    
    \subparagraph{Caso 2: $f(n)$ cresce alla STESSA VELOCITÀ}
    \begin{itemize}
        \item \textbf{Logica:} Costo bilanciato.
        \item \textbf{Condizione:} $f(n) = \Theta(n^{\log_b{a}}\cdot \ln^k{n})$ per $k \geq 0$.
        \item \textbf{Soluzione:} $T(n) = \Theta(n^{\log_b{a}} \cdot \ln^{k+1}{n})$.
    \end{itemize}
    
    \subparagraph{Caso 3: $f(n)$ cresce più VELOCEMENTE}
    \begin{itemize}
        \item \textbf{Logica:} Costo dominato dal lavoro extra $f(n)$.
        \item \textbf{Condizione 1:} $f(n) = \Omega(n^{\log_b{a} + \epsilon})$ per $\epsilon > 0$.
        \item \textbf{Condizione 2 (Regolarità):} $a \cdot f(\frac{n}{b}) \leq c \cdot f(n)$ per $c < 1$.
        \item \textbf{Soluzione:} $T(n) = \Theta(f(n))$.
    \end{itemize}
\end{example}

\begin{observation}[Limiti del Teorema]
Il Teorema Master \textbf{non si può usare} se $f(n)$ non cresce \emph{polinomialmente} più velocemente o più lentamente di $n^{\log_b{a}}$.
Ad esempio, se $f(x) = \log{n}$ non potremmo utilizzarlo (perché non è polinomialmente diverso da $n^0=1$).
\end{observation}

\newpage



% ... ecc.

\end{document}